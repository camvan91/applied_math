\flushbottom


%%
%% Present the canonical forms in Zwillinger.
%%


%%============================================================================
%%============================================================================
\chapter{Transformations and Canonical Forms}
\index{transformations!of differential equations}
\index{canonical forms!of differential equations}



Prize intensity more than extent.  Excellence resides in quality not in 
quantity.  The best is always few and rare - abundance lowers value.
Even among men, the giants are usually really dwarfs.  Some reckon books
by the thickness, as if they were written to exercise the brawn more
than the brain.  Extent alone never rises above mediocrity; it is the 
misfortune of universal geniuses that in attempting to be at home
everywhere are so nowhere.  Intensity gives eminence and rises to the heroic 
in matters sublime.

\begin{flushright}
  -Balthasar Gracian
\end{flushright}






%%=============================================================================
\section{The Constant Coefficient Equation}
\index{canonical forms!constant coefficient equation}


The solution of any second order linear homogeneous differential equation
can be written in terms of the solutions to either
\[
y'' = 0, \quad \mathrm{or} \quad y'' - y = 0
\]

Consider the general equation
\[
y'' + a y' + b y = 0.
\]
We can solve this differential equation by making the substitution
$y = \e^{\lambda x}$.  This yields the algebraic equation
\[
\lambda^2 + a \lambda + b = 0.
\]
\[
\lambda = \frac{1}{2}\left( -a \pm \sqrt{a^2 - 4b} \right)
\]
There are two cases to consider.  If $a^2 \neq 4b$ then the solutions are
\[
y_1 = \e^{(-a + \sqrt{a^2 - 4b}) x / 2}, \qquad
y_2 = \e^{(-a - \sqrt{a^2 - 4b}) x / 2}
\]
If $a^2 = 4b$ then we have
\[
y_1 = \e^{-a x / 2}, \qquad y_2 = x \e^{-a x / 2}
\]
Note that regardless of the values of $a$ and $b$ the solutions are of the 
form
\[
y = \e^{-a x / 2} u(x)
\]



We would like to write the solutions to the general differential equation
in terms of the solutions to simpler differential equations.
We make the substitution
\[
y = \e^{\lambda x} u
\]
The derivatives of $y$ are
\begin{align*}
  y'      &= \e^{\lambda x} (u' + \lambda u) \\
  y''     &= \e^{\lambda x} (u'' + 2 \lambda u' + \lambda^2 u)
\end{align*}
Substituting these into the differential equation yields
\[
u'' + (2 \lambda + a) u' + (\lambda^2 + a \lambda + b) u = 0
\]
In order to get rid of the $u'$ term we choose
\[
\lambda = - \frac{a}{2}.
\]
The equation is then
\[
u'' + \left( b - \frac{a^2}{4} \right) u = 0.
\]
There are now two cases to consider.

\paragraph{Case 1.}
If $b = a^2 / 4$ then the differential equation is
\[
u'' = 0
\]
which has solutions $1$ and $x$.  The general solution for $y$ is then
\[
y = \e^{-ax / 2} (c_1 + c_2 x).
\]


\paragraph{Case 2.}
If $b \neq a^2 / 4$ then the differential equation is
\[
u'' - \left( \frac{a^2}{4} - b\right) u = 0.
\]
We make the change variables
\[
u(x) = v(\xi), \qquad x = \mu \xi.
\]
The derivatives in terms of $\xi$ are
\begin{align*}
  \frac{\dd}{\dd x} &= \frac{\dd \xi}{\dd x} \frac{\dd}{\dd \xi} = \frac{1}{\mu}\frac{\dd}{\dd \xi} \\
  \frac{\dd^2}{\dd x^2} &=  \frac{1}{\mu}\frac{\dd}{\dd \xi}  \frac{1}{\mu}\frac{\dd}{\dd \xi}
  = \frac{1}{\mu^2} \frac{\dd^2}{\dd \xi^2}.
\end{align*}
The differential equation for $v$ is
\[
\frac{1}{\mu^2} v'' - \left( \frac{a^2}{4} - b\right) v = 0
\]
\[
v'' - \mu^2 \left( \frac{a^2}{4} - b\right) v = 0
\]
We choose
\[
\mu = \left( \frac{a^2}{4} - b\right)^{-1/2}
\]
to obtain
\[
v'' - v = 0
\]
which has solutions $\e^{\pm \xi}$.
The solution for $y$ is
\[
y = \e^{\lambda x} \left( c_1 \e^{x/\mu} + c_2 \e^{-x / \mu}\right)
\]
\[
y = \e^{-ax/2} \left( c_1 \e^{\sqrt{a^2 /4 - b}\ x}
  + c_2 \e^{-\sqrt{a^2 / 4 - b}\ x} \right)
\]








%%==============================================================================
\section{Normal Form}
\index{normal form!of differential equations}

%%-----------------------------------------------------------------------------
\subsection{Second Order Equations}

Consider the second order equation
\begin{equation}
  \label{can_nf_soe}
  y'' + p(x) y' + q(x) y = 0.
\end{equation}
Through a change of dependent variable, this equation can be transformed
to 
\[
u'' + I(x) y = 0.
\]
This is known as the \textbf{normal form} of (\ref{can_nf_soe}).
The function $I(x)$ is known as the \textbf{invariant} of the equation.

Now to find the change of variables that will accomplish this transformation.
We make the substitution $y(x) = a(x) u(x)$ in (\ref{can_nf_soe}).
\[
a u'' + 2 a' u' + a'' u + p (a u' + a' u) + q a u = 0
\]
\[
u'' + \left( 2 \frac{a'}{a} + p \right) u' + \left( \frac{a''}{a} 
  + \frac{p a'}{a} + q \right) u = 0
\]
To eliminate the $u'$ term, $a(x)$ must satisfy
\[
2 \frac{a'}{a} + p = 0
\]
\[
a' + \frac{1}{2} p a = 0
\]
\[
a = c \exp\left( - \frac{1}{2} \int p(x) \,\dd x \right).
\]
For this choice of $a$, our differential equation for $u$ becomes
\[
u'' + \left( q - \frac{p^2}{4} - \frac{p'}{2} \right) u = 0.
\]

Two differential equations having the same normal form are called
\textbf{equivalent}.  

\begin{Result}
  The change of variables
  \[
  y(x) = \exp\left( - \frac{1}{2} \int p(x) \,\dd x \right) u(x)
  \]
  transforms the differential equation
  \[
  y'' + p(x) y' + q(x) y = 0
  \]
  into its normal form
  \[
  u'' + I(x) u = 0
  \]
  where the invariant of the equation, $I(x)$, is
  \[
  I(x) =  q - \frac{p^2}{4} - \frac{p'}{2}.
  \]
\end{Result}















%%-----------------------------------------------------------------------------
\subsection{Higher Order Differential Equations}

Consider the third order differential equation
\[
y''' + p(x) y'' + q(x) y' + r(x) y = 0.
\]
We can eliminate the $y''$ term.
Making the change of dependent variable 
\begin{align*}
  y &= u \exp\left( - \frac{1}{3} \int p(x)\,\dd x \right) \\
  y' &= \left[u' -\frac{1}{3} p u \right] 
  \exp\left( - \frac{1}{3} \int p(x)\,\dd x \right) \\
  y'' &= \left[u'' -\frac{2}{3} p u' +\frac{1}{9} (p^2 - 3 p') u \right] 
  \exp\left( - \frac{1}{3} \int p(x)\,\dd x \right) \\
  y'' &= \left[u''' - p u'' +\frac{1}{3} (p^2- 3 p') u'
    +\frac{1}{27} (9p' - 9p'' - p^3 ) u \right] 
  \exp\left( - \frac{1}{3} \int p(x)\,\dd x \right) 
\end{align*}
yields the differential equation
\[
u''' + \frac{1}{3} (3 q - 3 p' - p^2) u' + \frac{1}{27} (27 r - 9 p q
-9 p'' + 2 p^3 ) u = 0.
\]



\begin{Result}
  The change of variables
  \[
  y(x) = \exp\left( - \frac{1}{n} \int p_{n-1}(x) \,\dd x \right) u(x)
  \]
  transforms the differential equation
  \[
  y^{(n)} + p_{n-1}(x) y^{(n-1)} + p_{n-2}(x) y^{(n-2)} + \cdots + p_0(x) y = 0
  \]
  into the form
  \[
  u^{(n)} + a_{n-2}(x) u^{(n-2)} + a_{n-3}(x) u^{(n-3)} + \cdots + a_0(x) u = 0.
  \]
\end{Result}








%%%%%%%%%%%%%%%%%%%%%%%%%%%%%%%%%%%%%%%%%%%%%%%%%%%%%%%%%%%%%%%%%%%%%%%%%%%%%%%%%%%%%%%%%%%%%%%%%%%%%%%%%%%%%%%%%%%%%%%%%%%%%%%%%%%%%%%%%%%%%%%%%%%%%%%%%%%%
%%==============================================================================
%%\section{Scale Transformations}
%%%%%%%%%%%%%%%%%%%%%%%%%%%%%%%%%%%%%%%%%%%%%%%%%%%%%%%%%%%%%%%%%%%%%%%%%%%%%%%%%%%%%%%%%%%%%%%%%%%%%%%%%%%%%%%%%%%%%%%%%%%%%%%%%%%%%%%%%%%%%%%%%%%%%%%%%%%%


%%==============================================================================
\section{Transformations of the Independent Variable}
\index{transformations!of independent variable}

%%-----------------------------------------------------------------------------
\subsection{Transformation to the form u'' + a(x) u = 0}

Consider the second order linear differential equation
\[
y'' + p(x) y' + q(x) y = 0.
\]
We make the change of independent variable
\[
\xi = f(x), \qquad u(\xi) = y(x).
\]
The derivatives in terms of $\xi$ are
\begin{align*}
  \frac{\dd}{\dd x} &= \frac{\dd \xi}{\dd x} \frac{\dd}{\dd \xi} = f' \frac{\dd}{\dd \xi} \\
  \frac{\dd^2}{\dd x^2} &= f' \frac{\dd}{\dd \xi} f' \frac{\dd}{\dd \xi}
  = (f')^2 \frac{\dd^2}{\dd \xi^2} + f'' \frac{\dd}{\dd \xi}
\end{align*}
The differential equation becomes
\[
(f')^2 u'' + f'' u' + p f' u' + q u = 0.
\]
In order to eliminate the $u'$ term, $f$ must satisfy
\[
f'' + p f' = 0
\]
\[
f' = \exp \left( - \int p(x) \,\dd x \right) 
\]
\[
f = \int \exp \left( - \int p(x) \,\dd x \right) \,\dd x.
\]
The differential equation for $u$ is then
\[
u'' + \frac{q}{(f')^2} u = 0
\]
\[
u''(\xi) + q(x) \exp \left(2 \int p(x)\,\dd x \right) u(\xi) = 0.
\]



\begin{Result}
  The change of variables
  \[
  \xi = \int \exp \left( - \int p(x) \,\dd x \right) \,\dd x,
  \qquad u(\xi) = y(x)
  \]
  transforms the differential equation
  \[
  y'' + p(x) y' + q(x) y = 0
  \]
  into
  \[
  u''(\xi)  + q(x) \exp \left(2 \int p(x)\,\dd x \right) u(\xi) = 0.
  \]
\end{Result}














%%-----------------------------------------------------------------------------
\subsection{Transformation to a Constant Coefficient Equation}
\index{transformations!to constant coefficient equation}

Consider the second order linear differential equation
\[
y'' + p(x) y' + q(x) y = 0.
\]
With the change of independent variable
\[
\xi = f(x), \qquad u(\xi)=y(x),
\]
the differential equation becomes
\[
(f')^2 u'' + (f'' + p f') u' + q u = 0.
\]
For this to be a constant coefficient equation we must have
\[
(f')^2 = c_1 q, \qquad \mathrm{and} \qquad f'' + p f' = c_2 q,
\]
for some constants $c_1$ and $c_2$.
Solving the first condition,
\[
f' = c \sqrt{q},
\]
\[
f = c \int \sqrt{q(x)} \,\dd x.
\]
The second constraint becomes
\begin{gather*}
  \frac{f'' + p f'}{q} = \mathrm{const} \\
  \frac{\frac{1}{2} c q^{-1/2} q' + p c q^{1/2}}{q} = \mathrm{const} \\
  \frac{q' + 2 p q}{q^{3/2}} = \mathrm{const}.
\end{gather*}



\begin{Result}
  Consider the differential equation
  \[
  y'' + p(x) y' + q(x) y = 0.
  \]
  If the expression
  \[
  \frac{q' + 2 p q}{q^{3/2}}
  \]
  is a constant then the change of variables
  \[
  \xi = c \int \sqrt{q(x)} \,\dd x, \quad u(\xi) = y(x),
  \]
  will yield a constant coefficient differential equation.  (Here $c$ is
  an arbitrary constant.)
\end{Result}





















%%=============================================================================
\section{Integral Equations}
\index{integral equations}
\index{integral equations!initial value problems}
\index{integral equations!boundary value problems}
\index{Volterra equations}
\index{Fredholm equations}
\index{transformations!to integral equations}


\paragraph{Volterra's Equations.}
Volterra's integral equation of the first kind has the form
\[
\int_a^x N(x,\xi) f(\xi) \,\dd \xi = f(x).
\]
The Volterra equation of the second kind is
\[
y(x) = f(x) + \lambda \int_a^x N(x,\xi) y(\xi) \,\dd \xi.
\]
$N(x,\xi)$ is known as the kernel of the equation.



\paragraph{Fredholm's Equations.}
Fredholm's integral equations of the first and second kinds are
\[
\int_a^b N(x,\xi) f(\xi) \,\dd \xi = f(x),
\]
\[
y(x) = f(x) + \lambda \int_a^b N(x,\xi) y(\xi) \,\dd \xi.
\]




%%-----------------------------------------------------------------------------
\subsection{Initial Value Problems}

Consider the initial value problem
\[
y'' + p(x) y' + q(x) y = f(x), \qquad y(a)=\alpha, \quad y'(a)=\beta.
\]
Integrating this equation twice yields
\[
\int_a^x \int_a^\eta y''(\xi) + p(\xi) y'(\xi) + q(\xi) y(\xi) \,\dd \xi \,\dd \eta
= \int_a^x \int_a^\eta f(\xi) \,\dd \xi \,\dd \eta
\]
\[
\int_a^x (x-\xi)[y''(\xi) + p(\xi) y'(\xi) + q(\xi) y(\xi)] \,\dd \xi 
= \int_a^x (x-\xi) f(\xi) \,\dd \xi .
\]
Now we use integration by parts.
\begin{align*}
  &\big[ (x-\xi) y'(\xi) \big]_a^x - \int_a^x -y'(\xi) \,\dd \xi
  +\big[ (x-\xi) p(\xi) y(\xi) \big]_a^x 
  - \int_a^x [(x-\xi)p'(\xi) - p(\xi)]y(\xi) \,\dd \xi \\
  &\qquad + \int_a^x (x-\xi) q(\xi) y(\xi) \,\dd \xi 
  = \int_a^x (x-\xi) f(\xi) \,\dd \xi .
\end{align*}
\begin{align*}
  &- (x-a) y'(a) + y(x) - y(a) -(x-a) p(a) y(a) 
  - \int_a^x [(x-\xi)p'(\xi) - p(\xi)]y(\xi) \,\dd \xi \\
  & \qquad + \int_a^x (x-\xi) q(\xi) y(\xi) \,\dd \xi 
  = \int_a^x (x-\xi) f(\xi) \,\dd \xi .
\end{align*}
We obtain a Volterra integral equation of the second kind for $y(x)$.
\[
\boxed{
  y(x) = \int_a^x (x-\xi) f(\xi) \,\dd \xi + (x-a) (\alpha p(a) +\beta) + \alpha
  + \int_a^x \big\{ (x-\xi) [p'(\xi)-q(\xi)]-p(\xi) \big\} y(\xi)\,\dd \xi .
  }
\]

Note that the initial conditions for the differential equation are ``built 
into'' the Volterra equation.  Setting $x=a$ in the Volterra equation 
yields $y(a)=\alpha$.  Differentiating the Volterra equation,
\[
y'(x) = \int_a^x f(\xi) \,\dd \xi + (\alpha p(a) + \beta) - p(x) y(x) 
+ \int_a^x [p'(\xi)-q(\xi)]-p(\xi)y(\xi) \,\dd \xi
\]
and setting $x=a$ yields
\[
y'(a) = \alpha p(a) + \beta - p(a) \alpha = \beta.
\]

(Recall from calculus that
\[
\frac{\dd}{\dd x} \int^x g(x,\xi) \,\dd \xi = g(x,x) + \int^x \frac{\partial}{\partial x} [g(x,\xi)]
\,\dd \xi.)
\]






\begin{Result}
  The initial value problem
  \[
  y'' + p(x) y' + q(x) y = f(x), \qquad y(a)=\alpha, \quad y'(a)=\beta.
  \]
  is equivalent to the Volterra equation of the second kind
  \[
  y(x) = F(x) + \int_a^x N(x,\xi) y(\xi) \,\dd \xi
  \]
  where
  \begin{align*}
    F(x) &= \int_a^x (x-\xi) f(\xi) \,\dd \xi + (x-a) (\alpha p(a) +\beta) +\alpha \\
    N(x,\xi) &= (x-\xi) [p'(\xi)-q(\xi)]-p(\xi).
  \end{align*}
\end{Result}




















%%-----------------------------------------------------------------------------
\subsection{Boundary Value Problems}


Consider the boundary value problem
\begin{equation}
  \label{can_ydpefx}
  y'' = f(x), \qquad y(a) = \alpha, \quad y(b) = \beta.
\end{equation}
To obtain a problem with homogeneous boundary conditions, we make the 
change of variable
\[
y(x) = u(x) + \alpha + \frac{\beta-\alpha}{b-a} (x-a)
\]
to obtain the problem
\[
u'' = f(x), \qquad u(a) = u(b) = 0.
\]
Now we will use Green's functions to write the solution as an integral.  
First we solve the problem
\[
G'' = \delta(x-\xi), \qquad G(a|\xi) = G(b|\xi) = 0.
\]
The homogeneous solutions of the differential equation that satisfy the 
left and right boundary conditions are
\[
c_1 (x-a) \quad \mathrm{and} \quad c_2 (x-b).
\]
Thus the Green's function has the form
\[
G(x|\xi) = 
\begin{cases}
  c_1 (x-a), \quad &\mathrm{for}\ x \leq \xi \\
  c_2 (x-b), \quad &\mathrm{for}\ x \geq \xi
\end{cases}
\]
Imposing continuity of $G(x|\xi)$ at $x=\xi$ and a unit jump of $G(x|\xi)$ 
at $x=\xi$, we obtain
\[
G(x|\xi) = 
\begin{cases}
  \frac{(x-a)(\xi-b)}{b-a}, \quad &\mathrm{for}\ x \leq \xi \\
  \frac{(x-b)(\xi-a)}{b-a}, \quad &\mathrm{for}\ x \geq \xi
\end{cases}
\]
Thus the solution of the (\ref{can_ydpefx}) is
\[
y(x) = \alpha + \frac{\beta-\alpha}{b-a} (x-a)
+ \int_a^b G(x|\xi) f(\xi) \,\dd \xi.
\]




Now consider the boundary value problem
\[
y'' + p(x) y' + q(x) y = 0, \qquad y(a) = \alpha, \quad y(b) = \beta.
\]
From the above result we can see that the solution satisfies
\[
y(x) = \alpha + \frac{\beta-\alpha}{b-a} (x-a)
+ \int_a^b G(x|\xi) [f(\xi)-p(\xi)y'(\xi)-q(\xi) y(\xi)] \,\dd \xi.
\]
Using integration by parts, we can write
\begin{align*}
  - \int_a^b G(x|\xi) p(\xi)y'(\xi)\,\dd \xi
  &= -\big[ G(x|\xi) p(\xi) y(\xi) \big]_a^b 
  + \int_a^b \left[\frac{\partial G(x|\xi)}{\partial \xi} p(\xi) 
    + G(x|\xi) p'(\xi) \right] y(\xi) \,\dd \xi \\
  &= \int_a^b \left[\frac{\partial G(x|\xi)}{\partial \xi} p(\xi) 
    + G(x|\xi) p'(\xi) \right] y(\xi) \,\dd \xi .
\end{align*}
Substituting this into our expression for $y(x)$, 
\[
y(x) = \alpha + \frac{\beta-\alpha}{b-a} (x-a)
+ \int_a^b G(x|\xi) f(\xi) \,\dd \xi
+ \int_a^b \left[\frac{\partial G(x|\xi)}{\partial \xi} p(\xi)+ G(x|\xi) [p'(\xi) 
  -q(\xi)] \right] y(\xi) \,\dd \xi,
\]
we obtain a Fredholm integral equation of the second kind.



\begin{Result}
  The boundary value problem
  \[
  y'' + p(x) y' + q(x) y = f(x), \qquad y(a)=\alpha, \quad y(b)=\beta.
  \]
  is equivalent to the Fredholm equation of the second kind
  \[
  y(x) = F(x) + \int_a^b N(x,\xi) y(\xi) \,\dd \xi
  \]
  where
  \begin{align*}
    F(x) &= \alpha + \frac{\beta-\alpha}{b-a} (x-a)
    + \int_a^b G(x|\xi) f(\xi) \,\dd \xi, \\
    N(x,\xi) &= \int_a^b H(x|\xi) y(\xi) \,\dd \xi, \\
    G(x|\xi) &= 
    \begin{cases}
      \frac{(x-a)(\xi-b)}{b-a}, \quad &\mathrm{for}\ x \leq \xi \\
      \frac{(x-b)(\xi-a)}{b-a}, \quad &\mathrm{for}\ x \geq \xi,
    \end{cases} \\
    H(x|\xi) &= 
    \begin{cases}
      \frac{(x-a)}{b-a} p(\xi) + \frac{(x-a)(\xi-b)}{b-a}[p'(\xi)-q(\xi)] 
      \quad &\mathrm{for}\ x \leq \xi \\
      \frac{(x-b)}{b-a} p(\xi) + \frac{(x-b)(\xi-a)}{b-a}[p'(\xi)-q(\xi)]
      \quad &\mathrm{for}\ x \geq \xi.
    \end{cases}
  \end{align*}
\end{Result}

























\raggedbottom
%%===========================================================================
\exercises{
\pagebreak
\flushbottom
\section{Exercises}




%%-----------------------------------------------------------------------------
\begin{large}
  \noindent
  \textbf{The Constant Coefficient Equation}
\end{large}



%%-----------------------------------------------------------------------------
\begin{large}
  \noindent
  \textbf{Normal Form}
\end{large}



%% Solve the differential equation... by transforming to normal form.
\begin{Exercise}
  \label{exercise y243xy192412x4x2y=0}
  Solve the differential equation
  \[
  y'' + \left( 2 + \frac{4}{3} x \right) y' + \frac{1}{9}
  \left( 24 + 12 x + 4 x^2 \right) y = 0.
  \]

  \hintsolution{y243xy192412x4x2y=0}
\end{Exercise}



%%-----------------------------------------------------------------------------
\begin{large}
  \noindent
  \textbf{Transformations of the Independent Variable}
\end{large}



%%-----------------------------------------------------------------------------
\begin{large}
  \noindent
  \textbf{Integral Equations}
\end{large}








%% CONTINUE HERE

%%222222222222222222222222222222222222222222222222222222222222222222222222222222
\begin{Exercise}
  \label{exercise y2abxycdxex2y=0}
  Show that the solution of the differential equation
  \[
  y'' + 2 (a + b x) y' + (c + d x + e x^2) y = 0
  \]
  can be written in terms of one of the following canonical forms:
  \begin{align*}
    &v'' + (\xi^2 + A) v = 0 \\
    &v'' = \xi v \\
    &v'' + v = 0 \\
    &v'' = 0.
  \end{align*}

  \hintsolution{y2abxycdxex2y=0}
\end{Exercise}





%%333333333333333333333333333333333333333333333333333333333333333333333333333333
\begin{Exercise}
  \label{exercise canonical y2abxycdxex2y=0}
  Show that the solution of the differential equation
  \[
  y'' + 2\left( a + \frac{b}{x} \right) y' + \left( c + \frac{d}{x}
    + \frac{e}{x^2} \right) y = 0
  \]
  can be written in terms of one of the following canonical forms:
  \begin{align*}
    &v'' + \left( 1 + \frac{A}{\xi} + \frac{B}{\xi^2} \right) v = 0 \\
    &v'' + \left( \frac{1}{\xi} + \frac{A}{\xi^2} \right) v = 0 \\
    &v'' + \frac{A}{\xi^2} v = 0 
  \end{align*}

  \hintsolution{canonical y2abxycdxex2y=0}
\end{Exercise}






%%44444444444444444444444444444444444444444444444444444444444444444444444444444
\begin{Exercise} 
  \label{exercise x2yaxyay=0}
  Show that the second order Euler equation
  \[ 
  x^2 \frac{\dd^2 y}{\dd^2 x} + a_1 x \frac{\dd y}{\dd x} + a_0 y = 0 
  \]
  can be transformed to a constant coefficient equation.

  \hintsolution{x2yaxyay=0}
\end{Exercise}






%%55555555555555555555555555555555555555555555555555555555555555555555555555555
\begin{Exercise}
  \label{exercise y1xy114x2y=0}
  Solve Bessel's equation of order $1/2$,
  \[
  y'' + \frac{1}{x} y' + \left( 1 - \frac{1}{4x^2} \right) y = 0.
  \]

  \hintsolution{y1xy114x2y=0}
\end{Exercise}






\raggedbottom
}
%%============================================================================
\hints{
\pagebreak
\flushbottom
\section{Hints}




%%-----------------------------------------------------------------------------
\begin{large}
  \noindent
  \textbf{The Constant Coefficient Equation}
\end{large}



%%-----------------------------------------------------------------------------
\begin{large}
  \noindent
  \textbf{Normal Form}
\end{large}



%% Solve the differential equation... by transforming to normal form.
\begin{Hint}
  \label{hint y243xy192412x4x2y=0}
  Transform the equation to normal form.
\end{Hint}



%%-----------------------------------------------------------------------------
\begin{large}
  \noindent
  \textbf{Transformations of the Independent Variable}
\end{large}



%%-----------------------------------------------------------------------------
\begin{large}
  \noindent
  \textbf{Integral Equations}
\end{large}





%%222222222222222222222222222222222222222222222222222222222222222222222222222222
\begin{Hint}
  \label{hint y2abxycdxex2y=0}
  Transform the equation to normal form and then apply the scale transformation
  $x = \lambda \xi + \mu$.
\end{Hint}



%%333333333333333333333333333333333333333333333333333333333333333333333333333333
\begin{Hint}
  \label{hint canonical y2abxycdxex2y=0}
  Transform the equation to normal form and then apply the scale transformation
  $x = \lambda \xi$.
\end{Hint}





%%44444444444444444444444444444444444444444444444444444444444444444444444444444
\begin{Hint}
  \label{hint x2yaxyay=0}
  Make the change of variables $x = \e^t$, $y(x) = u(t)$.
  Write the derivatives with respect to $x$ in terms of $t$.
  \begin{gather*}
    x = \e^t \\
    d x = \e^t d t \\
    \frac{\dd}{\dd x} = \e^{-t} \frac{\dd}{\dd t} \\
    x \frac{\dd}{\dd x} = \frac{\dd}{\dd t}
  \end{gather*}
\end{Hint}




%%55555555555555555555555555555555555555555555555555555555555555555555555555555
\begin{Hint}
  \label{hint y1xy114x2y=0}
  Transform the equation to normal form.
\end{Hint}






\raggedbottom
}
%%============================================================================
\solutions{
\pagebreak
\flushbottom
\section{Solutions}




%%-----------------------------------------------------------------------------
\begin{large}
  \noindent
  \textbf{The Constant Coefficient Equation}
\end{large}



%%-----------------------------------------------------------------------------
\begin{large}
  \noindent
  \textbf{Normal Form}
\end{large}



%% Solve the differential equation... by transforming to normal form.
\begin{Solution}
  \label{solution y243xy192412x4x2y=0}
  \[
  y'' + \left( 2 + \frac{4}{3} x \right) y' + \frac{1}{9}
  \left( 24 + 12 x + 4 x^2 \right) y = 0
  \]
  To transform the equation to normal form we make the substitution
  \begin{align*}
    y &= \exp\left( - \frac{1}{2} \int \left( 2 + \frac{4}{3} x \right)\,\dd x 
    \right) u \\
    &= \e^{-x - x^2 / 3} u
  \end{align*}
  The invariant of the equation is
  \begin{align*}
    I(x) &= \frac{1}{9} \left( 24 + 12 x + 4 x^2 \right) 
    - \frac{1}{4} \left( 2 + \frac{4}{3} x \right)^2
    - \frac{1}{2} \frac{\dd}{\dd x} \left(2 + \frac{4}{3} x \right) \\
    &= 1.
  \end{align*}
  The normal form of the differential equation is then
  \[
  u'' + u = 0
  \]
  which has the general solution
  \[
  u = c_1 \cos x + c_2 \sin x
  \]
  Thus the equation for $y$ has the general solution
  \[
  \boxed{
    y = c_1 \e^{-x-x^2/3} \cos x + c_2 \e^{-x-x^2/3} \sin x.
    }
  \]
\end{Solution}



%%-----------------------------------------------------------------------------
\begin{large}
  \noindent
  \textbf{Transformations of the Independent Variable}
\end{large}



%%-----------------------------------------------------------------------------
\begin{large}
  \noindent
  \textbf{Integral Equations}
\end{large}
















%%222222222222222222222222222222222222222222222222222222222222222222222222222222
\begin{Solution}
  \label{solution y2abxycdxex2y=0}
  The substitution that will transform the equation to normal form is
  \begin{align*}
    y       &= \exp\left( -\frac{1}{2} \int 2(a + b x) \,\dd x \right) u \\
    &= \e^{-a x - b x^2 / 2} u.
  \end{align*}
  The invariant of the equation is
  \begin{align*}
    I(x)    &= c + d x + e x^2 - \frac{1}{4} (2(a+b x))^2 - \frac{1}{2}
    \frac{\dd}{\dd x} (2(a+b x)) \\
    &= c - b - a^2 + (d - 2 a b) x + (e - b^2) x^2 \\
    &\equiv \alpha + \beta x + \gamma x^2
  \end{align*}
  The normal form of the differential equation is
  \[
  u'' + (\alpha + \beta x + \gamma x^2) u = 0
  \]
  We consider the following cases:
  \begin{description}
  \item{$\gamma = 0$.}
    \begin{description}
    \item{$\beta = 0$.}
      \begin{description}
      \item{$\alpha = 0$.}
        We immediately have the equation
        \[
        u'' = 0.
        \]
      \item{$\alpha \neq 0$.}
        With the change of variables
        \[
        v(\xi) = u(x), \qquad x = \alpha^{-1/2} \xi,
        \]
        we obtain
        \[
        v'' + v = 0.
        \]
      \end{description}
    \item{$\beta \neq 0$.}
      We have the equation
      \[
      y'' + (\alpha + \beta x) y = 0.
      \]
      The scale transformation $x = \lambda  \xi + \mu$ yields
      \[
      v'' + \lambda^2 (\alpha + \beta (\lambda \xi + \mu)) y = 0
      \]
      \[
      v'' = [  \beta \lambda^3 \xi + \lambda^2 (\beta \mu + \alpha)]v.
      \]
      Choosing
      \[
      \lambda = (-\beta)^{-1/3}, \qquad \mu = - \frac{\alpha}{\beta}
      \]
      yields the differential equation
      \[
      v'' = \xi v.
      \]
    \end{description}
  \item{$\gamma \neq 0$.}
    The scale transformation $x = \lambda \xi + \mu$ yields
    \[
    v'' + \lambda^2 [\alpha + \beta(\lambda \xi + \mu) 
    + \gamma(\lambda \xi + \mu)^2] v = 0
    \]
    \[
    v'' + \lambda^2 [ \alpha + \beta \mu + \gamma \mu^2
    + \lambda (\beta + 2 \gamma \mu) \xi + \lambda^2 \gamma \xi^2]v=0.
    \]
    Choosing
    \[
    \lambda = \gamma^{-1/4}, \qquad \mu = -\frac{\beta}{2 \gamma}
    \]
    yields the differential equation
    \[
    v'' + (\xi^2 + A) v = 0
    \]
    where
    \[
    A = \gamma^{-1/2} - \frac{1}{4} \beta \gamma^{-3/2}.
    \]
  \end{description}
\end{Solution}








%%333333333333333333333333333333333333333333333333333333333333333333333333333333
\begin{Solution}
  \label{solution canonical y2abxycdxex2y=0}
  The substitution that will transform the equation to normal form is
  \begin{align*}
    y &= \exp \left( -\frac{1}{2} \int 2 \left( a + \frac{b}{x} \right) 
      \,\dd x \right) u \\
    &= x^{-b} \e^{-ax} u.
  \end{align*}
  The invariant of the equation is
  \begin{align*}
    I(x) &= c + \frac{d}{x} + \frac{e}{x^2} - \frac{1}{4} \left( 
      2 \left( a + \frac{b}{x} \right) \right)^2 - \frac{1}{2} 
    \frac{\dd}{\dd x} \left( 2 \left(a + \frac{b}{x} \right) \right) \\
    &= c - a^x + \frac{d-2a b}{x} + \frac{e+b-b^2}{x^2} \\
    &\equiv \alpha + \frac{\beta}{x} + \frac{\gamma}{x^2}.
  \end{align*}
  The invariant form of the differential equation is
  \[
  u'' + \left( \alpha + \frac{\beta}{x} + \frac{\gamma}{x^2} \right)u = 0.
  \]
  We consider the following cases:
  \begin{description}
  \item{$\alpha = 0$.}
    \begin{description}
    \item{$\beta = 0$.}
      We immediately have the equation 
      \[
      u'' + \frac{\gamma}{x^2} u = 0.
      \]
    \item{$\beta \neq 0$.}
      We have the equation
      \[
      u'' + \left( \frac{\beta}{x} + \frac{\gamma}{x^2} \right) u = 0.
      \]
      The scale transformation $u(x) = v(\xi)$, $x = \lambda \xi$ yields
      \[
      v'' + \left( \frac{\beta \lambda}{\xi} + \frac{\gamma}{\xi^2} \right)u=0.
      \]
      Choosing $\lambda = \beta^{-1}$, we obtain
      \[
      v'' + \left( \frac{1}{\xi} + \frac{\gamma}{\xi^2} \right)u=0.
      \]
    \end{description}
  \item{$\alpha \neq 0$.}
    The scale transformation $x = \lambda \xi$ yields
    \[
    v'' + \left( \alpha \lambda^2 + \frac{\beta \lambda}{\xi} 
      + \frac{\gamma}{\xi^2} \right) v = 0.
    \]
    Choosing $\lambda = \alpha^{-1/2}$, we obtain
    \[
    v'' + \left( 1 + \frac{\alpha^{-1/2} \beta}{\xi} + \frac{\gamma}{\xi^2}
    \right) v = 0.
    \]
  \end{description}
\end{Solution}









%%44444444444444444444444444444444444444444444444444444444444444444444444444444
%% CONTINUE - Derive the change of variable.
\begin{Solution}
  \label{solution x2yaxyay=0}
  We write the derivatives with respect to $x$ in terms of $t$.
  \begin{gather*}
    x = \e^t \\
    d x = \e^t d t \\
    \frac{\dd}{\dd x} = \e^{-t} \frac{\dd}{\dd t} \\
    x \frac{\dd}{\dd x} = \frac{\dd}{\dd t}
  \end{gather*}
  Now we express $x^2 \frac{\dd^2}{\dd x^2}$ in terms of $t$.
  \[
  x^2 \frac{\dd^2}{\dd x^2} 
  = x \frac{\dd}{\dd x}\left(x \frac{\dd}{\dd x}\right) - x \frac{\dd}{\dd x} 
  = \frac{\dd^2}{\dd t^2} - \frac{\dd}{\dd t}
  \]
  Thus under the change of variables, $x = \e^t$, $y(x) = u(t)$,
  the Euler equation becomes
  \begin{gather*}
    u'' - u' + a_1 u' + a_0 u = 0 \\
    \boxed{u'' + (a_1-1)u' + a_0 u = 0.}
  \end{gather*}
\end{Solution}








%%55555555555555555555555555555555555555555555555555555555555555555555555555555
\begin{Solution}
  \label{solution y1xy114x2y=0}
  The transformation
  \[
  y = \exp\left(-\frac{1}{2} \int \frac{1}{x} \,\dd x \right)
  = x^{-1/2} u
  \]
  will put the equation in normal form.  The invariant is
  \[
  I(x) = \left( 1 - \frac{1}{4x^2} \right) - \frac{1}{4} \left( \frac{1}{x^2}
  \right) - \frac{1}{2} \frac{-1}{x^2} = 1.
  \]
  Thus we have the differential equation
  \[
  u'' + u = 0,
  \]
  with the solution
  \[
  u = c_1 \cos x + c_2 \sin x.
  \]
  The solution of Bessel's equation of order $1/2$ is
  \[
  \boxed{
    y = c_1 x^{-1/2} \cos x + c_2 x^{-1/2} \sin x.
    }
  \]
\end{Solution}



\raggedbottom
}
