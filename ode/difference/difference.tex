\flushbottom

%%%%
%% Improve the exposition of the section on Euler equations and add back in.
%%%%
%% Include all the material in chap 5 of Spiegel.
%%%%



%%===========================================================================
%%===========================================================================
\chapter{Difference Equations}

Televisions should have a dial to turn up the intelligence.  There is a 
brightness knob, but it doesn't work.

\begin{flushright}
  -?
\end{flushright}



%%============================================================================
\section{Introduction}
\begin{Example}
  \label{gambler's-ruin-intro}
  \index{gambler's ruin problem}
  \textbf{Gambler's ruin problem.}
  Consider a gambler that initially has $n$ dollars.  He plays a game in which
  he has a probability $p$ of winning a dollar and $q$ of losing a dollar.
  (Note that $p + q = 1$.)
  The gambler has decided that if he attains $N$ dollars he will stop playing
  the game.  In this case we will say that he has succeeded.  Of course if he 
  runs out of money before that happens, we will say that he is ruined.  What is
  the probability of the gambler's ruin?  Let us denote this probability by
  $a_n$.  We know that if he has no money left, then his ruin is certain, so 
  $a_0$ = 1.  If he reaches $N$ dollars he will quit the game, so that
  $a_N = 0$.  If he is somewhere in between ruin and success then the probability
  of his ruin is equal to $p$ times the probability of his ruin if he had
  $n+1$ dollars plus $q$ times the probability of his ruin if he had
  $n-1$ dollars.  Writing this in an equation,
  \[ a_n = pa_{n+1} + q a_{n-1} \quad \mathrm{subject to}\quad a_0=1, \quad
  a_N = 0.\]
  This is an example of a difference equation.  You will learn how to solve this
  particular problem in the section on constant coefficient equations.
\end{Example}




\index{discrete derivative}
Consider the sequence $a_1, a_2, a_3, \ldots$ Analogous to a derivative of
a continuous function, we can define a discrete derivative on the sequence
\[      D a_n = a_{n+1} - a_n. \]
The second discrete derivative is then defined as
\[      D^2a_n = D[a_{n+1} - a_n] = a_{n+2} - 2 a_{n+1} + a_n. \]
\index{discrete integral}
The discrete integral of $a_n$ is
\[      \sum_{i=n_0}^n a_i. \]
Corresponding to
\[ \int_\alpha^\beta \frac{\dd f}{\dd x}\,dx = f(\beta) - f(\alpha),\]
in the discrete realm we have
\[ \sum_{n=\alpha}^{\beta-1} D[a_n] = \sum_{n=\alpha}^{\beta-1} (a_{n+1}-a_n)
= a_\beta - a_\alpha. \]

Linear difference equations have the form 
\[ D^r a_n + p_{r-1}(n) D^{r-1}a_n + \cdots + p_1(n)D a_n + p_0(n)a_n = f(n).\]
From the definition of the discrete derivative an equivalent form is
\[ a_{n+r} + q_{r-1}(n) a_{n_r-1} + \cdots + q_1(n)a_{n+1} + q_0(n)a_n=f(n).\]

Besides being important in their own right, we will need to solve
difference equations in order to develop series solutions of 
differential equations.  Also, some methods of solving differential equations
numerically are based on approximating them with difference equations.

There are many similarities between differential and difference equations.
Like differential equations, an $r^{t h}$ order homogeneous difference 
equation has $r$ linearly independent solutions.  The general solution
to the $r^{t h}$ order inhomogeneous equation is the sum of the 
particular solution and an arbitrary linear combination of the homogeneous
solutions.  

For an $r^{t h}$ order difference equation, the initial condition is given
by specifying the values of the first $r$ $a_n$'s.



\begin{Example}
  Consider the difference equation $a_{n-2} - a_{n-1} - a_n = 0$ subject to
  the initial condition $a_1 = a_2 = 1$.  Note that although we may not 
  know a closed-form formula for the $a_n$ we can calculate the $a_n$ in 
  order by substituting into the difference equation.  The first few $a_n$ are
  $1, 1, 2, 3, 5, 8, 13, 21, \ldots$  We recognize this as the 
  Fibonacci sequence.  
\end{Example}





%%===========================================================================
\section{Exact Equations}
\index{difference equations!exact equations}
Consider the sequence $a_1, a_2, \ldots$.
Exact difference equations on this sequence have the form
\[ D[F(a_n, a_{n+1}, \ldots, n)] = g(n).\]
We can reduce the order of, (or solve for first order), 
this equation by summing from $1$ to $n-1$.
\begin{gather*}
  \sum_{j=1}^{n-1} D[F(a_j, a_{j+1}, \ldots, j)] = \sum_{j=1}^{n-1} g(j) \\
  F(a_n, a_{n+1}, \ldots, n) - F(a_1, a_2, \ldots, 1) = \sum_{j=1}^{n-1} g(j) \\
  F(a_n, a_{n+1}, \ldots, n) = \sum_{j=1}^{n-1} g(j) + F(a_1, a_2, \ldots, 1) 
\end{gather*}


\begin{Result}
  We can reduce the order of the exact difference equation
  \[ D[F(a_n, a_{n+1}, \ldots, n)] = g(n), \qquad \mathrm{for}\ n \geq 1\]
  by summing both sides of the equation to obtain
  \[F(a_n, a_{n+1}, \ldots, n) = \sum_{j=1}^{n-1} g(j) +F(a_1, a_2, \ldots, 1).\]
\end{Result}



\begin{Example} 
  Consider the difference equation, $D[n a_n] = 1$.   
  Summing both sides of this equation
  \begin{gather*} 
    \sum_{j=1}^{n-1} D[j a_j] = \sum_{j=1}^{n-1} 1 \\ 
    n a_n - a_1 = n-1 \\ 
    \boxed{ a_n = \frac{n + a_1 - 1}{n}.} 
  \end{gather*} 
\end{Example} 





%%============================================================================ 
\section{Homogeneous First Order} 
\index{difference equations!first order homogeneous} 
Consider the homogeneous first order difference equation
\[      a_{n+1} = p(n)a_n, \qquad \mathrm{for}\ n \geq 1. \] 
We can directly solve for $a_n$. 
\begin{align*} 
  a_n     &=     a_n \frac{a_{n-1}}{a_{n-1}} \frac{a_{n-2}}{a_{n-2}} \cdots  
  \frac{a_1}{a_1} \\ 
  &=     a_1 \frac{a_n}{a_{n-1}} \frac{a_{n-1}}{a_{n-2}} \cdots 
  \frac{a_2}{a_1}  \\ 
  &=     a_1 p(n-1)p(n-2) \cdots p(1) \\ 
  &=     a_1 \prod_{j=1}^{n-1} p(j) 
\end{align*} 


Alternatively, we could solve this equation by making it exact. 
Analogous to an integrating factor for differential equations,  
we multiply the equation by the summing factor
\[S(n) = \left[\prod_{j=1}^{n} p(j) \right]^{-1}. \] 
\begin{gather*} 
  a_{n+1} - p(n) a_n = 0 \\ 
  \frac{a_{n+1}}{\prod_{j=1}^{n} p(j)} - \frac{a_n}{\prod_{j=1}^{n-1} p(j)}=0\\ 
  D \left[\frac{a_n}{\prod_{j=1}^{n-1} p(j)} \right] = 0 \\ 
  \intertext{Now we sum from $1$ to $n-1$.} 
  \frac{a_n}{\prod_{j=1}^{n-1} p(j)} - a_1 = 0 \\ 
  a_n = a_1 \prod_{j=1}^{n-1} p(j) 
\end{gather*} 




\begin{Result} 
  The solution of the homogeneous first order difference equation
  \[ a_{n+1} = p(n)a_n, \qquad \mathrm{for}\ n \geq 1, \] 
  is  
  \[a_n = a_1 \prod_{j=1}^{n-1} p(j). \] 
\end{Result} 



\begin{Example} 
  Consider the equation $a_{n+1} = n a_n$ with the initial condition
  $a_1 = 1$. 
  \[      a_n = a_1 \prod_{j=1}^{n-1} j = (1) (n-1)!  
  = \Gamma(n) \] 
  Recall that $\Gamma(z)$ is the generalization of the factorial function.  For 
  positive integral values of the argument, $\Gamma(n) = (n-1)!$. 
\end{Example} 


%%============================================================================ 
\section{Inhomogeneous First Order}
\index{difference equations!first order inhomogeneous}
Consider the equation
\[a_{n+1} = p(n) a_n + q(n) \quad \mathrm{for} \quad n \geq 1. \]
Multiplying by $S(n)= \left[ \prod_{j=1}^{n} p(j) \right]^{-1}$ yields
\[ \frac{a_{n+1}}{\prod_{j=1}^{n} p(j)} - \frac{a_n}{\prod_{j=1}^{n-1} p(j)}
= \frac{q(n)}{\prod_{j=1}^{n} p(j)}.  \]
The left hand side is a discrete derivative.
\[ D \left[ \frac{a_n}{\prod_{j=1}^{n-1} p(j)} \right] = 
\frac{q(n)}{\prod_{j=1}^{n} p(j)} \]
Summing both sides from $1$ to $n-1$, 
\[ \frac{a_n}{\prod_{j=1}^{n-1} p(j)} - a_1 = 
\sum_{k=1}^{n-1} \left[ \frac{q(k)}{\prod_{j=1}^{k} p(j)} \right] \]
\[ a_n = \left[ \prod_{m=1}^{n-1} p(m) \right]
\left[ \sum_{k=1}^{n-1} \left[ \frac{q(k)}{\prod_{j=1}^{k} p(j)} 
  \right] + a_1  \right].   \]


\begin{Result}
  The solution of the inhomogeneous first order difference equation
  \[a_{n+1} = p(n) a_n + q(n) \quad \mathrm{for} \quad n \geq 1 \]
  is 
  \[ a_n = \left[ \prod_{m=1}^{n-1} p(m) \right]
  \left[ \sum_{k=1}^{n-1} \left[ \frac{q(k)}{\prod_{j=1}^{k} p(j)} 
    \right] + a_1  \right].   \]
\end{Result}








\begin{Example} 
  Consider the equation $a_{n+1} = n a_n + 1$ for $n \geq 1$.
  The summing factor is
  \begin{gather*}
    S(n) = \left[ \prod_{j=1}^n j \right]^{-1} = \frac{1}{n!}. \\
    \intertext{Multiplying the difference equation by the summing factor,}
    \frac{a_{n+1}}{n!} - \frac{a_n}{(n-1)!} = \frac{1}{n!} \\
    D\left[ \frac{a_n}{(n-1)!} \right] = \frac{1}{n!} \\
    \frac{a_n}{(n-1)!} - a_1 = \sum_{k=1}^{n-1} \frac{1}{k!} \\
    \boxed{ a_n = (n-1)! \left[ \sum_{k=1}^{n-1} \frac{1}{k!} 
        + a_1 \right].} 
  \end{gather*}
\end{Example}





\begin{Example}
  Consider the equation
  \[
  a_{n+1} = \lambda a_n + \mu, \quad \mathrm{for}\ n \geq 0.
  \]
  From the above result, (with the products and sums starting at zero instead
  of one), the solution is
  \begin{align*}
    a_0     &= \left[ \prod_{m=0}^{n-1} \lambda \right] 
    \left[ \sum_{k=0}^{n-1} \left[ \frac{\mu}
        {\prod_{j=0}^k \lambda} \right] + a_0 \right] \\
    &= \lambda^n \left[ \sum_{k=0}^{n-1} \left[ \frac{\mu}
        {\lambda^{k+1}} \right] + a_0 \right] \\
    &= \lambda^n \left[ \mu \frac{\lambda^{-n-1}-\lambda^{-1}}
      {\lambda^{-1} - 1} + a_0 \right] \\
    &= \lambda^n \left[ \mu \frac{\lambda^{-n}-1}
      {1 - \lambda} + a_0 \right] \\
    &= \mu \frac{1 - \lambda^n} {1 - \lambda} + a_0 \lambda^n. 
  \end{align*}
\end{Example}







%%============================================================================
\section{Homogeneous Constant Coefficient Equations}
\index{difference equations!constant coefficient equations}
Homogeneous constant coefficient equations have the form
\[ a_{n+N} + p_{N-1} a_{n+N-1} + \cdots + p_1 a_{n+1} + p_0 a_n = 0. \]
The substitution $a_n = r^n$ yields
\begin{align*}
  r^N + p_{N-1}r^{N-1} + \cdots + p_1 r + p_0 &= 0 \\
  (r-r_1)^{m_1}\cdots (r-r_k)^{m_k} &= 0.
\end{align*}

If $r_1$ is a distinct root then the associated linearly independent solution
is $r_1^n$.  If $r_1$ is a root of multiplicity $m > 1$ then the associated
solutions are $r_1^n, n r_1^n, n^2r_1^n, \ldots, n^{m-1} r_1^n$.



\begin{Result}
  Consider the homogeneous constant coefficient difference equation
  \[ a_{n+N} + p_{N-1} a_{n+N-1} + \cdots + p_1 a_{n+1} + p_0 a_n = 0. \]
  The substitution $a_n = r^n$ yields the equation
  \[(r-r_1)^{m_1}\cdots (r-r_k)^{m_k} = 0. \]
  A set of linearly independent solutions is
  \[ \{r_1^n, n r_1^n, \ldots, n^{m_1-1}r_1^n, \ldots, r_k^n, n r_k^n, \ldots,
  n^{m_k-1}r_k^n\}.\]
\end{Result}



\begin{Example}
  Consider the equation $a_{n+2} - 3a_{n+1} + 2a_n = 0$ with the initial 
  conditions $a_1 = 1$ and $a_2 = 3$.  The substitution $a_n = r^n$ yields
  \[      r^2 - 3r + 2 = (r-1)(r-2) = 0. \]
  Thus the general solution is
  \[      a_n = c_1 1^n + c_2 2^n. \]
  The initial conditions give the two equations,
  \begin{align*}
    a_1 = 1 &= c_1 + 2c_2  \\
    a_2 = 3 &= c_1 + 4c_2
  \end{align*}
  Since $c_1 = -1$ and $c_2 = 1$, the solution to the difference equation
  subject to the initial conditions is
  \[ a_n = 2^n - 1.\]
\end{Example}



\begin{Example}
  \index{gambler's ruin problem}
  Consider the gambler's ruin problem that was introduced in 
  Example~\ref{gambler's-ruin-intro}.
  The equation for the probability of the gambler's ruin at $n$ dollars is
  \[ a_n = pa_{n+1} + q a_{n-1} \quad \mathrm{subject to}\quad a_0=1, \quad
  a_N = 0.\]
  We assume that $0<p<1$.  
  With the substitution $a_n = r^n$ we obtain
  \[ r = p r^2 + q. \]
  The roots of this equation are
  \begin{align*}
    r       &= \frac{1 \pm \sqrt{1 - 4p q}}{2p} \\
    &= \frac{1 \pm \sqrt{1 - 4p(1-p)}}{2p} \\
    &= \frac{1 \pm \sqrt{(1 - 2p)^2}}{2p} \\
    &= \frac{1 \pm |1-2p|}{2p}.
  \end{align*}
  We will consider the two cases $p \neq 1/2$ and $p = 1/2$.
  \begin{description}
  \item{$\mathbf{p \boldsymbol{\neq} 1 \boldsymbol{/} 2}$\textbf{.}} 
    If $p < 1/2$, the roots are 
    \begin{gather*}
      r = \frac{1 \pm (1-2p)}{2p} \\
      r_1 = \frac{1-p}{p} = \frac{q}{p}, \qquad r_2 = 1.
    \end{gather*}
    If $p > 1/2$ the roots are
    \begin{gather*}
      r = \frac{1 \pm (2p-1)}{2p} \\
      r_1 = 1, \qquad r_2 = \frac{-p + 1}{p} = \frac{q}{p}.
    \end{gather*}
    Thus the general solution for $p \neq 1/2$ is
    \[a_n = c_1  + c_2\left(\frac{q}{p}\right)^n.\]
    The boundary condition $a_0=1$ requires that $c_1 + c_2 = 1$.  From the 
    boundary condition $a_N=0$ we have
    \begin{gather*}
      (1-c_2) + c_2 \left(\frac{q}{p}\right)^N = 0 \\
      c_2 = \frac{-1}{-1 + (q/p)^N} \\
      c_2 = \frac{p^N}{p^N-q^N}.
    \end{gather*}
    Solving for $c_1$,
    \begin{gather*}
      c_1 = 1 - \frac{p^N}{p^N-q^N} \\
      c_1 = \frac{-q^N}{p^N - q^N}.
    \end{gather*}
    Thus we have 
    \[\boxed{a_n =  \frac{-q^N}{p^N-q^N} 
      + \frac{p^N}{p^N-q^N}\left(\frac{q}{p}\right)^n.} \]

  \item{$\mathbf{p \boldsymbol{=} 1 \boldsymbol{/} 2}$\textbf{.}}
    In this case, the two roots of the polynomial are both $1$.  The 
    general solution is
    \[ a_n = c_1 + c_2 n.\]
    The left boundary condition demands that $c_1 = 1$.  From the right 
    boundary condition we obtain
    \begin{gather*}
      1 + c_2 N = 0 \\
      c_2 = -\frac{1}{N}.
    \end{gather*}
    Thus the solution for this case is
    \[ \boxed{a_n = 1 - \frac{n}{N}.} \]
    As a check that this formula makes sense, 
    we see that for $n = N/2$ the probability of ruin is 
    $1 - \frac{N/2}{N} = \frac{1}{2}$.
  \end{description}
\end{Example}





%%============================================================================
\section{Reduction of Order}
\index{reduction of order!difference equations}
Consider the difference equation
\begin{equation} \label{difference_a_n}
  (n+1)(n+2)a_{n+2}-3(n+1)a_{n+1}+2a_n = 0\quad \mathrm{for}\quad n \geq 0 
\end{equation}
We see that one solution to this equation is $a_n = 1/n!$.
Analogous to the reduction of order for differential equations, the substitution
$a_n = b_n / n!$ will reduce the order of the difference equation.
\begin{gather}
  \frac{(n+1)(n+2)b_{n+2}}{(n+2)!} - \frac{3(n+1)b_{n+1}}{(n+1)!} 
  + \frac{2b_n}{n!} = 0 \nonumber \\
  b_{n+2} - 3b_{n+1} + 2b_n = 0 \label{difference_b_n}
\end{gather}
At first glance it appears that we have not reduced the order of the equation,
but writing it in terms of discrete derivatives
\[D^2 b_n - D b_n = 0 \]
shows that we now have a first order difference equation for $D b_n$.
The substitution $b_n = r^n$ in equation~\ref{difference_b_n} yields the
algebraic equation
\[r^2 - 3r + 2 = (r-1)(r-2) = 0.\]
Thus the solutions are $b_n = 1$ and $b_n = 2^n$.  Only the $b_n=2^n$ 
solution will give us another linearly independent solution for $a_n$. 
Thus the second solution for $a_n$ is $a_n = b_n / n! = 2^n / n!$.
The general solution to equation~\ref{difference_a_n} is then
\[\boxed{ a_n = c_1 \frac{1}{n!} + c_2 \frac{2^n}{n!}. } \]


\begin{Result} 
  Let $a_n = s_n$ be a homogeneous solution of a linear difference equation. 
  The substitution $a_n = s_n b_n$ will yield a difference equation for $b_n$ 
  that is of order one less than the equation for $a_n$. 
\end{Result} 



%%=========================================================================== 
%%%%%%%%%%%%%%%%%%%%%%%%%%%%%%%%%%%%%%%%%%%%%%%%%%%%%%%%%%%%%%%%%%%%%%%%%%%%%%%%%%
%%\section{*Euler Equations} 
%%\index{difference equations!Euler equation} 
%%The second order homogeneous Euler difference equation has the form
%%\[ (n+1)n D^2 a_n + p_1 n D a_n + p_0 a_n = 0.\] 
%%We look for solutions of the form 
%%\[a_n = \frac{\Gamma(n+r)}{\Gamma(n)}.\] 
%%Recall that $\Gamma(x+1) = x \Gamma(x)$, and for non-negative integral $n$,  
%%$\Gamma(n+1) = n!$. 
%%Now we 
%%find the finite derivatives of $a_n = \Gamma(n+r)/\Gamma(n)$. 
%%\begin{align*} 
%%D a_n   &= D \frac{\Gamma(n+r)}{\Gamma(n)} \\ 
%%       &= \frac{\Gamma(n+1+r)}{\Gamma(n+1)} -  
%%               \frac{\Gamma(n+r)}{\Gamma(n)} \\ 
%%       &= \frac{(n+r)\Gamma(n+r)}{n\Gamma(n)} -  
%%               \frac{n\Gamma(n+r)}{n\Gamma(n)} \\ 
%%       &= \frac{r\Gamma(n+r)}{n\Gamma(n)} \\ 
%%D^2a_n  &= \frac{r\Gamma(n+1+r)}{(n+1)\Gamma(n+1)} -  
%%               \frac{r\Gamma(n+r)}{n\Gamma(n)} \\ 
%%       &= \frac{r(n+r)\Gamma(n+r)}{(n+1)n\Gamma(n)} -  
%%               \frac{r(n+1)\Gamma(n+r)}{(n+1)n\Gamma(n)} \\ 
%%       &= \frac{r(r-1)\Gamma(n+r)}{(n+1)n\Gamma(n)} 
%%\end{align*} 
%%Substituting these formulas into the difference equation, 
%%\begin{gather*} 
%%(n+1)n \frac{r(r-1)\Gamma(n+r)}{(n+1)n\Gamma(n)} +  
%%       p_1 n \frac{r\Gamma(n+r)}{n\Gamma(n)} +  
%%       p_0 \frac{\Gamma(n+r)}{\Gamma(n)} = 0 \\ 
%%r(r-1) + p_1 r + p_0 = 0. \\ 
%%\intertext{Factoring this equation,} 
%%(r-r_1)(r-r_2) = 0. 
%%\end{gather*} 
%%If the roots are distinct, then the general solution to the difference  
%%equation is 
%%\[ a_n = c_1 \frac{\Gamma(n+r_1)}{\Gamma(n)} +  
%%               c_2 \frac{\Gamma(n+r_2)}{\Gamma(n)}. \] 
%%If the roots are not distinct, then we can use reduction of order to find 
%%the second linearly independent solution. 

%%For the $n^{t h}$ order Euler equation we use the same approach as for  
%%the second order equation. 






%%\begin{Result} 
%%Consider the second order Euler difference equation
%%\[ (n+1)n D^2 a_n + p_1 n D a_n + p_0 a_n = 0.\] 
%%The substitution
%%\[a_n = \frac{\Gamma(n+r)}{\Gamma(n)}\] 
%%will yield the algebraic equation
%%\[(r-r_1)(r-r_2) = 0.\] 
%%If the roots are distinct, the linearly independent solutions are 
%%\[ a_n = \frac{\Gamma(n+r_1)}{\Gamma(n)} \qquad \mathrm{and} \qquad 
%%       a_n = \frac{\Gamma(n+r_2)}{\Gamma(n)}.\] 
%%If the roots are not distinct, reduction of order can be used to find the 
%%second solution. 
%%\end{Result} 







%%\begin{Example} 
%%Consider the difference equation
%%\[(n^2 + n)a_{n+2} - (2n^2 + 4n)a_{n+1} + (n^2+3n+2)a_n = 0.\] 
%%First we write this equation in terms of finite derivatives. 
%%\begin{gather*} 
%%(n^2 + n)(a_{n+2}-2a_{n+1}+a_n) + (-2n^2 -4n + 2n^2 + 2n)a_{n+1} 
%%       + (n^2 + 3n + 2 - n^2 - n)a_n = 0 \\ 
%%(n+1)n D^2a_n -2n a_{n+1} + (2n+2) a_n = 0 \\ 
%%(n+1)n D^2a_n -2n (a_{n+1} - a_n) + (2n + 2 - 2n)a_n = 0 \\ 
%%(n+1)n D^2a_n -2n D_n + 2 a_n = 0  
%%\end{gather*} 
%%We see that this is a second order Euler equation.  The substitution 
%%$a_n = \Gamma(n+r)/\Gamma(n)$ yields the algebraic equation 
%%\[r(r-1) - 2r + 2 = 0, \] 
%%which has roots $r_1 = 1$, $r_2 = 2$.  Thus the general solution is 
%%\begin{gather*} 
%%a_n = c_1 \frac{\Gamma(n+1)}{\Gamma(n)} + c_2 %%\frac{\Gamma(n+2)}{\Gamma(n)}\\ 
%%\boxed{a_n = c_1 n + c_2 (n+1)n.} 
%%\end{gather*} 
%%\end{Example} 





\raggedbottom
%%=========================================================================== 
\exercises{
\pagebreak
\flushbottom
\section{Exercises} 






\begin{Exercise}
  \label{exercise fibonacci}
  Find a formula for the $n^{t h}$ term in the Fibonacci sequence
  $1,1,2,3,5,8,13,\ldots$. 
  \index{Fibonacci sequence} 

  \hintsolution{fibonacci}
\end{Exercise} 




\begin{Exercise} 
  \label{exercise an2=2nan}
  Solve the difference equation
  \[ 
  a_{n+2} = \frac{2}{n} a_n, \quad a_1 = a_2 = 1. 
  \] 

  \hintsolution{an2=2nan}
\end{Exercise} 









\raggedbottom 
}
%%=========================================================================== 
\hints{
\pagebreak
\flushbottom
\section{Hints} 







\begin{Hint}
  \label{hint fibonacci}
  The difference equation corresponding to the Fibonacci sequence is 
  \[a_{n+2} - a_{n+1} - a_n = 0, \qquad a_1 = a_2 = 1. \] 
\end{Hint} 






\begin{Hint} 
  \label{hint an2=2nan}
  Consider this exercise as two first order difference equations; one for  
  the even terms, one for the odd terms. 
\end{Hint} 








\raggedbottom 
}
%%=========================================================================== 
\solutions{
\pagebreak
\flushbottom
\section{Solutions} 








\begin{Solution} 
  \label{solution fibonacci}
  We can describe the Fibonacci sequence with the difference equation 
  \begin{gather*} 
    a_{n+2} - a_{n+1} - a_n = 0, \qquad a_1 = a_2 = 1. \\ 
    \intertext{With the substitution $a_n = r^n$ we obtain the equation} 
    r^2 - r - 1 = 0. \\ 
    \intertext{This equation has the two distinct roots} 
    r_1 = \frac{1 + \sqrt{5}}{2}, \qquad r_2 = \frac{1 - \sqrt{5}}{2}. \\ 
    \intertext{Thus the general solution is} 
    a_n = c_1 \left(\frac{1 + \sqrt{5}}{2}\right)^n +  
    c_2 \left(\frac{1 - \sqrt{5}}{2}\right)^n. 
  \end{gather*} 
  From the initial conditions we have 
  \begin{alignat*}{2} 
    &c_1 r_1 +      &c_2 r_2        &= 1 \\ 
    &c_1 r_1^2 +    &c_2 r_2^2      &= 1. 
  \end{alignat*} 
  Solving for $c_2$ in the first equation, 
  \[c_2 = \frac{1}{r_2}(1 - c_1 r_1).\] 
  We substitute this into the second equation. 
  \begin{gather*} 
    c_1r_1^2 + \frac{1}{r_2}(1 - c_1 r_1)r_2^2 = 1 \\ 
    c_1(r_1^2 - r_1r_2) = 1-r_2 
  \end{gather*} 
  \begin{align*} 
    c_1     &= \frac{1-r_2}{r_1^2 - r_1r_2} \\ 
    &= \frac{1 - \frac{1-\sqrt{5}}{2}} 
    {\frac{1+\sqrt{5}}{2} \sqrt{5}} \\ 
    &= \frac{\frac{1+\sqrt{5}}{2}} 
    {\frac{1+\sqrt{5}}{2} \sqrt{5}} \\ 
    &= \frac{1}{\sqrt{5}} 
  \end{align*} 
  Substitute this result into the equation for $c_2$. 
  \begin{align*} 
    c_2     &= \frac{1}{r_2}\left(1 - \frac{1}{\sqrt{5}} r_1\right) \\ 
    &= \frac{2}{1-\sqrt{5}} 
    \left(1 - \frac{1}{\sqrt{5}} \frac{1+\sqrt{5}}{2}\right) \\ 
    &= - \frac{2}{1-\sqrt{5}}  
    \left( \frac{1-\sqrt{5}}{2\sqrt{5}} \right) \\ 
    &= -\frac{1}{\sqrt{5}} 
  \end{align*} 
  Thus the $n^{t h}$ term in the Fibonacci sequence has the formula 
  \[ \boxed{ a_n = \frac{1}{\sqrt{5}} \left(\frac{1 + \sqrt{5}}{2}\right)^n - 
    \frac{1}{\sqrt{5}} \left(\frac{1 - \sqrt{5}}{2}\right)^n.} \] 
  
  It is interesting to note that although the Fibonacci sequence is defined 
  in terms of integers, one cannot express the formula form the $n^{t h}$  
  element in terms of rational numbers. 
\end{Solution} 






\begin{Solution} 
  \label{solution an2=2nan}
  We can consider  
  \[ a_{n+2} = \frac{2}{n} a_n, \quad a_1 = a_2 = 1 \] 
  to be a first order difference equation. 
  First consider the odd terms. 
  \begin{align*} 
    a_1     &= 1 \\ 
    a_3     &= \frac{2}{1} \\ 
    a_5     &= \frac{2}{3} \frac{2}{1} \\ 
    a_n     &= \frac{2^{(n-1)/2}}{(n-2)(n-4)\cdots(1)} 
  \end{align*} 
  For the even terms, 
  \begin{align*} 
    a_2     &= 1 \\ 
    a_4     &= \frac{2}{2} \\ 
    a_6     &= \frac{2}{4} \frac{2}{2} \\ 
    a_n     &= \frac{2^{(n-2)/2}}{(n-2)(n-4)\cdots(2)}. 
  \end{align*} 
  Thus 
  \[ \boxed{ a_n =  
    \begin{cases} 
      \frac{2^{(n-1)/2}}{(n-2)(n-4)\cdots(1)}  \quad &\mathrm{for odd}\ n \\ 
      \frac{2^{(n-2)/2}}{(n-2)(n-4)\cdots(2)}  \quad &\mathrm{for even}\ n. 
    \end{cases} } 
  \] 
\end{Solution} 








\raggedbottom
}
