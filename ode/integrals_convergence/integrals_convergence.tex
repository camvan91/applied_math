\flushbottom

%%
%% Add examples of principle value in the complex plane where you take half the
%% value of the residue.
%%




%%============================================================================
%%============================================================================
\chapter{Integrals and Convergence}



Never try to teach a pig to sing.  It wastes your time and annoys the pig.

\begin{flushright}
  -?
\end{flushright}





%%============================================================================
\section{Uniform Convergence of Integrals}
\index{convergence!of integrals}
\index{uniform convergence!of integrals}

Consider the improper integral
\[ \int_c^\infty f(x,t)\,\dd t.\]
The integral is convergent to $S(x)$ if, given any $\epsilon > 0$, there exists
$T(x, \epsilon)$ such that
\[ \left| \int_c^\tau f(x,t)\,\dd t - S(x) \right| < \epsilon \quad 
\mathrm{for all}\ \tau > T(x, \epsilon).\]
The sum is uniformly convergent if $T$ is independent of $x$.

Similar to the Weierstrass M-test for infinite sums we have a uniform
convergence test for integrals. 
If there exists a continuous function $M(t)$ such
that $|f(x,t)| \leq M(t) $ and $\int_c^\infty M(t)\,\dd t$ is convergent,
then $\int_c^\infty f(x,t)\,\dd t$ is uniformly convergent.

If $\int_c^\infty f(x,t)\,\dd t$ is uniformly convergent, we have the following properties:
\begin{itemize}
\item
  If $f(x,t)$ is continuous for $x \in [a,b]$ and $t \in [c,\infty)$ then for
  $a < x_0 < b$,
  \[ \lim_{x \to x_0} \int_c^\infty f(x,t)\,\dd t = 
  \int_c^\infty \left(\lim_{x \to x_0} f(x,t) \right)\,\dd t.\]
\item
  If $a \leq x_1 < x_2 \leq b$ then we can interchange the order of integration.
  \[ \int_{x_1}^{x_2} \left( \int_c^\infty f(x,t)\,\dd t \right)\,\dd x = 
  \int_c^\infty \left(\int_{x_1}^{x_2} f(x,t)\,\dd x\right)\,\dd t\]
\item
  If $\frac{\partial f}{\partial x}$ is continuous, then
  \[ \frac{\dd}{\dd x} \int_c^\infty f(x,t)\,\dd t = \int_c^\infty \frac{\partial}{\partial x}
  f(x,t)\,\dd t.\]
\end{itemize}










%%============================================================================
\section{The Riemann-Lebesgue Lemma}
\index{Riemann-Lebesgue lemma}

\begin{Result}
  If $\int_a^b |f(x)|\,\dd x$ exists, then
  \[ \int_a^b f(x) \sin(\lambda x)\,\dd x \to 0\ \mathrm{as}\ \lambda \to \infty.\]
\end{Result}


Before we try to justify the Riemann-Lebesgue lemma, we will need 
a preliminary result.  Let $\lambda$ be a positive constant.
\begin{align*}
  \left| \int_a^b \sin(\lambda x)\,\dd x \right|
  &= \left| \left[- \frac{1}{\lambda} \cos(\lambda x) \right]_a^b 
  \right| \\
  &\leq \frac{2}{\lambda}.
\end{align*}



We will prove the Riemann-Lebesgue lemma for the case when $f(x)$ has
limited total fluctuation on the interval $(a,b)$.  We can express $f(x)$
as the difference of two functions
\[ f(x) = \psi_+(x) - \psi_-(x),\]
where $\psi_+$ and $\psi_-$ are positive, increasing, bounded functions.

From the mean value theorem for positive, increasing functions, there
exists an $x_0$, $a \leq x_0 \leq b$, such that
\begin{align*}
  \left| \int_a^b \psi_+(x) \sin(\lambda x)\,\dd x \right|
  &= \left| \psi_+(b) \int_{x_0}^b \sin(\lambda x)\,\dd x \right| \\
  &\leq |\psi_+(b)| \frac{2}{\lambda}.
\end{align*}
Similarly,
\[ \left| \int_a^b \psi_-(x) \sin(\lambda x)\,\dd x \right|
\leq |\psi_-(b)| \frac{2}{\lambda}.\]
Thus 
\begin{align*}
  \left| \int_a^b f(x) \sin(\lambda x)\,\dd x \right|
  &\leq \frac{2}{\lambda} (|\psi_+(b)| + |\psi_-(b)|) \\
  &\to 0 \quad \mathrm{as}\ \lambda \to \infty.
\end{align*}










%%============================================================================
\section{Cauchy Principal Value}

\subsection{Integrals on an Infinite Domain}
The improper integral $\int_{-\infty}^\infty f(x)\,\dd x$ is defined
\[ \int_{-\infty}^\infty f(x)\,\dd x = \lim_{a \to -\infty} \int_{a}^0 f(x)\,\dd x
+ \lim_{b \to \infty} \int_0^b f(x)\,\dd x,\]
when these limits exist.  The Cauchy principal value of the integral is
defined
\[ \PV \int_{-\infty}^\infty f(x)\,\dd x = \lim_{a \to \infty} \int_{-a}^a f(x)\,\dd x.\]
The principal value may exist when the integral diverges.


\begin{Example}
  $\int_{-\infty}^\infty x\,\dd x$ diverges, but
  \[ \PV \int_{-\infty}^\infty x\,\dd x = \lim_{a \to \infty} \int_{-a}^a x\,\dd x
  = \lim_{a \to \infty} (0) = 0. \]
\end{Example}


If the improper integral converges, then the Cauchy principal value exists
and is equal to the value of the integral.  The principal value of the
integral of an odd function is zero.  If the principal value of the integral
of an even function exists, then the integral converges.


\subsection{Singular Functions}
Let $f(x)$ have a singularity at $x = 0$. Let $a$ and $b$ satisfy
$a < 0 < b$.  The integral of $f(x)$ is defined
\[ \int_a^b f(x)\,\dd x = \lim_{\epsilon_1 \to 0^-} \int_a^{\epsilon_1} f(x)\,\dd x
+ \lim_{\epsilon_2 \to 0^+} \int_{\epsilon_2}^b f(x)\,\dd x,\]
when the limits exist.  The Cauchy principal value of the integral is defined
\[\PV \int_a^b f(x)\,\dd x = \lim_{\epsilon \to 0^+} \left( \int_a^{-\epsilon}
  f(x)\,\dd x + \int_\epsilon^b f(x)\,\dd x \right),\]
when the limit exists.


\begin{Example}
  The integral
  \[ \int_{-1}^2 \frac{1}{x}\,\dd x \]
  diverges, but the principal value exists.
  \begin{align*}
    \PV \int_{-1}^2 \frac{1}{x}\,\dd x
    &= \lim_{\epsilon \to 0^+} \left( \int_{-1}^{-\epsilon} \frac{1}{x}\,\dd x
      + \int_\epsilon^2 \frac{1}{x}\,\dd x \right) \\
    &= \lim_{\epsilon \to 0^+} \left( -\int_\epsilon^1 \frac{1}{x}\,\dd x
      + \int_\epsilon^2 \frac{1}{x}\,\dd x \right) \\
    &= \int_1^2 \frac{1}{x}\,\dd x  \\
    &= \log 2
  \end{align*}
\end{Example}


















