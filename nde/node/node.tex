\flushbottom


%% CONTINUE




%%=============================================================================
%%=============================================================================
\chapter{Nonlinear Ordinary Differential Equations}












\raggedbottom
%%============================================================================
\pagebreak
\flushbottom
\section{Exercises}



%%-----------------------------------------------------------------------------
%%\begin{large}
%%\noindent
%%\textbf{}
%%\end{large}

%%A model set of equations to describe an epidemic, in which 
\begin{Exercise}
  A model set of equations to describe an epidemic, in which 
  $x(t)$ is the
  number infected, $y(t)$ is the number susceptible, is
  \[ 
  \frac{\dd x}{\dd t} = r x y - \gamma x, 
  \qquad \frac{\dd y}{\dd t} = - r x y + \beta, 
  \]
  where $r > 0$, $\beta \geq 0$, 
  $\gamma \geq 0$.  Initially $x = x_0$, 
  $y = y_0$ at $t = 0$.
  Directly from the equations, without using the phase plane:
  \begin{enumerate}
  \item
    Find the solution, $x(t)$, $y(t)$, in the case 
    $\beta = \gamma = 0$.
    %%
  \item
    Show for the case $\beta = 0$, $\gamma \neq 0$ 
    that $x(t)$ first decreases
    or increases according as $r y_0 < \gamma$ or 
    $r y_0 > \gamma$.
    Show that $x(t) \to 0$ as $t \to \infty$ in both cases.
    Find $x$ as a function of $y$.
    %%
  \item
    In the phase plane:
    Find the position of the singular point and its type when
    $\beta > 0$, $\gamma > 0$.
  \end{enumerate}
\end{Exercise}


%%\frac{\dd u}{\dd x} &= r u + v (1-v)(p-v), \qquad r > 0, \ 0 < p < 1, \\
\begin{Exercise}
  Find the singular points and their types for the system
  \begin{align*}
    \frac{\dd u}{\dd x} &= r u + v (1-v)(p-v), \qquad r > 0, \ 0 < p < 1, \\
    \frac{\dd v}{\dd x} &= u,
  \end{align*}
  which comes from one of our nonlinear diffusion problems.  
  Note that there is a solution with
  \[ 
  u = \alpha (1-v) 
  \]
  for special values of $\alpha$ and $r$.  
  Find $v(x)$ for this special case.
\end{Exercise}



%%Check that $r=1$ is a limit cycle for
\begin{Exercise}
  Check that $r=1$ is a limit cycle for
  \begin{align*}
    \frac{\dd x}{\dd t} &= -y + x(1-r^2) \\
    \frac{\dd y}{\dd t} &= x + y(1-r^2)
  \end{align*}
  ($r=x^2+y^2$), and that all solution curves spiral into it.
\end{Exercise}



%%\epsilon \dot{y} &= f(y) - x \\
\begin{Exercise}
  Consider
  \begin{align*}
    \epsilon \dot{y} &= f(y) - x \\
    \dot{x} &= y
  \end{align*}
  Introduce new coordinates, $R$, $\theta$ given by
  \begin{align*}
    x &= R \cos \theta \\
    y &= \frac{1}{\sqrt{\epsilon}} R \sin \theta
  \end{align*}
  and obtain the exact differential equations for $R(t)$, $\theta(t)$.
  Show that $R(t)$ continually increases with $t$ when $R \neq 0$.
  Show that $\theta(t)$ continually decreases when $R > 1$.
\end{Exercise}



%%One choice of the Lorenz equations is
\begin{Exercise}
  One choice of the Lorenz equations is
  \begin{align*}
    \dot{x} &= -10x+10y \\
    \dot{y} &= R x-y-x z \\
    \dot{z} &= -\frac{8}{3} z + x y
  \end{align*}
  Where $R$ is a positive parameter.
  \begin{enumerate}
    %%
  \item
    Invistigate the nature of the sigular point at $(0,0,0)$ by finding the
    eigenvalues and their behavior for all $0 < R < \infty$.
    %%
  \item
    Find the other singular points when $R > 1$.
    %%
  \item
    Show that the appropriate eigenvalues for these other singular points
    satisfy the cubic
    \[ 
    3\lambda^3 + 41 \lambda^2 + 8(10+R)\lambda + 160 (R-1) = 0.
    \]
    %%
  \item
    There is a special value of $R$, call it $R_c$, for which the cubic has
    two pure imaginary roots, $\pm \imath \mu$ say.  Find $R_c$ and $\mu$; then
    find the third root.
  \end{enumerate}
\end{Exercise}



%%In polar coordinates $(r,\phi)$, Einstein's equations lead to the 
\begin{Exercise}
  In polar coordinates $(r,\phi)$, Einstein's equations lead to the 
  equation
  \[ 
  \frac{\dd^2 v}{\dd \phi^2} + v = 1 + \epsilon v^2, \quad v = \frac{1}{r}, 
  \]
  for planetary orbits.  For Mercury, $\epsilon=8 \times 10^{-8}$.
  When $\epsilon=0$ (Newtonian theory) the orbit is given by
  \[ v=1+A\cos\phi, \ \mathrm{period}\ 2\pi.\]
  Introduce $\theta=\omega\phi$ and use perturbation expansions for $v(\theta)$
  and $\omega$ in powers of $\epsilon$ to find the corrections proportional
  to $\epsilon$.

  [$A$ is not small; $\epsilon$ is the small parameter].
\end{Exercise}



%%\ddot{x} +\omega_0^2 x+\alpha x^2=0, \quad x=a, \dot{x}=0\ \mathrm{at}\ t=0
\begin{Exercise}
  Consider the problem
  \[ 
  \ddot{x} +\omega_0^2 x+\alpha x^2=0, \quad x=a, \dot{x}=0\ \mathrm{at}\ t=0
  \]
  Use expansions
  \begin{align*}
    x&=a\cos\theta+a^2 x_2(\theta)+a^3x_3(\theta)+\cdots,\ \theta=\omega t \\
    \omega &=\omega_0+a^2\omega_2+\cdots,
  \end{align*}
  to find a periodic solution and its natural frequency $\omega$.

  Note that, with the expansions given, there are no ``secular term'' troubles
  in the determination of $x_2(\theta)$, but $x_2(\theta)$ is needed in the
  subsequent determination of $x_3(\theta)$ and $\omega$.

  Show that a term $a\omega_1$ in the expansion for $\omega$ would have 
  caused trouble, so $\omega_1$ would have to be taken equal to zero.
\end{Exercise}





%%Consider the linearized traffic problem
\begin{Exercise}
  Consider the linearized traffic problem
  \begin{gather*}
    \frac{\dd p_n(t)}{\dd t} = \alpha \left[ p_{n-1}(t) - p_n(t)\right], \quad n \geq 1,\\
    p_n(0) = 0, \quad n \geq 1, \\
    p_0(t) = a e^{\imath \omega t}, \quad t > 0.
  \end{gather*}
  (We take the imaginary part of $p_n(t)$ in the final answers.)
  \begin{enumerate}
    %%
  \item
    Find $p_1(t)$ directly from the equation for $n=1$ and note the behavior as
    $t \to \infty$.
    %%
  \item
    Find the generating function
    \[ G(s,t) = \sum_{n = 1}^\infty p_n(t) s^n. \]
    %%
  \item
    Deduce that 
    \[ p_n(t) \sim A_n e^{\imath \omega t}, \quad \mathrm{as}\ t \to \infty, \]
    and find the expression for $A_n$.  Find the imaginary part of this $p_n(t)$.
  \end{enumerate}
\end{Exercise}



%%\frac{\dd}{\dd t} p_n(t+\tau) = \alpha [p_{n-1}(t) - p_n(t) ] \quad n\geq 1,
\begin{Exercise}
  \begin{enumerate}
    %%
  \item
    For the equation modified with a reaction time, namely
    \[ 
    \frac{\dd}{\dd t} p_n(t+\tau) = \alpha [p_{n-1}(t) - p_n(t) ] \quad n\geq 1,
    \]
    find a solution of the form in 1(c) by direct substitution in the 
    equation.  Again take its imaginary part.
    %%
  \item
    Find a condition that the disturbance is stable, i.e. $p_n(t)$ remains
    bounded as $n \to \infty$.
    %%
  \item
    In the stable case show that the disturbance is wave-like and find 
    the wave velocity.
  \end{enumerate}
\end{Exercise}






\raggedbottom
%%============================================================================
\pagebreak
\flushbottom
\section{Hints}



%%-----------------------------------------------------------------------------
%%\begin{large}
%%\noindent
%%\textbf{}
%%\end{large}

%%A model set of equations to describe an epidemic, in which 
\begin{Hint}
\end{Hint}


%%\frac{\dd u}{\dd x} &= r u + v (1-v)(p-v), \qquad r > 0, \ 0 < p < 1, \\
\begin{Hint}
\end{Hint}



%%Check that $r=1$ is a limit cycle for
\begin{Hint}
\end{Hint}



%%\epsilon \dot{y} &= f(y) - x \\
\begin{Hint}
\end{Hint}



%%One choice of the Lorenz equations is
\begin{Hint}
\end{Hint}



%%In polar coordinates $(r,\phi)$, Einstein's equations lead to the 
\begin{Hint}
\end{Hint}



%%\ddot{x} +\omega_0^2 x+\alpha x^2=0, \quad x=a, \dot{x}=0\ \mathrm{at}\ t=0
\begin{Hint}
\end{Hint}



%%Consider the linearized traffic problem
\begin{Hint}
\end{Hint}



%%\frac{\dd}{\dd t} p_n(t+\tau) = \alpha [p_{n-1}(t) - p_n(t) ] \quad n\geq 1,
\begin{Hint}
\end{Hint}








\raggedbottom
%%============================================================================
\pagebreak
\flushbottom
\section{Solutions}



%%-----------------------------------------------------------------------------
%%\begin{large}
%%\noindent
%%\textbf{}
%%\end{large}



%%A model set of equations to describe an epidemic, in which 
\begin{Solution}
  \begin{enumerate}
  \item 
    When $\beta = \gamma = 0$ the equations are
    \[ \frac{\dd x}{\dd t} = r x y, \qquad \frac{\dd y}{\dd t} = - r x y. \]
    Adding these two equations we see that
    \begin{gather*}
      \frac{\dd x}{\dd t} = - \frac{\dd y}{\dd t}. \\
      \intertext{Integrating and applying the initial conditions $x(0)=x_0$ and
        $y(0)=y_0$ we obtain}
      x = x_0 + y_0 - y \\
      \intertext{Substituting this into the differential equation for $y$,}
      \frac{\dd y}{\dd t} = -r (x_0 + y_0 - y) y \\
      \frac{\dd y}{\dd t} = -r (x_0 + y_0) y + r y^2. \\
      \intertext{We recognize this as a Bernoulli equation and make the substitution
        $u = y^{-1}$.}
      - y^{-2} \frac{\dd y}{\dd t} = r (x_0 + y_0) y^{-1} - r \\
      \frac{\dd u}{\dd t} = r (x_0 + y_0) u - r \\
      \frac{\dd}{\dd t} \left( e^{-r(x_0+y_0) t} u \right) = -r e^{-r(x_0+y_0)t} \\
      u = e^{r(x_0+y_0)t} \int^t -r e^{-r(x_0+y_0)t}\,d t
      + c e^{r(x_0+y_0)t} \\
      u = \frac{1}{x_0+y_0} + c e^{r(x_0+y_0)t} \\
      y = \left(\frac{1}{x_0+y_0} + c e^{r(x_0+y_0)t} \right)^{-1} \\
      \intertext{Applying the initial condition for $y$,}
      \left( \frac{1}{x_0+y_0} + c \right)^{-1} = y_0 \\
      c = \frac{1}{y_0} - \frac{1}{x_0+y_0}. \\
      \intertext{The solution for $y$ is then}
      y = \left[ \frac{1}{x_0+y_0} + \left( \frac{1}{y_0} - \frac{1}{x_0+y_0} \right)
        e^{r(x_0+y_0)t} \right]^{-1}
    \end{gather*}
    Since $x = x_0 + y_0 - y$, the solution to the system of differential 
    equations is
    \[ \boxed{ x = x_0 + y_0 - \left[ \frac{1}{y_0} + \frac{1}{x_0+y_0}
        \left(1 - e^{r(x_0+y_0)t} \right) \right]^{-1}, \quad
      y = \left[ \frac{1}{y_0} + \frac{1}{x_0+y_0} \left(1 - 
          e^{r(x_0+y_0)t} \right) \right]^{-1}. } \]
    %%
    %%
  \item
    For $\beta = 0$, $\gamma \neq 0$, the equation for $x$ is
    \[ \dot{x} = r x y - \gamma x. \]
    At $t = 0$, 
    \[ \dot{x}(0) = x_0 (r y_0 - \gamma). \]
    Thus we see that if $ry_0 < \gamma$, $x$ is initially decreasing.
    If $r y_0 > \gamma$, $x$ is initially increasing.





    Now to show that $x(t) \to 0$ as $t \to \infty$.  First note that if the 
    initial conditions satisfy $x_0, y_0 > 0$ then $x(t), y(t) > 0$ for all
    $t \geq 0$ because the axes are a seqaratrix.  
    $y(t)$ is is a strictly decreasing function of time.  Thus we
    see that at some time the quantity $x(r y - \gamma)$ will become negative.
    Since $y$ is decreasing, this quantity will remain negative.  Thus after
    some time, $x$ will become a strictly decreasing quantity.  Finally
    we see that regardless of the initial conditions,
    (as long as they are positive), $x(t) \to 0$ as $t \to \infty$.









    Taking the ratio of the two differential equations,
    \begin{align*}
      \frac{\dd x}{\dd y} &= -1 + \frac{\gamma}{r y}. \\
      x &= -y + \frac{\gamma}{r} \ln y + c \\
      \intertext{Applying the intial condition,}
      x_0 &= -y_0 + \frac{\gamma}{r} \ln y_0 + c \\
      c &= x_0 + y_0 - \frac{\gamma}{r} \ln y_0.
    \end{align*}
    Thus the solution for $x$ is
    \[ \boxed{ x = x_0 + (y_0 - y) + \frac{\gamma}{r} \ln \left( \frac{y}{y_0}
      \right) . } \]
    %%
    %%
  \item
    When $\beta > 0$ and $\gamma > 0$ the system of equations is
    \begin{align*}
      \dot{x} &= r x y - \gamma x \\
      \dot{y} &= -r x y + \beta.
    \end{align*}
    The equilibrium solutions occur when
    \begin{align*}
      x (r y - \gamma) &= 0 \\
      \beta - r x y &= 0.
    \end{align*}
    Thus the singular point is
    \[ \boxed{ x = \frac{\beta}{\gamma}, \qquad y = \frac{\gamma}{r}. } \]
    Now to classify the point.  We make the substitution 
    $u = (x-\frac{\beta}{\gamma})$, $v = (y - \frac{\gamma}{r})$.
    \begin{align*}
      \dot{u} &= r \left( u + \frac{\beta}{\gamma} \right) \left( v + 
        \frac{\gamma}{r} \right) - \gamma \left( u + \frac{\beta}{\gamma}
      \right) \\
      \dot{v} &= -r \left(u + \frac{\beta}{\gamma} \right) \left( v + 
        \frac{\gamma}{r} \right) + \beta
    \end{align*}
    \begin{align*}
      \dot{u} &= \frac{r \beta}{\gamma} v + r u v \\
      \dot{v} &= -\gamma u - \frac{r \beta}{\gamma} v - r u v
    \end{align*}
    The linearized system is
    \begin{align*}
      \dot{u} &= \frac{r \beta}{\gamma} v \\
      \dot{v} &= -\gamma u - \frac{r \beta}{\gamma} v
    \end{align*}
    Finding the eigenvalues of the linearized system,
    \[
    \begin{vmatrix}
      \lambda      &       -\frac{r \beta}{\gamma} \\
      \gamma      &       \lambda + \frac{r \beta}{\gamma}
    \end{vmatrix}
    = \lambda^2 + \frac{r \beta}{\gamma} \lambda + r \beta = 0
    \]
    \[ \lambda = \frac{ -\frac{r \beta}{\gamma} \pm 
      \sqrt{ (\frac{r \beta}{\gamma})^2 - 4 r \beta}}{2} \]
    Since both eigenvalues have negative real part, we see that the singular
    point is asymptotically stable.
    A plot of the vector field for $r = \gamma = \beta = 1$ is attached.
    We note that there appears to be a stable singular point at $x = y = 1$
    which corroborates the previous results.
  \end{enumerate}
\end{Solution}



%%\frac{\dd u}{\dd x} &= r u + v (1-v)(p-v), \qquad r > 0, \ 0 < p < 1, \\
\begin{Solution}
  The singular points are
  \[ u=0, v=0, \qquad u=0, v=1, \qquad u=0, v=p. \]

  \textbf{The point $\mathbf{u = 0, v = 0}$.}
  The linearized system about $u=0$, $v=0$ is 
  \begin{align*}
    \frac{\dd u}{\dd x} &= r u \\
    \frac{\dd v}{\dd x} &= u.
  \end{align*}
  The eigenvalues are
  \[
  \begin{vmatrix}
    \lambda-r    &       0       \\
    -1              &       \lambda
  \end{vmatrix}
  = \lambda^2 - r \lambda = 0.
  \]
  \[ \lambda = 0, r. \]
  Since there are positive eigenvalues, this point is a source.  The critical 
  point is unstable.

  \textbf{The point $\mathbf{u = 0, v = 1}$.}
  Linearizing the system about $u=0$, $v = 1$, we make the substitution
  $w = v-1$.
  \begin{align*}
    \frac{\dd u}{\dd x} &= r u + (w+1)(-w)(p-1-w) \\
    \frac{\dd w}{\dd x} &= u
  \end{align*}
  \begin{align*}
    \frac{\dd u}{\dd x} &= r u + (1-p) w \\
    \frac{\dd w}{\dd x} &= u
  \end{align*}
  \[ 
  \begin{vmatrix}
    \lambda-r    &       (p-1)   \\
    -1              &       \lambda
  \end{vmatrix}
  = \lambda^2 - r \lambda + p -1 = 0 \]
  \[ \lambda = \frac{r \pm \sqrt{r^2 - 4 (p-1)}}{2} \]
  Thus we see that this point is a saddle point.
  The critical point is unstable.


  \textbf{The point $\mathbf{u = 0, v = p}$.}
  Linearizing the system about $u=0$, $v=p$, we make the substitution
  $w=v-p$.
  \begin{align*}
    \frac{\dd u}{\dd x} &= r u + (w+p)(1-p-w)(-w) \\
    \frac{\dd w}{\dd x} &= u
  \end{align*}
  \begin{align*}
    \frac{\dd u}{\dd x} &= r u + p(p-1)w \\
    \frac{\dd w}{\dd x} &= u
  \end{align*}
  \[
  \begin{vmatrix}
    \lambda-r    &       p(1-p)  \\
    -1              &       \lambda
  \end{vmatrix}
  = \lambda^2 - r \lambda + p(1-p) = 0 \]
  \[ \lambda = \frac{r \pm \sqrt{r^2 - 4p(1-p)}}{2} \]
  Thus we see that this point is a source.
  The critical point is unstable.




  \textbf{The solution of for special values of $\boldsymbol{\alpha}$ 
  and $\mathbf{r}$.}
  Differentiating $u = \alpha v (1-v)$,
  \[ \frac{\dd u}{\dd v} = \alpha - 2 \alpha v. \]

  Taking the ratio of the two differential equations,
  \begin{align*}
    \frac{\dd u}{\dd v}   
    &= r + \frac{v(1-v)(p-v)}{u} \\
    &= r + \frac{v(1-v)(p-v)}{\alpha v (1-v)} \\
    &= r + \frac{(p-v)}{\alpha}
  \end{align*}
  Equating these two expressions,
  \[ \alpha - 2 \alpha v = r + \frac{p}{\alpha} - \frac{v}{\alpha}. \]
  Equating coefficients of $v$, we see that $\alpha = \frac{1}{\sqrt{2}}$.
  \[\frac{1}{\sqrt{2}} = r + \sqrt{2} p \]
  Thus we have the solution $u = \frac{1}{\sqrt{2}} v (1-v)$ when 
  $r = \frac{1}{\sqrt{2}} - \sqrt{2} p$.
  In this case, the differential equation for $v$ is
  \begin{gather*}
    \frac{\dd v}{\dd x} = \frac{1}{\sqrt{2}} v (1-v) \\
    -v^{-2} \frac{\dd v}{\dd x} = - \frac{1}{\sqrt{2}} v^{-1} + \frac{1}{\sqrt{2}} \\
    \intertext{We make the change of variablles $y = v^{-1}$.}
    \frac{\dd y}{\dd x} = -\frac{1}{\sqrt{2}} y + \frac{1}{\sqrt{2}} \\
    \frac{\dd}{\dd x} \left( e^{x / \sqrt{2}} y \right) 
    = \frac{e^{x / \sqrt{2}}}{\sqrt{2}} \\
    y = e^{-x / \sqrt{2}} \int^x \frac{e^{x / \sqrt{2}}}{\sqrt{2}} \,d x 
    + c e^{-x / \sqrt{2}} \\
    y = 1 + c e^{-x / \sqrt{2}} \\
    \intertext{The solution for $v$ is}
    \boxed{ v(x) = \frac{1}{1 + c e^{-x / \sqrt{2}}}. }
  \end{gather*}
\end{Solution}



%%Check that $r=1$ is a limit cycle for
\begin{Solution}
  We make the change of variables
  \begin{align*}
    x       &= r \cos \theta \\
    y       &= r \sin \theta.
  \end{align*}
  Differentiating these expressions with respect to time,
  \begin{align*}
    \dot{x} &= \dot{r} \cos \theta - r \dot{\theta} \sin \theta \\
    \dot{y} &= \dot{r} \sin \theta + r \dot{\theta} \cos \theta.
  \end{align*}
  Substituting the new variables into the pair of differential equations,
  \begin{align*}
    \dot{r} \cos \theta - r \dot{\theta} \sin \theta 
    &= -r \sin \theta + r \cos \theta (1-r^2) \\
    \dot{r} \sin \theta + r \dot{\theta} \cos \theta
    &= r \cos \theta + r \sin \theta(1-r^2).
  \end{align*}
  Multiplying the equations by $\cos \theta$ and $\sin \theta$ and taking their
  sum and difference yields
  \begin{align*}
    \dot{r} &= r(1-r^2) \\
    r \dot{\theta} &= r.
  \end{align*}
  We can integrate the second equation.
  \begin{align*}
    \dot{r} &= r(1-r^2) \\
    \theta      &= t + \theta_0
  \end{align*}
  At this point we could note that $\dot{r} > 0$ in $(0,1)$ and $\dot{r} < 0$
  in $(1,\infty)$.  Thus if $r$ is not initially zero, then the solution tends
  to $r=1$.

  Alternatively, we can solve the equation for $r$ exactly.
  \begin{gather*}
    \dot{r} = r - r^3 \\
    \frac{\dot{r}}{r^3} = \frac{1}{r^2} - 1 \\
    \intertext{We make the change of variables $u = 1/r^2$.}
    -\frac{1}{2} \dot{u} = u - 1 \\
    \dot{u} + 2 u = 2 \\
    u = e^{-2t} \int^t 2 e^{2t} \,d t + c e^{-2t} \\
    u = 1 + c e^{-2t} \\
    r = \frac{1}{\sqrt{1 + c e^{-2t}}}
  \end{gather*}
  Thus we see that if $r$ is initiall nonzero, the solution tends to 1 as 
  $t \to \infty$.
\end{Solution}



%%\epsilon \dot{y} &= f(y) - x \\
\begin{Solution}
  The set of differential equations is
  \begin{align*}
    \epsilon \dot{y} &= f(y) - x \\
    \dot{x} &= y.
  \end{align*}
  We make the change of variables
  \begin{align*}
    x &= R \cos \theta \\
    y &= \frac{1}{\sqrt{\epsilon}} R \sin \theta
  \end{align*}
  Differentiating $x$ and $y$,
  \begin{align*}
    \dot{x} &= \dot{R} \cos \theta - R \dot{\theta} \sin \theta \\
    \dot{y} &= \frac{1}{\sqrt{\epsilon}} \dot{R} \sin \theta + 
    \frac{1}{\sqrt{\epsilon}} R \dot{\theta} \cos \theta.
  \end{align*}
  The pair of differential equations become
  \begin{align*}
    \sqrt{\epsilon} \dot{R} \sin \theta +
    \sqrt{\epsilon} R \dot{\theta} \cos \theta &= 
    f\left(\frac{1}{\sqrt{\epsilon}} R \sin\theta \right) -R \cos \theta \\
    \dot{R} \cos \theta - R \dot{\theta} \sin \theta &= \frac{1}{\sqrt{\epsilon}}
    R \sin \theta.
  \end{align*}
  \begin{align*}
    \dot{R} \sin \theta + R \dot{\theta} \cos \theta &= 
    - \frac{1}{\sqrt{\epsilon}} R \cos \theta 
    \frac{1}{\sqrt{\epsilon}}
    f\left(\frac{1}{\sqrt{\epsilon}} R \sin\theta \right)\\ 
    \dot{R} \cos \theta - R \dot{\theta} \sin \theta &= \frac{1}{\sqrt{\epsilon}}
    R \sin \theta.
  \end{align*}
  Multiplying by $\cos\theta$ and $\sin\theta$ and taking the sum and difference
  of these differential equations yields
  \begin{align*}
    \dot{R} &= \frac{1}{\sqrt{\epsilon}} \sin\theta 
    f\left(\frac{1}{\sqrt{\epsilon}} R \sin\theta \right) \\
    R \dot{\theta} &= - \frac{1}{\sqrt{\epsilon}}R + \frac{1}{\sqrt{\epsilon}}
    \cos\theta f\left(\frac{1}{\sqrt{\epsilon}} R \sin\theta \right).
  \end{align*}
  Dividing by $R$ in the second equation,
  \begin{align*}
    \dot{R} &= \frac{1}{\sqrt{\epsilon}} \sin\theta 
    f\left(\frac{1}{\sqrt{\epsilon}} R \sin\theta \right) \\
    \dot{\theta} &= - \frac{1}{\sqrt{\epsilon}} + \frac{1}{\sqrt{\epsilon}}
    \frac{\cos\theta}{R} 
    f\left(\frac{1}{\sqrt{\epsilon}} R \sin\theta \right).
  \end{align*}

  We make the assumptions that $0 < \epsilon < 1$ and that $f(y)$ 
  is an odd function that is nonnegative for positive $y$ and 
  satisfies $|f(y)| \leq 1$ for all $y$.

  Since $\sin\theta$ is odd,
  \[ \sin\theta f\left(\frac{1}{\sqrt{\epsilon}} R \sin\theta \right) \]
  is nonnegative.  Thus $R(t)$ continually increases with $t$ when $R \neq 0$.

  If $R > 1$ then
  \begin{align*}
    \left|\frac{\cos\theta}{R} f\left(\frac{1}{\sqrt{\epsilon}} R \sin\theta 
      \right)\right|
    &\leq \left|f\left(\frac{1}{\sqrt{\epsilon}} R \sin\theta 
      \right)\right| \\
    &\leq 1.
  \end{align*}
  Thus the value of $\dot{\theta}$,
  \[ - \frac{1}{\sqrt{\epsilon}} + \frac{1}{\sqrt{\epsilon}}\frac{\cos\theta}{R}
  f\left(\frac{1}{\sqrt{\epsilon}} R \sin\theta \right), \]
  is always nonpositive.  Thus $\theta(t)$ continually decreases with $t$.
\end{Solution}



%%One choice of the Lorenz equations is
\begin{Solution}
  \begin{enumerate}
    %%
    %%
    %%
  \item
    Linearizing the Lorentz equations about $(0,0,0)$ yields 
    \[ 
    \begin{pmatrix}
      \dot{x} \\
      \dot{y} \\
      \dot{z}
    \end{pmatrix}
    =
    \begin{pmatrix}
      -10     & 10    & 0     \\
      R       & -1    & 0     \\
      0       & 0     & -8/3
    \end{pmatrix}
    \begin{pmatrix}
      x \\
      y \\
      z
    \end{pmatrix}
    \]
    The eigenvalues of the matrix are
    \begin{align*}
      \lambda_1 &= -\frac{8}{3}, \\
      \lambda_2 &= \frac{-11-\sqrt{81+40R}}{2} \\
      \lambda_3 &= \frac{-11+\sqrt{81+40R}}{2} .
    \end{align*}

    There are three cases for the eigenvalues of the linearized system.
    \begin{description}
    \item{$\mathbf{R < 1}$.}
      There are three negative, real eigenvalues.  In the linearized and also the
      nonlinear system, the origin is a stable, sink.
    \item{$\mathbf{R=1}$.}
      There are two negative, real eigenvalues and one zero eigenvalue.  In the
      linearized system the origin is stable and has a center manifold plane.
      The linearized system does not tell us if the nonlinear system is 
      stable or unstable.
    \item{$\mathbf{R>1}$.}
      There are two negative, real eigenvalues, and one positive, real eigenvalue.
      The origin is a saddle point.
    \end{description}
    %%
    %%
    %%
  \item
    The other singular points when $R>1$ are
    \[
    \left(\pm\sqrt{\frac{8}{3}(R-1)},\ \pm\sqrt{\frac{8}{3}(R-1)},\ R-1\right).
    \]
    %%
    %%
    %%
  \item
    Linearizing about the point
    \[\left(\sqrt{\frac{8}{3}(R-1)},\ \sqrt{\frac{8}{3}(R-1)},\ R-1\right)\]
    yields
    \[
    \begin{pmatrix}
      \dot{X} \\
      \dot{Y} \\
      \dot{Z}
    \end{pmatrix}
    =
    \begin{pmatrix}
      -10     & 10    & 0     \\
      1       & -1    & -\sqrt{\frac{8}{3}(R-1)} \\
      \sqrt{\frac{8}{3}(R-1)} & \sqrt{\frac{8}{3}(R-1)} & -\frac{8}{3}
    \end{pmatrix}
    \begin{pmatrix}
      X \\
      Y \\
      Z
    \end{pmatrix}
    \]
    The characteristic polynomial of the matrix is
    \[ \lambda^3 + \frac{41}{3} \lambda^2 + \frac{8(10+R)}{3} \lambda
    + \frac{160}{3}(R-1). \]
    Thus the eigenvalues of the matrix satisfy the polynomial,
    \[ 3 \lambda^3 + 41 \lambda^2 + 8(10+R) \lambda + 160(R-1) = 0. \]

    Linearizing about the point
    \[\left(-\sqrt{\frac{8}{3}(R-1)},\ -\sqrt{\frac{8}{3}(R-1)},\ R-1\right)\]
    yields
    \[
    \begin{pmatrix}
      \dot{X} \\
      \dot{Y} \\
      \dot{Z}
    \end{pmatrix}
    =
    \begin{pmatrix}
      -10     & 10    & 0     \\
      1       & -1    & \sqrt{\frac{8}{3}(R-1)} \\
      -\sqrt{\frac{8}{3}(R-1)} & -\sqrt{\frac{8}{3}(R-1)} & -\frac{8}{3}
    \end{pmatrix}
    \begin{pmatrix}
      X \\
      Y \\
      Z
    \end{pmatrix}
    \]
    The characteristic polynomial of the matrix is
    \[ \lambda^3 + \frac{41}{3} \lambda^2 + \frac{8(10+R)}{3} \lambda
    + \frac{160}{3}(R-1). \]
    Thus the eigenvalues of the matrix satisfy the polynomial,
    \[ 
    3 \lambda^3 + 41 \lambda^2 + 8(10+R) \lambda + 160(R-1) = 0. 
    \]
    %%
    %%
    %%
  \item
    If the characteristic polynomial has two pure imaginary roots $\pm \imath \mu$
    and one real root, then it has the form
    \[ (\lambda-r)(\lambda^2+\mu^2) = \lambda^3 - r \lambda^2 + \mu^2 \lambda
    - r \mu^2. \]
    Equating the $\lambda^2$ and the $\lambda$ term with the characteristic
    polynomial yields
    \[ r = -\frac{41}{3}, \qquad \mu = \sqrt{\frac{8}{3}(10+R)}. \]
    Equating the constant term gives us the equation
    \[ \frac{41}{3} \frac{8}{3} (10+R_c) = \frac{160}{3} (R_c-1) \]
    which has the solution
    \[ R_c = \frac{470}{19}. \]
    For this critical value of $R$ the characteristic polynomial has the
    roots
    \begin{align*}
      \lambda_1 &= -\frac{41}{3} \\
      \lambda_2 &= \frac{4}{19} \sqrt{2090} \\
      \lambda_3 &= - \frac{4}{19} \sqrt{2090} .
    \end{align*}
  \end{enumerate}
\end{Solution}



%%In polar coordinates $(r,\phi)$, Einstein's equations lead to the 
\begin{Solution}
  The form of the perturbation expansion is 
  \begin{align*}
    v(\theta) &= 1 + A\cos\theta + \epsilon u(\theta) + \mathcal{O}(\epsilon^2) \\
    \theta &= (1+\epsilon\omega_1 + \mathcal{O}(\epsilon^2))\phi.
  \end{align*}
  Writing the derivatives in terms of $\theta$,
  \begin{align*}
    \frac{\dd}{\dd \phi} &= (1 + \epsilon \omega_1 + \cdots)\frac{\dd}{\dd \theta} \\
    \frac{\dd^2}{\dd \phi^2} &= (1 + 2 \epsilon \omega_1 + \cdots) \frac{\dd^2}{\dd \theta^2}.
  \end{align*}
  Substituting these expressions into the differential equation for $v(\phi)$,
  \begin{align*}
    &\left[1+2\epsilon\omega_1+\mathcal{O}(\epsilon^2)\right]\left[-A\cos\theta
      +\epsilon u'' +\mathcal{O}(\epsilon^2)\right] + 1+A\cos\theta
    +\epsilon u(\theta) +\mathcal{O}(\epsilon^2) \\
    &\qquad\qquad
    =1+\epsilon\left[1+2A\cos\theta+A^2\cos^2\theta+\mathcal{O}(\epsilon)
    \right] 
  \end{align*}
  \begin{gather*}
    \epsilon u'' + \epsilon u - 2\epsilon\omega_1 A\cos\theta = \epsilon
    + 2\epsilon A\cos\theta +\epsilon A^2\cos^2\theta 
    +\mathcal{O}(\epsilon^2). \\
    \intertext{Equating the coefficient of $\epsilon$,}
    u''+u=1+2\epsilon(1+\omega_1)A\cos\theta+\frac{1}{2}A^2(\cos 2\theta+1) \\
    u''+u=(1+\frac{1}{2}A^2) + 2\epsilon(1+\omega_1)A\cos\theta
    +\frac{1}{2}A^2\cos 2\theta.
  \end{gather*}
  To avoid secular terms, we must have $\omega_1=-1$.  A particular solution
  for $u$ is
  \[ u=1+\frac{1}{2}A^2-\frac{1}{6}A^2\cos 2\theta. \]
  The the solution for $v$ is
  \[ 
  \boxed{ 
    v(\phi)=1+A\cos((1-\epsilon)\phi)+\epsilon\left[1+\frac{1}{2}A^2
      -\frac{1}{6}A^2\cos(2(1-\epsilon)\phi)\right] +\mathcal{O}(\epsilon^2).
    }
  \]
\end{Solution}



%%\ddot{x} +\omega_0^2 x+\alpha x^2=0, \quad x=a, \dot{x}=0\ \mathrm{at}\ t=0
\begin{Solution}
  Substituting the expressions for $x$ and $\omega$ into the differential 
  equations yields
  \[ a^2\left[\omega_0^2\left(\frac{\dd^2 x_2}{\dd \theta^2}+x_2\right) 
    + \alpha \cos^2\theta \right] 
  + a^3\left[\omega_0^2\left(\frac{\dd^2 x_3}{\dd \theta^2} + x_3\right) 
    -2\omega_0 \omega_2 \cos\theta + 2\alpha x_2 \cos\theta\right]
  + \mathcal{O}(a^4) = 0 \]
  Equating the coefficient of $a^2$ gives us the differential equation
  \[ \frac{\dd^2 x_2}{\dd \theta^2} + x2 = -\frac{\alpha}{2\omega_0^2}(1+\cos 2\theta). \]
  The solution subject to the initial conditions $x_2(0)=x_2'(0)=0$ is
  \[ x_2 = \frac{\alpha}{6\omega_0^2}(-3+2\cos\theta+\cos 2\theta). \]

  Equating the coefficent of $a^3$ gives us the differential equation
  \[ \omega_0^2\left(\frac{\dd^2 x_3}{\dd \theta^2} + x_3 \right) 
  + \frac{\alpha^2}{3\omega_0^2}
  -\left(2\omega_0\omega_2+\frac{5\alpha^2}{6\omega_0^2}\right)\cos\theta
  +\frac{\alpha^2}{3\omega_0^2}\cos 2\theta
  +\frac{\alpha^2}{6\omega_0^2}\cos 3\theta = 0. \]
  To avoid secular terms we must have
  \[ \omega_2 = -\frac{5\alpha^2}{12\omega_0}. \]
  Solving the differential equation for $x_3$ subject to the intial conditions
  $x_3(0) = x_3'(0) = 0$,
  \[ x_3 = \frac{\alpha^2}{144\omega_0^4}(-48+29\cos\theta+16\cos 2\theta
  + 3\cos 3\theta). \]
  Thus our solution for $x(t)$ is
  \[ \boxed{ x(t) = a \cos\theta 
    +a^2\left[\frac{\alpha}{6\omega_0^2}
      (-3+2\cos\theta+\cos 2\theta)\right]
    + a^3\left[\frac{\alpha^2}{144\omega_0^4}
      (-48+29\cos\theta+16\cos 2\theta + 3\cos 3\theta)\right]
    +\mathcal{O}(a^4) } \]
  where $\theta = \left(\omega_0 - a^2 \frac{5\alpha^2}{12\omega_0} \right)t$.


  Now to see why we didn't need an $a\omega_1$ term.
  Assume that
  \begin{align*}
    x &= a\cos\theta + a^2 x_2(\theta) + \mathcal{O}(a^3);\quad \theta=\omega t \\
    \omega &= \omega_0 + a \omega_1 + \mathcal{O}(a^2).
  \end{align*}
  Substituting these expressions into the differential equation for $x$ yields
  \[ a^2 \left[ \omega_0^2(x_2''+x_2) - 2\omega_0 \omega_1 \cos\theta
    +\alpha\cos^2\theta \right] = \mathcal{O}(a^3) \]
  \[ x_2'' + x_2 = 2\frac{\omega_1}{\omega_0}\cos\theta 
  - \frac{\alpha}{2 \omega_0^2} (1+\cos 2\theta). \]
  In order to eliminate secular terms, we need $\omega_1=0$.
\end{Solution}





%%Consider the linearized traffic problem
\begin{Solution}
  \begin{enumerate}
    %%
    %%
    %%
  \item
    The equation for $p_1(t)$ is
    \begin{gather*}
      \frac{\dd p_1(t)}{\dd t} = \alpha [p_0(t) - p_1(t)].  \\
      \frac{\dd p_1(t)}{\dd t} = \alpha [a e^{\imath \omega t} - p_1(t)] \\
      \frac{\dd}{\dd t}\left(e^{\alpha t} p_1(t) \right) = \alpha a e^{\alpha t} 
      e^{\imath \omega t} \\
      p_1(t) = \frac{\alpha a}{\alpha + \imath \omega} e^{\imath \omega t} 
      + c e^{-\alpha t} \\
      \intertext{Applying the initial condition, $p_1(0) = 0$,}
      \boxed{p_1(t) = \frac{\alpha a}{\alpha + \imath \omega} \left(e^{\imath \omega t} 
          - e^{-\alpha t} \right) }
    \end{gather*}
    %%
    %%
    %%
  \item
    We start with the differential equation for $p_n(t)$.
    \[ \frac{\dd p_n(t)}{\dd t} = \alpha [p_{n-1}(t) - p_n(t)] \]
    Multiply by $s^n$ and sum from $n=1$ to $\infty$.
    \begin{gather*}
      \sum_{n = 1}^\infty p_n'(t) s^n = \sum_{n = 1}^\infty \alpha [p_{n-1}(t) - p_n(t)] s^n \\
      \frac{\partial G(s,t)}{\partial t} = \alpha \sum_{n = 0}^\infty p_n s^{n+1} 
      - \alpha G(s,t) \\
      \frac{\partial G(s,t)}{\partial t} = \alpha s p_0 + \alpha \sum_{n = 1}^\infty p_n s^{n+1} 
      - \alpha G(s,t) \\
      \frac{\partial G(s,t)}{\partial t} = \alpha a s e^{\imath \omega t} 
      + \alpha s G(s,t) - \alpha G(s,t) \\
      \frac{\partial G(s,t)}{\partial t} = \alpha a s e^{\imath \omega t} 
      + \alpha (s-1) G(s,t)  \\
      \frac{\partial}{\partial t} \left( e^{\alpha(1-s)t} G(s,t) \right) = \alpha a s 
      e^{\alpha(1-s)t} e^{\imath \omega t} \\
      G(s,t) = \frac{\alpha a s}{\alpha(1-s) + \imath \omega} e^{\imath \omega t}
      + C(s) e^{\alpha(s-1) t} 
    \end{gather*}
    The initial condition is
    \[ G(s,0) = \sum_{n = 1}^\infty p_n(0) s^n = 0. \]
    The generating function is then
    \[ \boxed{G(s,t)=\frac{\alpha a s}{\alpha(1-s) + \imath \omega} 
      \left(\alpha e^{\imath \omega t} - e^{\alpha(s-1) t} \right).} \]
    %%
    %%
    %%
  \item
    Assume that $|s| < 1$. In the limit $t \to \infty$ we have
    \begin{gather*}
      G(s,t) \sim \frac{\alpha a s}{\alpha(1-s) + \imath \omega} e^{\imath \omega t} \\
      G(s,t) \sim \frac{a s}{1 + \imath \omega/\alpha - s} e^{\imath \omega t} \\
      G(s,t) \sim \frac{a s/(1+ \imath \omega/\alpha)}{1 - s/(1 + \imath \omega/\alpha)} 
      e^{\imath \omega t} \\
      G(s,t) \sim \frac{a s e^{\imath \omega t}}{1+ \imath \omega/\alpha} 
      \sum_{n = 0}^\infty \left(\frac{s} {1+\imath \omega/\alpha} \right)^n \\
      G(s,t) \sim a e^{\imath \omega t} \sum_{n = 1}^\infty \frac{s^n}
      {(1 + \imath \omega/\alpha)^n}  
    \end{gather*}
    Thus we have 
    \[ 
    \boxed{p_n(t) \sim \frac{a}{(1 + \imath \omega/\alpha)^n} e^{\imath \omega t} \quad
      \mathrm{as}\ t \to \infty.} 
    \]


    \begin{align*}
      \Im(p_n(t))
      &\sim \Im \left[\frac{a}{(1 + \imath \omega/\alpha)^n} e^{\imath \omega t}
      \right] \\
      &= a \left(\frac{1 - \imath \omega/\alpha}{1 + (\omega/\alpha)^2} 
      \right)^n [ \cos(\omega t) + \imath \sin(\omega t) ] \\
      &= \frac{a}{(1+(\omega/\alpha)^2)^n} \left[ \cos(\omega t)
        \Im[(1 - \imath \omega/\alpha)^n] + \sin(\omega t)
        \Re[(1 - \imath \omega/\alpha)^n] \right] \\
      &= \frac{a}{(1+(\omega/\alpha)^2)^n} \left[ \cos(\omega t)
        \sum_{\substack{j = 1 \\ \mathrm{odd}\ j}}^n
        (-1)^{(j+1)/2} \left(\frac{\omega}{\alpha}\right)^j 
        + \sin(\omega t) 
        \sum_{ \substack{j = 0 \\ \mathrm{even}\ j}}^n
        (-1)^{j/2} \left(\frac{\omega}{\alpha}\right)^j \right] 
    \end{align*}
  \end{enumerate}
\end{Solution}



%%\frac{\dd}{\dd t} p_n(t+\tau) = \alpha [p_{n-1}(t) - p_n(t) ] \quad n\geq 1,
\begin{Solution}
  \begin{enumerate}
    %%
    %%
    %%
  \item
    Substituting $p_n = A_n e^{\imath \omega t}$ into the differential equation
    yields
    \begin{gather*}
      A_n \imath \omega e^{\imath \omega(t+\tau)} = \alpha[ A_{n-1} e^{\imath \omega t} - 
      A_n e^{\imath \omega t} ] \\
      A_n (\alpha + \imath \omega e^{\imath \omega \tau}) = \alpha A_{n-1} \\
      \intertext{We make the substitution $A_n = r^n$.}
      r^n (\alpha + \imath \omega e^{\imath \omega \tau}) = \alpha r^{n-1} \\
      r = \frac{\alpha}{\alpha + \imath \omega e^{\imath \omega \tau}}
    \end{gather*}
    Thus we have
    \[ \boxed{p_n(t)=\left(\frac{1}{1+\imath \omega e^{\imath \omega \tau} / \alpha}\right)^n
      e^{\imath \omega t}. } \]
    Taking the imaginary part,
    \begin{align*}
      \Im(p_n(t))
      &= \Im\left[ \left(\frac{1}{1+\imath \frac{\omega }{\alpha}
            e^{\imath \omega \tau}}\right)^n e^{\imath \omega t} \right] \\
      &= \Im\left[ \left( 
          \frac{1-\imath \frac{\omega}{\alpha} e^{-\imath \omega \tau}}
          {1 + \imath \frac{\omega}{\alpha}(e^{\imath \omega \tau}
            - e^{-\imath \omega \tau}) + (\frac{\omega}{\alpha})^2} \right)^n
        \big( \cos(\omega t) + \imath \sin(\omega t) \big) \right] \\
      &= \Im\left[ \left( 
          \frac{1 - \frac{\omega}{\alpha} \sin(\omega \tau)
            - \imath \frac{\omega}{\alpha} \cos(\omega \tau) }
          {1 - 2 \frac{\omega}{\alpha} \sin(\omega \tau) 
            +(\frac{\omega}{\alpha})^2} \right)^n
        \big( \cos(\omega t) + \imath \sin(\omega t) \big) \right] \\
      &= \left(\frac{1}{1 - 2 \frac{\omega}{\alpha} \sin(\omega \tau)
          +(\frac{\omega}{\alpha})^2} \right)^n
      \Big[ \cos(\omega t) \Im\left[\left(1 
          - \frac{\omega}{\alpha} \sin(\omega \tau)
          - \imath \frac{\omega}{\alpha} \cos(\omega \tau)\right)^n \right] \\
      &\qquad + \sin(\omega t) \Re\left[ \left(1 
          - \frac{\omega}{\alpha} \sin(\omega \tau)
          - \imath \frac{\omega}{\alpha} \cos(\omega \tau)\right)^n \right] 
      \Big] \\
      &= \left(\frac{1}{1 - 2 \frac{\omega}{\alpha} \sin(\omega \tau)
          +(\frac{\omega}{\alpha})^2} \right)^n \\
      &\qquad \Big[ \cos(\omega t) 
      \sum_{\substack{j=1 \\ \mathrm{odd}\ j}}^n
      (-1)^{(j+1)/2} \left[\frac{\omega}{\alpha} \cos(\omega \tau)
      \right]^j \left[1-\frac{\omega}{\alpha}\sin(\omega\tau)
      \right]^{n-j} \\
      &\qquad + \sin(\omega t) 
      \sum_{\substack{j=0 \\ \mathrm{even}\ j}}^n
      (-1)^{j/2} \left[\frac{\omega}{\alpha} \cos(\omega \tau)
      \right]^j \left[1-\frac{\omega}{\alpha}\sin(\omega\tau)
      \right]^{n-j} \Big] 
    \end{align*}
    %%
    %%
    %%
  \item
    $p_n(t)$ will remain bounded in time as $n \to \infty$ if 
    \begin{gather*}
      \left| \frac{1}{1 + \imath \frac{\omega}{\alpha} e^{\imath \omega \tau}} \right| 
      \leq 1 \\
      \left| 1 + \imath \frac{\omega}{\alpha} e^{\imath \omega \tau} \right|^2 \geq 1 \\
      1 - 2 \frac{\omega}{\alpha} \sin(\omega \tau) + \left(\frac{\omega}{\alpha}
      \right)^2 \geq 1 \\
      \boxed{ \frac{\omega}{\alpha} \geq 2 \sin(\omega \tau) }
    \end{gather*}
    %%
    %%
    %%
  \item
    %% CONTINUE
  \end{enumerate}
\end{Solution}




\raggedbottom







