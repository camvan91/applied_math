\documentclass{article}

\usepackage{color}
\usepackage[pdftex,
	    colorlinks=true]
           {hyperref}
%\usepackage
%        [pdftex,
%        colorlinks=true,
%        hyperindex=true,
%        plainpages=true,
%        bookmarks=true,
%%        pdfpagescrop={60 552 60 444}
%%       linkcolor=webgreen,
%        pdftitle={Introduction to Methods of Applied Mathematics},
%        pdfauthor={Sean Mauch},
%        pdfsubject={calculus,functions of a complex
%        variable,differential equations},
%        pdfpagemode=UseOutlines]{hyperref}

\title{Using the Source of ``Introduction to Methods of Applied Mathematics''}
\author{Sean Mauch}

\begin{document}
\maketitle



\section{Open Source Text}

My applied math text is \textit{open source}.  All of the source used in the 
creation of the text is available for download.  The 
\href{http://www.latex-project.org/}{\LaTeX} 
source which comprises the text, the 
\href{http://www.wolfram.com/}{Mathematica}
notebooks and 
\href{http://www.xfig.org/}{xfig}
files used to make the graphics, etc. are all available. I hope this will 
save instructors time, as they will be able to cut and paste portions 
of the text to make problem sets, handouts and slides.



\section{Downloading}

You can obtain the source (as well as the compiled 
\href{http://www.adobe.com/products/acrobat/}{PDF}
and 
\href{http://www.cs.wisc.edu/~ghost/}{postscript}
files) from my homepage: 
\href{http://www.its.caltech.edu/~sean}
{http://www.its.caltech.edu/{\~{}}sean}.
Follow the \textit{Publications} link, then the 
\textit{Applied Math Textbook} link and finally the 
\textit{Unabridged version} link.  Save either the zip archive or the gzip 
archive to your computer.  On linux systems, you can unpack the zip archive 
with: 
\begin{verbatim}
unzip applied_math_source.zip
\end{verbatim}
or unpack the gzip archive with:
\begin{verbatim}
gunzip applied_math_source.tar.gz
tar xvf applied_math_source.tar
\end{verbatim}
This will create the directory \verb+applied_math+.



\section{Operating Systems}

One can install 
\href{http://www.tug.org/}{\TeX} and \LaTeX\ on almost any operating system.
However, I only have experience with using \LaTeX\ on Linux systems and
some of the advise in this article is specific to Linux (or Unix) systems.  



\section{Layout}

Each part of the text has a top-level directory.
\begin{description}
\item{\verb+preface+} Preface
\item{\verb+algebra+} Algebra
\item{\verb+calculus+} Calculus
\item{\verb+fcv+} Functions of a Complex Variable
\item{\verb+ode+} Ordinary Differential Equations
\item{\verb+pde+} Partial Differential Equations
\item{\verb+cv+} Calculus of Variations
\item{\verb+nde+} Nonlinear Differential Equations
\item{\verb+appendix+} Appendices
\end{description}
These directories have subdirectories for each chapter.  Each chapter 
directory has a file with a \verb+.tex+ extension which contains the 
\LaTeX\ source for that chapter.  \LaTeX\ files in the root directory
include the sources in the chapters to make the various versions of 
the text.  Each chapter directory also contains the graphics for that
chapter.  Each graphic is supplied in PDF format (with a \verb+.pdf+
extension) and in encapsulated postscript format%
\footnote{You can convert encapsulated postscript files to PDF format
with
\href{http://www.ctan.org/tex-archive/support/epstopdf/}{epstopdf}.
}
(with a \verb+.eps+ extension).  This allows one to compile the source 
into either PDF or DVI formats.  (The DVI format can be converted to 
postscript with the 
\href{http://www.radicaleye.com/dvips.html}{dvips}
program.)  The files used to make the graphics are included as well.  Most
of the graphics are made with Mathematica (\verb+.nb+) or xfig (\verb+.fig+).

In the root directory \verb+header.tex+ contains includes, macros and 
definitions that are used throughout the source.  
(I have tried to use few custom definitions in order
to make cutting and pasting easier.)
Have a look at this file if you cut and past the source to make new documents.
You will probably want to include this file (or a modified version of it)
in your source.




\section{Compiling}

In the top level directory there is the make file \verb+Makefile+.  Executing
\begin{verbatim}
make
\end{verbatim}
will build the postscript, the PDF letter and the PDF screen versions of the 
text.  If you only want to build one of these choose 
from the following commands:
\begin{verbatim}
make ps
make pdf
make online
make users_guide
\end{verbatim}
Use \verb+make users_guide+ to make this document.



\section{Making Handouts}

If you cut and paste the source to make handouts, you will probably have to
include a few packages in the header of your document.  In particular, you
will need to put
\begin{verbatim}
\usepackage{amsmath}
\usepackage{amssymb}
\end{verbatim}
in your header.  If you use graphics, you will need either
\begin{verbatim}
\usepackage[pdftex]{graphicx}
\end{verbatim}
for PDF output from \verb+pdflatex+ or
\begin{verbatim}
\usepackage{graphicx}
\end{verbatim}
for DVI output from \verb+latex+.

In the root directory \verb+header.tex+ contains includes, macros and 
definitions that are used throughout the source. You may want to 
include this file instead of including individual packages.  Check out
the file \verb+applied_math.tex+ (which makes the PDF screen version of 
the text) to see an example of how to do this.



\section{Making Slides}

You can make slides using \LaTeX.  See the file 
\verb+fcv/number/number_slides.tex+ for an example.  Go to the 
\verb+fcv/number/+ directory and compile the slides with 
\begin{verbatim}
pdflatex number_slides.tex
\end{verbatim}
to make a PDF file.
To make a postscript file, change the \verb+pdf+ variable to 
\verb+false+ in the source and then execute 
\begin{verbatim}
latex number_slides.tex
\end{verbatim}

\end{document}
