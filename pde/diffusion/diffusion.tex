\flushbottom




%%============================================================================
%%============================================================================
\chapter{The Diffusion Equation}









\raggedbottom
\exercises{
%%=============================================================================
\pagebreak
\flushbottom
\section{Exercises}






\begin{Exercise}
  Is the solution of the Cauchy problem for the heat equation unique?
  \begin{gather*}
    u_t - \kappa u_{x x} = q(x,t), \quad -\infty < x < \infty, \quad t > 0
    \\
    u(x,0) = f(x)
  \end{gather*}
\end{Exercise}






\begin{Exercise}
  Consider the heat equation with a time-independent source term and 
  inhomogeneous boundary conditions.
  \begin{gather*}
    u_t = \kappa u_{x x} + q(x)
    \\
    u(0,t) = a, \quad u(h,t) = b, \quad u(x,0) = f(x)
  \end{gather*}
\end{Exercise}







\begin{Exercise}
  Is the Cauchy problem for the backward heat equation
  \begin{equation}
    \label{eqn ut+kuxx=0 ux0=f}
    u_t + \kappa u_{x x} = 0, \quad u(x,0) = f(x)
  \end{equation}
  well posed?
\end{Exercise}








%% heat equation in 3D
\begin{Exercise}
  Derive the heat equation for a general
  3 dimensional body, with non-uniform density $\rho(\mathbf{x})$, specific
  heat $c(\mathbf{x})$, and conductivity $k(\mathbf{x})$. Show that
  \[  
  \frac{ \partial u(\mathbf{x},t)}{\partial t} = \frac{1}{c \rho} \nabla \cdot (k \nabla u(\mathbf{x},t)) 
  \]
  where $u$ is the temperature, and you may assume there are no internal 
  sources or sinks. 
\end{Exercise}







%% Verify Duhamel's Principal
\begin{Exercise}
  Verify Duhamel's Principal:  If $u(x,t,\tau)$ is the solution of the initial
  value problem:
  \[
  u_t = \kappa u_{x x}, \quad u(x,0,\tau) = f(x,\tau),
  \]
  then the solution of
  \[
  w_t = \kappa w_{x x} + f(x,t), \quad w(x,0) = 0
  \]
  is
  \[
  w(x,t) = \int_0^t u(x,t-\tau,\tau) \,\dd \tau.
  \]
\end{Exercise}









%% Modify the derivation of the diffusion equation
\begin{Exercise}
  Modify the derivation of the diffusion equation
  \begin{equation}
    \label{eqn_diffusion}
    \phi_t = a^2 \phi_{x x}, \quad a^2 = \frac{k}{c \rho},
  \end{equation}
  so that it is valid for diffusion in a 
  non-homogeneous medium for which $c$ and $k$ are functions of $x$ and $\phi$
  and so that it is valid for a geometry in which $A$ is a function of $x$.
  Show that Equation~(\ref{eqn_diffusion}) above is in this case replaced
  by
  \[
  c \rho A \phi_t = \left( k A \phi_x \right)_x.
  \]
  Recall that $c$ is the specific heat, $k$ is the thermal conductivity,
  $\rho$ is the density, $\phi$ is the temperature and $A$ is the 
  cross-sectional area.
\end{Exercise}







\raggedbottom
}
%%=============================================================================
\hints{
\pagebreak
\flushbottom
\section{Hints}






\begin{Hint}
  %% CONTINUE
\end{Hint}






\begin{Hint}
  %% CONTINUE
\end{Hint}





\begin{Hint}
  %% CONTINUE
\end{Hint}





%% heat equation in 3D
\begin{Hint}
  %% CONTINUE
\end{Hint}




%% Verify Duhamel's Principal
\begin{Hint}
  Check that the expression for $w(x,t)$ satisfies the partial differential 
  equation and initial condition.  Recall that
  \[
  \frac{\partial}{\partial x} \int_a^x h(x,\xi) \,\dd \xi 
  = \int_a^x h_x(x,\xi) \,\dd \xi + h(x,x).
  \]
\end{Hint}








%% Modify the derivation of the diffusion equation
\begin{Hint}
  %% CONTINUE
\end{Hint}








\raggedbottom
}
%%=============================================================================
\solutions{
\pagebreak
\flushbottom
\section{Solutions}









\begin{Solution}
  Let $u$ and $v$ both be solutions of the Cauchy problem for the heat equation.
  Let $w$ be the difference of these solutions.  $w$ satisfies the problem
  \begin{gather*}
    w_t - \kappa w_{x x} = 0, \quad -\infty < x < \infty, \quad t > 0,
    \\
    w(x,0) = 0.
  \end{gather*}
  We can solve this problem with the Fourier transform.
  \begin{gather*}
    \hat{w}_t + \kappa \omega^2 \hat{w} = 0, \quad \hat{w}(\omega,0) = 0
    \\
    \hat{w} = 0
    \\
    w = 0
  \end{gather*}
  Since $u - v = 0$, we conclude that the solution of the Cauchy problem for 
  the heat equation is unique.
\end{Solution}









\begin{Solution}
  Let $\mu(x)$ be the equilibrium temperature.  It satisfies an ordinary
  differential equation boundary value problem.
  \[
  \mu'' = - \frac{q(x)}{\kappa}, \quad \mu(0) = a, \quad \mu(h) = b
  \]
  To solve this boundary value problem we find a particular solution
  $\mu_p$ that satisfies homogeneous boundary conditions
  and then add on a homogeneous solution $\mu_h$ 
  that satisfies the inhomogeneous boundary conditions.
  \begin{gather*}
    \mu_p'' = - \frac{q(x)}{\kappa}, \quad \mu_p(0) = \mu_p(h) = 0
    \\
    \mu_h'' = 0, \quad \mu_h(0) = a, \quad \mu_h(h) = b
  \end{gather*}

  We find the particular solution $\mu_p$ with the method of Green functions.
  \[
  G'' = \delta(x-\xi), \quad G(0|\xi) = G(h|\xi) = 0.
  \]
  We find homogeneous solutions which respectively satisfy the left and right 
  homogeneous boundary conditions.
  \[
  y_1 = x, \quad y_2 = h - x
  \]
  Then we compute the Wronskian of these solutions and write down the 
  Green function.
  \begin{gather*}
    W = 
    \begin{vmatrix}
      x & h - x \\
      1 & -1
    \end{vmatrix}
    = -h
    \\
    G(x|\xi) = - \frac{1}{h} x_< \left( h - x_> \right)
  \end{gather*}

  The homogeneous solution that satisfies the inhomogeneous boundary
  conditions is
  \[
  \mu_h = a + \frac{b - a}{h} x
  \]

  Now we have the equilibrium temperature.
  \begin{gather*}
    \mu = a + \frac{b - a}{h} x
    + \int_0^h - \frac{1}{h} x_< \left( h - x_> \right) \left( - \frac{q(\xi)}{\kappa} \right)
    \,\dd \xi
    \\
    \mu = a + \frac{b - a}{h} x
    + \frac{h-x}{h \kappa} \int_0^x \xi q(\xi) \,\dd \xi
    + \frac{x}{h \kappa} \int_x^h (h-\xi) q(\xi) \,\dd \xi
  \end{gather*}


  Let $v$ denote the deviation from the equilibrium temperature.
  \[
  u = \mu + v
  \]
  $v$ satisfies a heat equation with homogeneous boundary conditions and no 
  source term.
  \[
  v_t = \kappa v_{x x}, \quad
  v(0,t) = v(h,t) = 0, \quad
  v(x,0) = f(x) - \mu(x)
  \]
  We solve the problem for $v$ with separation of variables.
  \begin{gather*}
    v = X(x) T(t)
    \\
    X T' = \kappa X'' T
    \\
    \frac{T'}{\kappa T} = \frac{X''}{X} = - \lambda
  \end{gather*}
  We have a regular Sturm-Liouville problem for $X$ and a differential equation
  for $T$.
  \begin{gather*}
    X'' + \lambda X = 0, \quad X(0) = X(\lambda) = 0
    \\
    \lambda_n = \left( \frac{n \pi}{h} \right)^2, \quad
    X_n = \sin \left( \frac{n \pi x}{h} \right), \quad
    n \in \mathbb{Z}^+
    \\
    T' = - \lambda \kappa T
    \\
    T_n = \exp \left( - \kappa \left( \frac{n \pi}{h} \right)^2 t \right)
  \end{gather*}
  $v$ is a linear combination of the eigensolutions.
  \[
  v = \sum_{n = 1}^\infty v_n \sin \left( \frac{n \pi x}{h} \right)
  \exp \left( - \kappa \left( \frac{n \pi}{h} \right)^2 t \right)
  \]
  The coefficients are determined from the initial condition, 
  $v(x,0) = f(x) - \mu(x)$.
  \[
  v_n = \frac{2}{h} \int_0^h (f(x) - \mu(x)) \sin \left( \frac{n \pi x}{h} \right) \,\dd x
  \]

  We have determined the solution of the original problem in terms of the
  equilibrium temperature and the deviation from the equilibrium.
  $u = \mu + v$.
\end{Solution}






\begin{Solution}
  A problem is well posed if there exists a unique solution that depends 
  continiously on the nonhomogeneous data.

  First we find some solutions of the differential equation with the separation
  of variables $u = X(x) T(t)$.
  \begin{gather*}
    u_t + \kappa u_{x x} = 0, \quad \kappa > 0
    \\
    X T' + \kappa X'' T = 0
    \\
    \frac{T'}{\kappa T} = - \frac{X''}{X} = \lambda
    \\
    X'' + \lambda X = 0, \quad T' = \lambda \kappa T
    \\
    u = \cos\left( \sqrt{\lambda} x \right) \e^{\lambda \kappa t}, \quad
    u = \sin\left( \sqrt{\lambda} x \right) \e^{\lambda \kappa t}
  \end{gather*}
  Note that 
  \[
  u = \epsilon \cos\left( \sqrt{\lambda} x \right) \e^{\lambda \kappa t}
  \]
  satisfies the Cauchy problem
  \[
  u_t + \kappa u_{x x} = 0, \quad u(x,0) = \epsilon \cos\left( \sqrt{\lambda} x \right)
  \]
  Consider $\epsilon \ll 1$.  The initial condition is small, it satisfies 
  $|u(x,0)| < \epsilon$.  However the solution for any positive time can be made 
  arbitrarily large by choosing a sufficiently large, positive value of $\lambda$.
  We can make the solution exceed the value $M$ at time $t$ by choosing 
  a value of $\lambda$ such that
  \begin{gather*}
    \epsilon \e^{\lambda \kappa t} > M
    \\
    \lambda > \frac{1}{\kappa t} \ln \left( \frac{M}{\epsilon} \right).
  \end{gather*}
  Thus we see that Equation~\ref{eqn ut+kuxx=0 ux0=f} is ill posed because 
  the solution does not depend continuously on the initial data.  A small 
  change in the initial condition can produce an arbitrarily large change 
  in the solution for any fixed time.
\end{Solution}









%% heat equation in 3D
\begin{Solution}
  Consider a Region of material, $R$.  Let $u$ be the temperature and 
  $\boldsymbol{\phi}$ be the heat flux.  The amount of heat energy in the region
  is
  \[
  \int_R c \rho u \,\dd \mathbf{x}.
  \]
  We equate the rate of change of heat energy in the region with the 
  heat flux across the boundary of the region.
  \[
  \frac{\dd}{\dd t} \int_R c \rho u \,\dd \mathbf{x}
  = - \int_{\partial R} \boldsymbol{\phi} \cdot \mathbf{n} \,\dd s
  \]
  We apply the divergence theorem to change the surface integral to a volume 
  integral.
  \begin{gather*}
    \frac{\dd}{\dd t} \int_R c \rho u \,\dd \mathbf{x}
    = - \int_R \nabla \cdot \boldsymbol{\phi} \,\dd \mathbf{x}
    \\
    \int_R \left( c \rho \frac{\partial u}{\partial t} + \nabla \cdot \boldsymbol{\phi} \right) \,\dd \mathbf{x}
    = 0
  \end{gather*}
  Since the region is arbitrary, the integral must vanish identically.
  \[
  c \rho \frac{\partial u}{\partial t} = - \nabla \cdot \boldsymbol{\phi}
  \]
  We apply Fourier's law of heat conduction, $\boldsymbol{\phi} = - k \nabla u$,
  to obtain the heat equation.
  \[
  \frac{\partial u}{\partial t} = \frac{1}{c \rho} \nabla \cdot (k \nabla u)
  \]
\end{Solution}









%% Verify Duhamel's Principal
\begin{Solution}
  We verify Duhamel's principal by showing that the integral expression for 
  $w(x,t)$ satisfies the partial differential equation and the initial 
  condition.  Clearly the initial condition is satisfied.
  \[
  w(x,0) = \int_0^0 u(x,0-\tau, \tau) \,\dd \tau = 0
  \]
  Now we substitute the expression for $w(x,t)$ into the partial differential
  equation.
  \begin{gather*}
    \frac{\partial}{\partial t} \int_0^t u(x,t-\tau,\tau) \,\dd \tau 
    = \kappa \frac{\partial^2}{\partial x^2} \int_0^t u(x,t-\tau,\tau) \,\dd \tau + f(x,t) \\
    u(x,t-t,t) + \int_0^t u_t(x,t-\tau,\tau)\,\dd \tau
    = \kappa \int_0^t u_{x x}(x,t-\tau,\tau)\,\dd \tau + f(x,t) \\
    f(x,t) + \int_0^t u_t(x,t-\tau,\tau)\,\dd \tau
    = \kappa \int_0^t u_{x x}(x,t-\tau,\tau)\,\dd \tau + f(x,t) \\
    \int_0^t \left( u_t(x,t-\tau,\tau)\,\dd \tau - \kappa u_{x x}(x,t-\tau,\tau)
    \right) \,\dd \tau 
  \end{gather*}
  Since $u_t(x,t-\tau,\tau)\,\dd \tau - \kappa u_{x x}(x,t-\tau,\tau) = 0$, this
  equation is an identity.  
\end{Solution}







%% Modify the derivation of the diffusion equation
\begin{Solution}
  We equate the rate of change of thermal energy in the segment 
  $(\alpha \ldots \beta)$ with the heat entering the segment through
  the endpoints.
  \begin{gather*}
    \int_\alpha^\beta \phi_t c \rho A \,\dd x 
    = k(\beta, \phi(\beta)) A(\beta) \phi_x(\beta, t)
    - k(\alpha, \phi(\alpha)) A(\alpha) \phi_x(\alpha, t) \\
    \int_\alpha^\beta \phi_t c \rho A \,\dd x 
    = \left[ k A \phi_x \right]_\alpha^\beta \\
    \int_\alpha^\beta \phi_t c \rho A \,\dd x 
    = \int_\alpha^\beta \left( k A \phi_x \right)_x \,\dd x \\
    \int_\alpha^\beta c \rho A \phi_t - \left( k A \phi_x \right)_x \,\dd x = 0 \\
    \intertext{Since the domain is arbitrary, we conclude that}
    \boxed{
      c \rho A \phi_t = \left( k A \phi_x \right)_x.
      }
  \end{gather*}
\end{Solution}









\raggedbottom
}
