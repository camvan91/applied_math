\flushbottom




%%============================================================================
%%============================================================================
\chapter{Finite Transforms}





\begin{Example}
  Consider the problem
  \[
  \Delta u - \frac{1}{c^2} \frac{\partial^2 u}{\partial t^2} 
  = \delta(x-\xi)\delta(y-\eta) \e^{-\imath \omega t} \quad \mathrm{on}\  
  -\infty < x < \infty,\ 0 < y < b,
  \]
  with
  \[
  u_y(x,0,t) = u_y(x,b,t) = 0.
  \]
  Substituting $u(x,y,t) = v(x,y) \e^{-\imath \omega t}$ into the partial differential 
  equation yields the problem
  \[
  \Delta v + k^2 v 
  = \delta(x-\xi)\delta(y-\eta) \quad \mathrm{on}\  
  -\infty < x < \infty,\ 0 < y < b,
  \]
  with
  \[
  v_y(x,0) = v_y(x,b) = 0.
  \]
  We assume that the solution has the form
  \begin{equation}
    \label{fdevcn}
    v(x,y) = \frac{1}{2} c_0(x) + \sum_{n = 1}^\infty c_n(x) \cos\left(\frac{n \pi y}{b} 
    \right),
  \end{equation}
  and apply a finite cosine transform in the $y$ direction.
  Integrating from $0$ to $b$ yields
  \[
  \int_0^b v_{x x} + v_{y y} + k^2 v \,d y 
  = \int_0^b \delta(x-\xi) \delta(y-\eta)\,d y,
  \]
  \[
  \big[ v_y \big]_0^b + \int_0^b v_{x x} + k^2 v \,d y = \delta(x-\xi),
  \]
  \[
  \int_0^b v_{x x} + k^2 v \,d y = \delta(x-\xi).
  \]
  Substituting in Equation~\ref{fdevcn} and using the orthogonality of the
  cosines gives us
  \[
  \boxed{
    c_0''(x) + k^2 c_0(x) = \frac{2}{b} \delta(x-\xi).
    }
  \]
  Multiplying by $\cos(n \pi y/b)$ and integrating form $0$ to $b$ yields
  \[
  \int_0^b \left( v_{x x} + v_{y y} + k^2 v \right) 
  \cos\left(\frac{n\pi y}{b} \right)\,d y 
  = \int_0^b \delta(x-\xi) \delta(y-\eta) \cos\left(\frac{n\pi y}{b}\right)\,d y.
  \]
  The $v_{y y}$ term becomes
  \begin{align*}
    \int_0^b v_{y y} \cos\left(\frac{n\pi y}{b} \right)\,d y 
    &= \left[ v_y \cos\left(\frac{n\pi y}{b} \right) \right]_0^b -
    \int_0^b - \frac{n\pi}{b} v_y \sin\left(\frac{n\pi y}{b} \right) \,d y \\
    &= \left[ \frac{n\pi}{b} v \sin\left(\frac{n\pi y}{b} \right) 
    \right]_0^b
    - \int_0^b  \left(\frac{n\pi}{b}\right)^2 v \cos\left(\frac{n\pi y}{b} 
    \right) \,d y.
  \end{align*}
  The right-hand-side becomes
  \[
  \int_0^b \delta(x-\xi) \delta(y-\eta) \cos\left(\frac{n\pi y}{b}\right)\,d y
  = \delta(x-\xi) \cos\left(\frac{n\pi \eta}{b}\right).
  \]
  Thus the partial differential equation becomes
  \[
  \int_0^b \left( v_{x x} - \left(\frac{n\pi}{b}\right)^2 v + k^2 v \right) 
  \cos\left(\frac{n\pi y}{b} \right)\,d y 
  = \delta(x-\xi) \cos\left(\frac{n\pi \eta}{b}\right).
  \]
  Substituting in Equation~\ref{fdevcn} and using the orthogonality of the
  cosines gives us
  \[
  \boxed{
    c_n''(x) + \left[ k^2 - \left(\frac{n\pi}{b}\right)^2 \right] c_n(x)
    = \frac{2}{b} \delta(x-\xi) \cos\left(\frac{n\pi \eta}{b}\right).
    }
  \]

  Now we need to solve for the coefficients in the expansion of $v(x,y)$.
  The homogeneous solutions for $c_0(x)$ are $\e^{\pm \imath k x}$.  The solution for 
  $u(x,y,t)$ must satisfy the radiation condition.  The waves at $x=-\infty$
  travel to the left and the waves at $x=+\infty$ travel to the right.  The
  two solutions of that will satisfy these conditions are, respectively,
  \[
  y_1 = \e^{-\imath k x}, \qquad y_2 = \e^{\imath k x}.
  \]
  The Wronskian of these two solutions is $\imath 2 k$.  Thus the solution for 
  $c_0(x)$ is
  \[
  \boxed{
    c_0(x) = \frac{\e^{- \imath k x_<} \e^{\imath k x_>}}{\imath b k}
    }
  \]

  We need to consider three cases for the equation for $c_n$.
  \begin{description}
  \item{$\mathbf{k \boldsymbol{>} n \boldsymbol{\pi} / b}$}
    Let $\alpha = \sqrt{k^2-(n \pi/b)^2}$.
    The homogeneous solutions that satisfy the radiation condition are
    \[
    y_1 = \e^{-\imath \alpha x}, \qquad y_2 = \e^{\imath \alpha x}.
    \]
    The Wronskian of the two solutions is $\imath 2 \alpha$.  Thus the solution is
    \[
    \boxed{
      c_n(x) = \frac{\e^{- \imath \alpha x_<} \e^{\imath \alpha x_>}}{\imath b \alpha} 
      \cos\left(\frac{n\pi \eta}{b}\right).
      }
    \]
    In the case that $\cos\left(\frac{n\pi \eta}{b}\right) = 0$ this reduces to 
    the trivial solution.
  \item{$\mathbf{k \boldsymbol{=} n \boldsymbol{\pi} / b}$}
    The homogeneous solutions that are bounded at infinity are
    \[
    y_1 = 1, \qquad y_2 = 1.
    \]
    If the right-hand-side is nonzero there is no way to combine these solutions
    to satisfy both the continuity and the derivative jump conditions.
    Thus if $\cos\left(\frac{n\pi \eta}{b}\right) \neq 0$ there is no bounded 
    solution.  If $\cos\left(\frac{n\pi \eta}{b}\right) = 0$ then the solution 
    is not unique.
    \[
    c_n(x) = \mathrm{const}.
    \]
  \item{$\mathbf{k \boldsymbol{<} n \boldsymbol{\pi} / b}$}
    Let $\beta = \sqrt{(n \pi / b)^2 - k^2}$.
    The homogeneous solutions that are bounded at infinity are
    \[
    y_1 = \e^{\beta x}, \qquad y_2 = \e^{-\beta x}.
    \]
    The Wronskian of these solutions is $-2 \beta$.  Thus the solution is
    \[
    \boxed{
      c_n(x) = -\frac{\e^{\beta x_<} \e^{-\beta x_>}}{b \beta} 
      \cos\left(\frac{n \pi \eta}{b} \right)
      }
    \]
    In the case that $\cos\left(\frac{n\pi \eta}{b}\right) = 0$ this reduces to 
    the trivial solution.
  \end{description}
\end{Example}










\raggedbottom
%%=============================================================================
\exercises{
\pagebreak
\flushbottom
\section{Exercises}






%% Slab, insulated on bottom fixed temperature on top
\begin{Exercise}
  A slab is perfectly insulated at the surface $x = 0$ and has a specified 
  time varying temperature $f(t)$ at the surface $x = L$.  Initially the
  temperature is zero.  Find the temperature $u(x,t)$ if the heat conductivity
  in the slab is $\kappa = 1$.  
\end{Exercise}





%% Laplace's equation on a semi-infinite strip.
\begin{Exercise}
  Solve
  \begin{gather*}
    u_{x x} + u_{y y} = 0, \quad 0 < x < L, \quad y > 0, \\
    u(x, 0) = f(x), \quad u(0, y) = g(y), \quad u(L, y) = h(y),
  \end{gather*}
  with an eigenfunction expansion.
\end{Exercise}












\raggedbottom
}
{
}
%%=============================================================================
\hints{
\pagebreak
\flushbottom
\section{Hints}




%% Slab, insulated on bottom fixed temperature on top
\begin{Hint}
  %% CONTINUE
\end{Hint}







%% Laplace's equation on a semi-infinite strip.
\begin{Hint}
  %% CONTINUE
\end{Hint}






\raggedbottom
}
%%=============================================================================
\solutions{
\pagebreak
\flushbottom
\section{Solutions}




%% Slab, insulated on bottom fixed temperature on top
\begin{Solution}
  The problem is
  \begin{gather*}
    u_t = u_{x x}, \quad 0 < x < L, t > 0, \\
    u_x(0,t) = 0, \quad u(L,t) = f(t), \quad u(x,0) = 0.
  \end{gather*}
  We will solve this problem with an eigenfunction expansion.  We find these
  eigenfunction by replacing the inhomogeneous boundary condition with the
  homogeneous one, $u(L,t) = 0$.
  We substitute the separation of variables $v(x,t) = X(x) T(t)$ into the
  homogeneous partial differential equation.
  \begin{gather*}
    X T' = X'' T \\
    \frac{T'}{T} = \frac{X''}{X} = - \lambda^2.
  \end{gather*}
  This gives us the regular Sturm-Liouville eigenvalue problem,
  \[
  X'' = - \lambda^2 X, \quad X'(0) = X(L) = 0,
  \]
  which has the solutions,
  \[
  \lambda_n = \frac{\pi (2n-1)}{2L}, \quad
  X_n = \cos \left( \frac{\pi (2n-1) x}{2L} \right), \quad
  n \in \mathbb{N}.
  \]
  Our solution for $u(x,t)$ will be an eigenfunction expansion in these
  eigenfunctions.  Since the inhomogeneous boundary condition is a function of
  $t$, the coefficients will be functions of $t$.
  \[
  \boxed{
    u(x,t) = \sum_{n = 1}^\infty a_n(t) \cos( \lambda_n x ) 
    }
  \]
  Since $u(x,t)$ does not satisfy the homogeneous boundary conditions 
  of the eigenfunctions, the series is not uniformly convergent and we 
  are not allowed to differentiate it with respect to $x$.
  We substitute the expansion into the partial differential equation, 
  multiply by the eigenfunction and integrate from $x = 0$ to $x = L$.
  We use integration by parts to move derivatives from $u$ to the 
  eigenfunctions.
  \begin{gather*}
    u_t = u_{x x} \\
    \int_0^L u_t \cos( \lambda_m x ) \,d x 
    = \int_0^L u_{x x} \cos( \lambda_m x ) \,d x \\
    \int_0^L \left( \sum_{n = 1}^\infty a_n'(t) \cos( \lambda_n x ) \right)
    \cos( \lambda_m x ) \,d x 
    = \left[ u_x \cos( \lambda_m x ) \right]_0^L
    + \int_0^L u_x \lambda_m \sin( \lambda_m x ) \,d x \\
    \frac{L}{2} a_m'(t) 
    = \left[ u \lambda_m \sin( \lambda_m x ) \right]_0^L
    - \int_0^L u \lambda_m^2 \cos( \lambda_m x ) \,d x \\
    \frac{L}{2} a_m'(t) 
    = \lambda_m u(L,t) \sin( \lambda_m L ) 
    - \lambda_m^2 \int_0^L 
    \left( \sum_{n = 1}^\infty a_n(t) \cos( \lambda_n x ) \right)
    \cos( \lambda_m x ) \,d x \\
    \frac{L}{2} a_m'(t) 
    = \lambda_m (-1)^n f(t) - \lambda_m^2 \frac{L}{2} a_m(t) \\
    a_m'(t) + \lambda_m^2 a_m(t) = (-1)^n \lambda_m f(t) 
  \end{gather*}
  From the initial condition $u(x,0) = 0$ we see that $a_m(0) = 0$.  Thus 
  we have a first order differential equation and an initial condition for
  each of the $a_m(t)$.
  \[
  a_m'(t) + \lambda_m^2 a_m(t) = (-1)^n \lambda_m f(t), \quad a_m(0) = 0
  \]
  This equation has the solution,
  \[
  \boxed{
    a_m(t) = (-1)^n \lambda_m \int_0^t \e^{-\lambda_m^2 (t - \tau)} f(\tau)\,d\tau.
    }
  \]
\end{Solution}








%% Laplace's equation on a semi-infinite strip.
\begin{Solution}
  \begin{gather*}
    u_{x x} + u_{y y} = 0, \quad 0 < x < L, \quad y > 0, \\
    u(x, 0) = f(x), \quad u(0, y) = g(y), \quad u(L, y) = h(y),
  \end{gather*}
  We seek a solution of the form,
  \[
  u(x, y) = \sum_{n = 1}^\infty u_n(y) \sin \left( \frac{n \pi x}{L} \right).
  \]
  Since we have inhomogeneous boundary conditions at $x = 0, L$, we cannot
  differentiate the series representation with respect to $x$.
  We multiply Laplace's equation by the eigenfunction and integrate from
  $x = 0$ to $x = L$.
  \begin{gather*}
    \int_0^L \left( u_{x x} + u_{y y} \right) \sin \left( \frac{m \pi x}{L} \right)
    \, d x = 0 \\
    \intertext{We use integration by parts to move derivatives from $u$ to 
      the eigenfunctions.}
    \left[ u_x \sin \left( \frac{m \pi x}{L} \right) \right]_0^L
    - \frac{m \pi}{L} \int_0^L u_x \cos \left( \frac{m \pi x}{L} \right)
    \,d x + \frac{L}{2} u_m''(y) = 0 \\
    \left[ - \frac{m \pi}{L} u \cos\left( \frac{m \pi x}{L} \right) \right]_0^L
    - \left( \frac{m \pi}{L} \right)^2 \int_0^L u 
    \sin \left( \frac{m \pi x}{L} \right) \,d x
    + \frac{L}{2} u_m''(y) = 0 \\
    - \frac{m \pi}{L} h(y) (-1)^m + \frac{m \pi}{L} g(y) 
    - \frac{L}{2} \left( \frac{m \pi}{L} \right)^2 u_m(y)
    + \frac{L}{2} u_m''(y) = 0 \\
    u_m''(y) - \left( \frac{m \pi}{L} \right)^2 u_m(y) = 2 m \pi
    \left( (-1)^m h(y) - g(y) \right)
  \end{gather*}
  Now we have an ordinary differential equation for the $u_n(y)$.  In order 
  that the solution is bounded, we require that each $u_n(y)$ is bounded as
  $y \to \infty$.  We use the boundary condition $u(x, 0) = f(x)$ to determine
  boundary conditions for the $u_m(y)$ at $y = 0$.
  \begin{gather*}
    u(x, 0) = \sum_{n = 1}^\infty u_n(0) \sin \left( \frac{n \pi x}{L} \right) = f(x) \\
    u_n(0) = f_n \equiv \frac{2}{L} \int_0^L f(x) 
    \sin \left( \frac{n \pi x}{L} \right) \,d x
  \end{gather*}
  Thus we have the problems,
  \[
  u_n''(y) - \left( \frac{n \pi}{L} \right)^2 u_n(y) = 2 n \pi
  \left( (-1)^n h(y) - g(y) \right), 
  \quad u_n(0) = f_n, \quad u_n(+\infty)\ \mathrm{bounded},
  \]
  for the coefficients in the expansion.  We will solve these with Green 
  functions.  Consider the associated Green function problem
  \[
  G_n''(y; \eta) - \left( \frac{n \pi}{L} \right)^2 G_n(y; \eta) = 
  \delta(y - \eta), \quad
  G_n(0; \eta) = 0, \quad G_n(+\infty; \eta)\ \mathrm{bounded}.
  \]
  The homogeneous solutions that satisfy the boundary conditions are
  \[
  \sinh \left( \frac{n \pi y}{L} \right) \quad \mathrm{and} \quad
  \e^{-n \pi y / L},
  \]
  respectively.  The Wronskian of these solutions is
  \[
  \begin{vmatrix}
    \sinh \left( \frac{n \pi y}{L} \right) & \e^{-n \pi y / L} \\
    \frac{n \pi}{L} \sinh \left( \frac{n \pi y}{L} \right) 
    & - \frac{n \pi}{L} \e^{-n \pi y / L} 
  \end{vmatrix}
  =
  - \frac{n \pi}{L} \e^{- 2 n \pi y / L}.
  \]
  Thus the Green function is
  \[
  G_n(y; \eta) = - \frac{ L \sinh \left( \frac{n \pi y_<}{L} \right)
    \e^{- n \pi y_> / L} }{ n \pi \e^{-2 n \pi \eta / L} }.
  \]
  Using the Green function we determine the $u_n(y)$ and thus the solution 
  of Laplace's equation.
  \begin{gather*}
    \boxed{
      u_n(y) = f_n \e^{- n \pi y / L} + 2 n \pi \int_0^\infty G_n(y; \eta)
      \left( (-1)^n h(\eta) - g(\eta) \right) \,d\eta
      } \\
    \boxed{
      u(x, y) = \sum_{n = 1}^\infty u_n(y) \sin \left( \frac{n \pi x}{L} \right).
      }
  \end{gather*}
\end{Solution}










\raggedbottom
}
