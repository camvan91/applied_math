\flushbottom




%%============================================================================
%%============================================================================
\chapter{Classification of Partial Differential Equations}


%%
%% Classify PDE's
%%
%% Derive some partial differential equations.
%%






%%===========================================================================
\section{Classification of Second Order Quasi-Linear Equations}





Consider the general second order quasi-linear partial differential 
equation in two variables.
\begin{equation}
  \label{a(x,y)u_xx+2b(x,y)u_xy+c(x,y)u_yy=F(x,y,u,u_x,u_y)}
  a(x,y) u_{x x} + 2 b(x,y) u_{x y} + c(x,y) u_{y y} = F(x, y, u, u_x, u_y)
\end{equation}
We classify the equation by the sign of the discriminant.
At a given point $x_0$, $y_0$, the equation is classified as one 
of the following types:
\[
\begin{matrix}
  b^2 - a c > 0: \quad &\mathrm{hyperbolic} 
  \\
  b^2 - a c = 0: \quad &\mathrm{parabolic} 
  \\
  b^2 - a c < 0: \quad &\mathrm{elliptic} 
\end{matrix}
\]
If an equation has a particular type for all points $x$, $y$ in a domain
then the equation is said to be of that type in the domain.
Each of these types has a canonical form that can be obtained through
a change of independent variables.  The type of an equation indicates
much about the nature of its solution.

We seek a change of independent variables, (a different coordinate 
system), such that 
Equation~\ref{a(x,y)u_xx+2b(x,y)u_xy+c(x,y)u_yy=F(x,y,u,u_x,u_y)}
has a simpler form.  We will find that a second order quasi-linear
partial differential equation in two variables can be transformed to
one of the canonical forms:
\begin{alignat*}{2}
  &u_{\xi \psi} = G(\xi, \psi, u, u_\xi, u_\psi), &\quad &\mathrm{hyperbolic} 
  \\
  &u_{\xi \xi} = G(\xi, \psi, u, u_\xi, u_\psi), &\quad &\mathrm{parabolic} 
  \\
  &u_{\xi \xi} + u_{\psi \psi} = G(\xi, \psi, u, u_\xi, u_\psi), &\quad &\mathrm{elliptic}
\end{alignat*}


Consider the change of independent variables 
\[
\xi = \xi(x,y), \qquad \psi = \psi(x,y).
\]
We calculate the partial derivatives of $u$.
\begin{align*}
  u_x &= \xi_x u_\xi + \psi_x u_\psi 
  \\
  u_y &= \xi_y u_\xi + \psi_y u_\psi 
  \\
  u_{xx} &= \xi_x^2 u_{\xi \xi} + 2 \xi_x \psi_x u_{\xi\psi} + \psi_x^2 u_{\psi\psi} + \xi_{xx} u_\xi + \psi_{xx} u_{\psi} 
  \\
  u_{x y} &= \xi_x \xi_y u_{\xi\xi} + (\xi_x \psi_y + \xi_y \psi_x) u_{\xi\psi}      + \psi_x \psi_y u_{\psi\psi} + \xi_{x y} u_\xi + \psi_{x y} u_\psi 
  \\
  u_{y y} &= \xi_y^2 u_{\xi\xi} + 2 \xi_y \psi_y u_{\xi\psi} + \psi_y^2 u_{\psi\psi} + \xi_{y y} u_{\xi} + \psi_{y y} u_\psi
\end{align*}
Substituting these into 
Equation~\ref{a(x,y)u_xx+2b(x,y)u_xy+c(x,y)u_yy=F(x,y,u,u_x,u_y)}
yields an equation in $\xi$ and $\psi$.
\begin{multline*}
  \label{left(axi_x^2+2bxi_xxi_y+cxi_y^2right)u_xixi}
  \left( a \xi_x^2 + 2 b \xi_x \xi_y + c \xi_y^2 \right) u_{\xi\xi}
  + 2 \left( a \xi_x \psi_x + b(\xi_x \psi_y + \xi_y \psi_x ) + c \xi_y \psi_y \right) u_{\xi\psi} 
  \\
  + \left( a \psi_x^2 + 2 b \psi_x \psi_y + c \psi_y^2 \right) u_{\psi\psi} = H(\xi, \psi, u, u_\xi, u_\psi)
\end{multline*}
\begin{equation}
  \label{alpha(xi,eta)u_xixi+beta(xi,eta)u_xieta}
  \alpha(\xi,\psi) u_{\xi \xi} + \beta(\xi,\psi) u_{\xi \psi} + \gamma(\xi,\psi) u_{\psi \psi} = H(\xi, \psi, u, u_\xi, u_\psi)
\end{equation}










%%------------------------------------------------------------------------------
\subsection{Hyperbolic Equations}



We start with a hyperbolic equation, ($b^2 - a c > 0$).
We seek a change of independent variables that will put 
Equation~\ref{a(x,y)u_xx+2b(x,y)u_xy+c(x,y)u_yy=F(x,y,u,u_x,u_y)}
in the form
\begin{equation}
  \label{u_xieta=G(xi,eta,u,u_xi,u_eta)}
  u_{\xi \psi} = G(\xi, \psi, u, u_\xi, u_\psi)
\end{equation}
We require that the $u_{\xi\xi}$ and $u_{\psi\psi}$ terms vanish.
That is $\alpha = \gamma = 0$ in 
Equation~\ref{alpha(xi,eta)u_xixi+beta(xi,eta)u_xieta}.
This gives us two constraints on $\xi$ and $\psi$.
\begin{gather}
  \label{axi_x^2+2bxi_xxi_y+cxi_y^2=0}
  a \xi_x^2 + 2 b \xi_x \xi_y + c \xi_y^2 = 0, \qquad a \psi_x^2 + 2 b \psi_x \psi_y + c \psi_y^2 = 0 
  \\
  \nonumber
  \frac{\xi_x}{\xi_y} = \frac{-b + \sqrt{b^2 - a c}}{a}, \qquad
  \frac{\psi_x}{\psi_y} = \frac{-b - \sqrt{b^2 - a c}}{a}
  \\
  \nonumber
  \xi_x + \frac{b - \sqrt{b^2 - a c}}{a} \xi_y = 0, \qquad
  \psi_x + \frac{b + \sqrt{b^2 - a c}}{a} \psi_y = 0
\end{gather}
Here we chose the signs in the quadratic formulas to get different solutions
for $\xi$ and $\psi$.

Now we have first order quasi-linear partial differential equations for 
the coordinates $\xi$ and $\psi$.  We solve these equations with the method
of characteristics.  The characteristic equations for $\xi$ are 
\[
\frac{\dd y}{\dd x} = \frac{b - \sqrt{b^2 - a c}}{a}, \quad
\frac{\dd}{\dd x} \xi(x,y(x)) = 0
\]
Solving the differential equation for $y(x)$ determines 
$\xi(x,y)$.  We just write the solution for $y(x)$ in the form
$F(x,y(x)) = \mathrm{const}$.  Since the solution of the differential equation
for $\xi$ is $\xi(x,y(x)) = \mathrm{const}$, we then have $\xi = F(x,y)$.
Upon solving for $\xi$ and $\psi$ we divide 
Equation~\ref{alpha(xi,eta)u_xixi+beta(xi,eta)u_xieta}
by $\beta(\xi,\psi)$ to obtain the canonical form.





Note that we could have solved for $\xi_y /  \xi_x$ in 
Equation~\ref{axi_x^2+2bxi_xxi_y+cxi_y^2=0}.
\[
\frac{\dd x}{\dd y} = - \frac{\xi_y}{\xi_x} = \frac{b - \sqrt{b^2 - a c}}{c}
\]
This form is useful if $a$ vanishes.





Another canonical form for  hyperbolic equations is
\begin{equation}
  \label{u_sigmasigma-u_tautau=K(sigma,tau,u,u_sigma,u_tau)}
  u_{\sigma \sigma} - u_{\tau \tau} = K(\sigma, \tau, u, u_\sigma, u_\tau).
\end{equation}
We can transform Equation~\ref{u_xieta=G(xi,eta,u,u_xi,u_eta)} 
to this form with the change of variables
\[
\sigma = \xi + \psi, \quad \tau = \xi - \psi.
\]
Equation~\ref{u_xieta=G(xi,eta,u,u_xi,u_eta)} becomes
\[
u_{\sigma \sigma} - u_{\tau \tau} 
= G \left( \frac{\sigma + \tau}{2}, \frac{\sigma - \tau}{2}, u, u_\sigma + u_\tau, u_\sigma - u_\tau \right).
\]










\begin{Example}
  Consider the wave equation with a source.
  \[
  u_{t t} - c^2 u_{x x} = s(x,t)
  \]
  Since $0 - (1)(-c^2) > 0$, the equation is hyperbolic.
  We find the new variables.
  \begin{gather*}
    \frac{\dd x}{\dd t} = - c, \quad 
    x = - c t + \mathrm{const}, \quad
    \xi = x + c t 
    \\
    \frac{\dd x}{\dd t} = c, \quad 
    x = c t + \mathrm{const}, \quad
    \psi = x - c t
  \end{gather*}
  Then we determine $t$ and $x$ in terms of $\xi$ and $\psi$.
  \[
  t = \frac{\xi - \psi}{2 c}, \quad x = \frac{\xi + \psi}{2}
  \]

  We calculate the derivatives of $\xi$ and $\psi$.
  \begin{alignat*}{2}
    \xi_t &= c &\quad \xi_x &= 1 
    \\
    \psi_t &= - c &\quad \psi_x &= 1 
  \end{alignat*}
  Then we calculate the derivatives of $u$.
  \begin{align*}
    u_{t t} &= c^2 u_{\xi \xi} - 2 c^2 u_{\xi \psi} + c^2 u_{\psi \psi} 
    \\
    u_{x x} &= u_{\xi \xi} + u_{\psi \psi} 
  \end{align*}
  Finally we transform the equation to canonical form.
  \begin{gather*}
    - 2 c^2 u_{\xi \psi} = s \left( \frac{\xi + \psi}{2}, \frac{\xi - \psi}{2 c} \right) 
    \\
    \boxed{
      u_{\xi \psi} = - \frac{1}{2 c^2} s \left( \frac{\xi + \psi}{2}, \frac{\xi - \psi}{2 c} \right)
      }
  \end{gather*}

  If $s(x,t) = 0$, then the equation is $u_{\xi \psi} = 0$ we can
  integrate with respect to $\xi$ and $\psi$ to obtain the 
  solution, $u = f(\xi) + g(\psi)$.  Here $f$ and $g$ are arbitrary
  $C^2$ functions.  In terms of $t$ and $x$, we have
  \[
  \boxed{
    u(x,t) = f(x + c t) + g(x - c t).
    }
  \]

  To put the wave equation in the form of
  Equation~\ref{u_sigmasigma-u_tautau=K(sigma,tau,u,u_sigma,u_tau)}
  we make a change of variables
  \begin{gather*}
    \sigma = \xi + \psi = 2 x, \quad \tau = \xi - \psi = 2 c t
    \\
    u_{t t} - c^2 u_{x x} = s(x,t) 
    \\
    4 c^2 u_{\tau \tau} - 4 c^2 u_{\sigma \sigma} = s \left( \frac{\sigma}{2}, \frac{\tau}{2 c} \right) 
    \\
    \boxed{
      u_{\sigma \sigma} - u_{\tau \tau} = - \frac{1}{4 c^2} s \left( \frac{\sigma}{2}, \frac{\tau}{2 c} \right)
      }
  \end{gather*}
\end{Example}











\begin{Example}
  \label{y^2u_xx-x^2u_yy=0}
  Consider
  \[
  y^2 u_{x x} - x^2 u_{y y} = 0.
  \]
  For $x \neq 0$ and $y \neq 0$ this equation is hyperbolic. We find the new variables.
  \begin{gather*}
    \frac{\dd y}{\dd x} = - \frac{\sqrt{ y^2 x^2 }}{y^2} = - \frac{x}{y}, \quad 
    y \,\dd y = - x \,\dd x, \quad
    \frac{y^2}{2} = - \frac{x^2}{2} + \mathrm{const}, \quad
    \xi = y^2 + x^2 
    \\
    \frac{\dd y}{\dd x} = \frac{\sqrt{ y^2 x^2 }}{y^2} = \frac{x}{y}, \quad 
    y \,\dd y = x \,\dd x, \quad
    \frac{y^2}{2} = \frac{x^2}{2} + \mathrm{const}, \quad
    \psi = y^2 - x^2
  \end{gather*}

  We calculate the derivatives of $\xi$ and $\psi$.
  \begin{alignat*}{2}
    \xi_x &= 2 x &\quad \xi_y &= 2 y 
    \\
    \psi_x &= -2 x  &\quad \psi_y &= 2 y 
  \end{alignat*}
  Then we calculate the derivatives of $u$.
  \begin{gather*}
    u_x = 2 x ( u_\xi - u_\psi ) 
    \\
    u_y = 2 y ( u_\xi + u_\psi ) 
    \\
    u_{x x} = 4 x^2 ( u_{\xi \xi} - 2 u_{\xi \psi} + u_{\psi \psi} ) + 2 ( u_\xi - u_\psi ) 
    \\
    u_{y y} = 4 y^2 ( u_{\xi \xi} + 2 u_{\xi \psi} + u_{\psi \psi} ) + 2 ( u_\xi + u_\psi )
  \end{gather*}
  Finally we transform the equation to canonical form.
  \begin{gather*}
    y^2 u_{x x} - x^2 u_{y y} = 0 
    \\
    - 8 x^2 y^2 u_{\xi \psi} - 8 x^2 y^2 u_{\xi \psi} + 2 y^2 ( u_\xi - u_\psi ) + 2 x^2 ( u_\xi + u_\psi ) = 0 
    \\
    16 \frac{1}{2} (\xi - \psi) \frac{1}{2} (\xi + \psi) u_{\xi \psi} = 2 \xi u_\xi - 2 \psi u_\psi 
    \\
    \boxed{
      u_{\xi \psi} = \frac{\xi u_\xi - \psi u_\psi}{ 2 (\xi^2 - \psi^2) }
      }
  \end{gather*}
\end{Example}









\begin{Example}
  Consider Laplace's equation.
  \[
  u_{x x} + u_{y y} = 0
  \]
  Since $0 - (1)(1) < 0$, the equation is elliptic.
  We will transform this equation to the canical form of 
  Equation~\ref{u_xieta=G(xi,eta,u,u_xi,u_eta)}.
  We find the new variables.
  \begin{gather*}
    \frac{\dd y}{\dd x} = - \imath, \quad 
    y = - \imath x + \mathrm{const}, \quad
    \xi = x + \imath y 
    \\
    \frac{\dd y}{\dd x} = \imath, \quad 
    y = \imath x + \mathrm{const}, \quad
    \psi = x - \imath y
  \end{gather*}

  We calculate the derivatives of $\xi$ and $\psi$.
  \begin{alignat*}{2}
    \xi_x &= 1 &\quad \xi_y &= \imath 
    \\
    \psi_x &= 1 &\quad \psi_y &= -\imath 
  \end{alignat*}
  Then we calculate the derivatives of $u$.
  \begin{align*}
    u_{x x} &= u_{\xi \xi} + 2 u_{\xi \psi} + u_{\psi \psi} 
    \\
    u_{y y} &= - u_{\xi \xi} + 2 u_{\xi \psi} - u_{\psi \psi}
  \end{align*}
  Finally we transform the equation to canonical form.
  \begin{gather*}
    4 u_{\xi \psi} = 0 
    \\
    \boxed{
      u_{\xi \psi} = 0
      }
  \end{gather*}

  We integrate with respect to $\xi$ and $\psi$ to obtain the 
  solution, $u = f(\xi) + g(\psi)$.  Here $f$ and $g$ are arbitrary
  $C^2$ functions.  In terms of $x$ and $y$, we have
  \[
  \boxed{
    u(x,y) = f(x + \imath y) + g(x - \imath y).
    }
  \]
  This solution makes a lot of sense, because the real and imaginary
  parts of an analytic function are harmonic.
\end{Example}









%%------------------------------------------------------------------------------
\subsection{Parabolic equations}


Now we consider a parabolic equation, ($b^2 - a c = 0$).
We seek a change of independent variables that will put 
Equation~\ref{a(x,y)u_xx+2b(x,y)u_xy+c(x,y)u_yy=F(x,y,u,u_x,u_y)}
in the form
\begin{equation}
  \label{eqn uxixi=Gxipsiuuu}
  u_{\xi \xi} = G(\xi, \psi, u, u_\xi, u_\psi).
\end{equation}
We require that the $u_{\xi \psi}$ and $u_{\psi\psi}$ terms vanish.
That is $\beta = \gamma = 0$ in 
Equation~\ref{alpha(xi,eta)u_xixi+beta(xi,eta)u_xieta}.
This gives us two constraints on $\xi$ and $\psi$.
\[
a \xi_x \psi_x + b(\xi_x \psi_y + \xi_y \psi_x ) + c \xi_y \psi_y = 0, \qquad a \psi_x^2 + 2 b \psi_x \psi_y + c \psi_y^2 = 0 
\]
We consider the case $a \neq 0$.  The latter constraint allows us to solve 
for $\psi_x / \psi_y$.
\[
\frac{\psi_x}{\psi_y} = \frac{-b - \sqrt{b^2 - a c}}{a} = - \frac{b}{a}
\]
With this information, the former constraint is trivial.
\begin{gather*}
  a \xi_x \psi_x + b(\xi_x \psi_y + \xi_y \psi_x ) + c \xi_y \psi_y = 0
  \\
  a \xi_x (-b/a) + b(\xi_x + \xi_y (-b/a) ) + c \xi_y = 0
  \\
  (a c - b^2) \xi_y = 0
  \\
  0 = 0
\end{gather*}
Thus we have a first order partial differential equation for the $\psi$ 
coordinate which we can solve with the method of characteristics.
\[
\psi_x + \frac{b}{a} \psi_y = 0
\]
The $\xi$ coordinate is chosen to be anything linearly independent of $\psi$.
The characteristic equations for $\psi$ are 
\[
\frac{\dd y}{\dd x} = \frac{b}{a}, \quad
\frac{\dd}{\dd x} \psi(x,y(x)) = 0
\]
Solving the differential equation for $y(x)$ determines 
$\psi(x,y)$.  We just write the solution for $y(x)$ in the form
$F(x,y(x)) = \mathrm{const}$.  Since the solution of the differential equation
for $\psi$ is $\psi(x,y(x)) = \mathrm{const}$, we then have $\psi = F(x,y)$.
Upon solving for $\psi$ and choosing a linearly independent $\xi$, we divide 
Equation~\ref{alpha(xi,eta)u_xixi+beta(xi,eta)u_xieta}
by $\alpha(\xi,\psi)$ to obtain the canonical form.

In the case that $a = 0$, we would instead have the constraint,
\[
\psi_x + \frac{b}{c} \psi_y = 0.
\]



%% CONTINUE example





%%------------------------------------------------------------------------------
\subsection{Elliptic Equations}



We start with an elliptic equation, ($b^2 - a c < 0$).
We seek a change of independent variables that will put 
Equation~\ref{a(x,y)u_xx+2b(x,y)u_xy+c(x,y)u_yy=F(x,y,u,u_x,u_y)}
in the form
\begin{equation}
  \label{u_sigmasigma+u_tautau=G(sigma,tau,u,u_sigma,u_tau)}
  u_{\sigma \sigma} + u_{\tau \tau} = G(\sigma, \tau, u, u_\sigma, u_\tau)
\end{equation}
If we make the change of variables determined by
\[
\frac{\xi_x}{\xi_y} = \frac{-b + \imath \sqrt{a c - b^2}}{a}, \qquad
\frac{\psi_x}{\psi_y} = \frac{-b - \imath \sqrt{a c - b^2}}{a},
\]
the equation will have the form
\[
u_{\xi \psi} = G(\xi, \psi, u, u_\xi, u_\psi).
\]
$\xi$ and $\psi$ are complex-valued.  If we then make the change of 
variables 
\[
\sigma = \frac{\xi + \psi}{2}, \quad
\tau = \frac{\xi - \psi}{2 \imath}
\]
we will obtain the canonical form of 
Equation~\ref{u_sigmasigma+u_tautau=G(sigma,tau,u,u_sigma,u_tau)}.
Note that since $\xi$ and $\psi$ are complex conjugates, $\sigma$ and $\tau$ are 
real-valued.





\begin{Example}
  Consider
  \begin{equation}
    \label{y^2u_xx+x^2u_yy=0}
    y^2 u_{x x} + x^2 u_{y y} = 0.
  \end{equation}
  For $x \neq 0$ and $y \neq 0$ this equation is elliptic.
  We find new variables that will put this equation in the form
  $u_{\xi \psi} = G(\cdot)$.  From Example~\ref{y^2u_xx-x^2u_yy=0}
  we see that they are
  \begin{gather*}
    \frac{\dd y}{\dd x} = - \imath \frac{\sqrt{ y^2 x^2 }}{y^2} 
    = - \imath \frac{x}{y}, \quad 
    y \,\dd y = - \imath x \,\dd x, \quad
    \frac{y^2}{2} = - \imath \frac{x^2}{2} + \mathrm{const}, \quad
    \xi = y^2 + \imath x^2 
    \\
    \frac{\dd y}{\dd x} = \imath \frac{\sqrt{ y^2 x^2 }}{y^2} = \imath \frac{x}{y}, \quad 
    y \,\dd y = \imath x \,\dd x, \quad
    \frac{y^2}{2} = \imath \frac{x^2}{2} + \mathrm{const}, \quad
    \psi = y^2 - \imath x^2 \\
  \end{gather*}
  The variables that will put Equation~\ref{y^2u_xx+x^2u_yy=0} in 
  canonical form are
  \[
  \sigma = \frac{\xi + \psi}{2} = y^2, \quad
  \tau = \frac{\xi - \psi}{2 \imath} = x^2
  \]
  We calculate the derivatives of $\sigma$ and $\tau$.
  \begin{alignat*}{2}
    \sigma_x &= 0 &\quad \sigma_y &= 2 y 
    \\
    \tau_x &= 2 x  &\quad \tau_y &= 0
  \end{alignat*}
  Then we calculate the derivatives of $u$.
  \begin{gather*}
    u_x = 2 x u_\tau 
    \\
    u_y = 2 y u_\sigma 
    \\
    u_{x x} = 4 x^2 u_{\tau \tau} + 2 u_\tau 
    \\
    u_{y y} = 4 y^2 u_{\sigma \sigma} + 2 u_\sigma
  \end{gather*}
  Finally we transform the equation to canonical form.
  \begin{gather*}
    y^2 u_{x x} + x^2 u_{y y} = 0 
    \\
    \sigma ( 4 \tau u_{\tau \tau} + 2 u_\tau ) + \tau ( 4 \sigma u_{\sigma \sigma} + 2 u_\sigma ) = 0 
    \\
    \boxed{
      u_{\sigma \sigma} + u_{\tau \tau} = - \frac{1}{2 \sigma} u_\sigma - \frac{1}{2 \tau} u_{\tau}
      }
  \end{gather*}
\end{Example}














%%==============================================================================
\section{Equilibrium Solutions}


\begin{Example}
  Consider the equilibrium solution for the following problem.
  \[ 
  u_t = u_{x x}, \qquad u(x,0) = x, \quad u_x(0,t) = u_x(1,t) = 0
  \]
  Setting $u_t = 0$ we have an ordinary differential equation.
  \[ 
  \frac{\dd^2 u}{\dd x^2} = 0
  \]
  This equation has the solution,
  \[
  u = a x + b. 
  \]
  Applying the boundary conditions we see that
  \[ 
  u = b. 
  \]
  To determine the constant, we note that the heat energy in the rod is
  constant in time.
  \begin{gather*}
    \int_0^1 u(x,t)\,\dd x = \int_0^1 u(x,0) \,\dd x 
    \\
    \int_0^1 b \,\dd x = \int_0^1 x \,\dd x
  \end{gather*}
  Thus the equilibrium solution is
  \[ 
  u(x) = \frac{1}{2}. 
  \]
\end{Example}




%%%%%%%%%%%%%%%%%%%%%%%%%%%%%%%%%%%%%%%%%%%%%%%%%%%%%%%%%%%%%%%%%%%%%%%%%%%%%%%%%%%%%%%%%%%%%%%%%%%%%%%%%%%%%%%%%%%%%%%%%%%%%%%%%%%%%%%%%%%%%%%%%%%%%%%%%%%%%%
%% Do the same with the backward heat equation to show that a problem
%% may not have an equilibrium solution.
%%%%%%%%%%%%%%%%%%%%%%%%%%%%%%%%%%%%%%%%%%%%%%%%%%%%%%%%%%%%%%%%%%%%%%%%%%%%%%%%%%%%%%%%%%%%%%%%%%%%%%%%%%%%%%%%%%%%%%%%%%%%%%%%%%%%%%%%%%%%%%%%%%%%%%%%%%%%%%

















\raggedbottom
%%=============================================================================
\exercises{
\pagebreak
\flushbottom
\section{Exercises}






\begin{Exercise}
  \label{exercise classify transform uxx+1y2uyy}
  Classify and transform the following equation into canonical form.
  \[ 
  u_{x x} + (1+y)^2 u_{y y} = 0
  \]

  \hintsolution{classify transform uxx+1y2uyy}
\end{Exercise}










%%Classify in a region $R$ each of the equations:
\begin{Exercise}
  \label{exercise classify ut=puxx}
  Classify as hyperbolic, parabolic, or
  elliptic in a region $R$ each of the equations:
  \begin{enumerate}
  \item 
    $\displaystyle u_t = (p u_x)_x$
  \item 
    $\displaystyle u_{tt} = c^2u_{xx} - \gamma u$
  \item 
    $\displaystyle (qu_x)_x + (qu_t)_t = 0$
  \end{enumerate}
  where $p(x)$, $c(x,t)$, $q(x,t)$, and $\gamma(x)$ are given functions
  that take on only positive values in a region $R$ of the $(x,t)$ plane.

  \hintsolution{classify ut=puxx}
\end{Exercise}








%% Transform each of the following equations into canonical form.
\begin{Exercise}
  \label{exercise transform phixx-y2phiyy+phix-phi+x2=0}
  Transform each of the following equations
  for $\phi(x,y)$ into canonical form in appropriate regions
  \begin{enumerate}
  \item 
    $\displaystyle \phi_{xx} - y^2 \phi_{yy} + \phi_x -\phi + x^2 = 0$
  \item 
    $\displaystyle \phi_{xx} + x \phi_{yy} = 0$
  \end{enumerate}
  The equation in part (b) is known as {\it Tricomi's equation} and is
  a model for transonic fluid flow in which the flow speed changes from 
  supersonic to subsonic.

  \hintsolution{transform phixx-y2phiyy+phix-phi+x2=0}
\end{Exercise}






\raggedbottom
}
%%=============================================================================
\hints{
\pagebreak
\flushbottom
\section{Hints}








\begin{Hint}
  \label{hint classify transform uxx+1y2uyy}
  %% CONTINUE
\end{Hint}







%%Classify in a region $R$ each of the equations:
\begin{Hint}
  \label{hint classify ut=puxx}
  %% CONTINUE
\end{Hint}








%% Transform each of the following equations into canonical form.
\begin{Hint}
  \label{hint transform phixx-y2phiyy+phix-phi+x2=0}
  %% CONTINUE
\end{Hint}













\raggedbottom
}
%%=============================================================================
\solutions{
\pagebreak
\flushbottom
\section{Solutions}








\begin{Solution}
  \label{solution classify transform uxx+1y2uyy}
  For $y = -1$, the equation is parabolic.  For this case it is already in
  the canonical form, $u_{x x} = 0$.

  For $y \neq -1$, the equation is elliptic.  We find new variables that will put
  the equation in the form $u_{\xi \psi} = G(\xi,\psi,u,u_\xi,u_\psi)$.
  \begin{gather*}
    \frac{\dd y}{\dd x} = \imath \sqrt{(1+y)^2} = \imath (1 + y)
    \\
    \frac{\dd y}{1 + y} = \imath \dd x
    \\
    \log(1+y) = \imath x + c
    \\
    1 + y = c \e^{\imath x}
    \\
    (1 + y) \e^{-\imath x} = c
    \\
    \xi = (1 + y) \e^{-\imath x}
    \\
    \psi = \overline{\xi} = (1 + y) \e^{\imath x}
  \end{gather*}
  The variables that will put the equation in canonical form are 
  \[
  \sigma = \frac{\xi + \psi}{2} = (1 + y) \cos x, \quad
  \tau = \frac{\xi - \psi}{\imath 2} = (1 + y) \sin x.
  \]
  We calculate the derivatives of $\sigma$ and $\tau$.
  \begin{alignat*}{2}
    \sigma_x &= - (1 + y) \sin x &\quad \sigma_y &= \cos x
    \\
    \tau_x &= (1 + y) \cos x &\quad \tau_y &= \sin x
  \end{alignat*}
  Then we calculate the derivatives of $u$.
  \begin{gather*}
    u_x = - (1 + y) \sin(x) u_\sigma + (1 + y) \cos(x) u_\tau
    \\
    u_y = \cos(x) u_\sigma + \sin(x) u_\tau
    \\
    u_{x x} = (1 + y)^2 \sin^2(x) u_{\sigma \sigma} + (1 + y)^2 \cos^2(x) u_{\tau \tau}
    - (1 + y) \cos(x) u_\sigma - (1 + y) \sin(x) u_\tau
    \\
    u_{y y} = \cos^2(x) u_{\sigma \sigma} + \sin^2(x) u_{\tau \tau}
  \end{gather*}
  We substitute these results into the differential equation to obtain the 
  canonical form.
  \begin{gather*}
    u_{x x} + (1 + y)^2 u_{y y} = 0
    \\
    (1 + y)^2 \left( u_{\sigma \sigma} + u_{\tau \tau} \right) 
    - (1 + y) \cos(x) u_\sigma - (1 + y) \sin(x) u_\tau = 0
    \\
    \left( \sigma^2 + \tau^2 \right) \left( u_{\sigma \sigma} + u_{\tau \tau} \right) - \sigma u_\sigma - \tau u_\tau = 0
    \\
    \boxed{
      u_{\sigma \sigma} + u_{\tau \tau} = \frac{ \sigma u_\sigma + \tau u_\tau }{ \sigma^2 + \tau^2 }
      }
  \end{gather*}
\end{Solution}









%%Classify in a region $R$ each of the equations:
\begin{Solution}
  \label{solution classify ut=puxx}
  \begin{enumerate}


  \item
    \begin{gather*}
      u_t = (p u_x)_x 
      \\
      p u_{x x} + 0 u_{x t} + 0 u_{t t} + p_x u_x - u_t = 0
    \end{gather*}
    Since $0^2 - p 0 = 0$, the equation is parabolic.


  \item
    \begin{gather*}
      u_{tt} = c^2 u_{xx} - \gamma u 
      \\
      u_{tt} + 0 u_{t x} - c^2 u_{xx} + \gamma u = 0
    \end{gather*}
    Since $0^2 - (1)(-c^2) > 0$, the equation is hyperbolic.


  \item
    \begin{gather*}
      (q u_x)_x + (q u_t)_t = 0 
      \\
      q u_{x x} + 0 u_{x t} + q u_{t t} + q_x u_x + q_t u_t = 0
    \end{gather*}
    Since $0^2 - q q < 0$, the equation is elliptic.
  \end{enumerate}
\end{Solution}












%% Transform each of the following equations into canonical form.
\begin{Solution}
  \label{solution transform phixx-y2phiyy+phix-phi+x2=0}
  \begin{enumerate}


  \item
    For $y \neq 0$, the equation is hyperbolic.  We find the new 
    independent variables.
    \begin{gather*}
      \frac{\dd y}{\dd x} = \frac{\sqrt{y^2}}{1} = y, \quad
      y = c \e^x, \quad
      \e^{-x} y = c, \quad
      \xi = \e^{-x} y 
      \\
      \frac{\dd y}{\dd x} = \frac{-\sqrt{y^2}}{1} = -y, \quad
      y = c \e^{-x}, \quad
      \e^x y = c, \quad
      \psi = \e^x y
    \end{gather*}
    Next we determine $x$ and $y$ in terms of $\xi$ and $\psi$.
    \begin{gather*}
      \xi \psi = y^2, \quad y = \sqrt{\xi \psi} 
      \\
      \psi = \e^x \sqrt{\xi \psi}, \quad
      \e^x = \sqrt{\psi / \xi}, \quad
      x = \frac{1}{2} \log \left( \frac{\psi}{\xi} \right)
    \end{gather*}
    We calculate the derivatives of $\xi$ and $\psi$.
    \begin{gather*}
      \xi_x = - \e^{-x} y = - \xi 
      \\
      \xi_y = \e^{-x} = \sqrt{\xi / \psi} 
      \\
      \psi_x = \e^x y = \psi 
      \\
      \psi_y = \e^x = \sqrt{\psi / \xi}
    \end{gather*}
    Then we calculate the derivatives of $\phi$.
    \begin{gather*}
      \frac{\partial}{\partial x} = - \xi \frac{\partial}{\partial \xi} + \psi \frac{\partial}{\partial \psi}, \quad
      \frac{\partial}{\partial y} = \sqrt{ \frac{\xi}{\psi} } \frac{\partial}{\partial \xi}
      + \sqrt{ \frac{\psi}{\xi} } \frac{\partial}{\partial \psi} 
      \\
      \phi_x = -\xi \phi_\xi + \psi \phi_\psi, \quad
      \phi_y = \sqrt{ \frac{\xi}{\psi} } \phi_\xi + \sqrt{ \frac{\psi}{\xi} } \phi_\psi 
      \\
      \phi_{x x} = \xi^2 \phi_{\xi \xi} - 2 \xi \psi \phi_{\xi \psi} + \psi^2 \phi_{\psi \psi} + \xi \phi_\xi + \psi \phi_\psi,
      \quad
      \phi_{y y} = \frac{\xi}{\psi} \phi_{\xi \xi} + 2 \phi_{\xi \psi} + \frac{\psi}{\xi} \phi_{\psi \psi}
    \end{gather*}
    Finally we transform the equation to canonical form.
    \begin{gather*}
      \phi_{x x} - y^2 \phi_{y y} + \phi_x - \phi + x^2 = 0 
      \\
      -4 \xi \psi \phi_{\xi \psi} + \xi \phi_\xi + \psi \phi_\psi - \xi \phi_\xi + \psi \phi_\psi - \phi
      + \log \left( \frac{\psi}{\xi} \right) = 0 
      \\
      \boxed{
        \phi_{\xi \psi} = \frac{1}{2 \xi} \phi_\psi + \phi
        - \log \left( \frac{\psi}{\xi} \right)
        }
    \end{gather*}
    
    For $y = 0$ we have the ordinary differential equation
    \[
    \phi_{xx} + \phi_x -\phi + x^2 = 0.
    \]
    
    
  \item
    For $x < 0$, the equation is hyperbolic.  We find the new 
    independent variables.
    \begin{gather*}
      \frac{\dd y}{\dd x} = \sqrt{-x}, \quad
      y = \frac{2}{3} x \sqrt{-x} + c, \quad
      \xi = \frac{2}{3} x \sqrt{-x} - y 
      \\
      \frac{\dd y}{\dd x} = - \sqrt{-x}, \quad
      y = - \frac{2}{3} x \sqrt{-x} + c, \quad
      \psi = \frac{2}{3} x \sqrt{-x} + y
    \end{gather*}
    Next we determine $x$ and $y$ in terms of $\xi$ and $\psi$.
    \[
    x = - \left( \frac{3}{4} (\xi + \psi) \right)^{1/3}, \quad
    y = \frac{\psi - \xi}{2}
    \]
    We calculate the derivatives of $\xi$ and $\psi$.
    \begin{gather*}
      \xi_x = \sqrt{-x} = \left( \frac{3}{4} (\xi + \psi) \right)^{1/6}, \quad
      \xi_y = -1 
      \\
      \psi_x = \left( \frac{3}{4} (\xi + \psi) \right)^{1/6}, \quad
      \psi_y = 1
    \end{gather*}
    Then we calculate the derivatives of $\phi$.
    \begin{gather*}
      \phi_x = \left( \frac{3}{4} (\xi + \psi) \right)^{1/6} \left( \phi_\xi + \phi_\psi \right) 
      \\
      \phi_y = - \phi_\xi + \phi_\psi 
      \\
      \phi_{x x} = \left( \frac{3}{4} (\xi + \psi) \right)^{1/3}
      \left( \phi_{\xi \xi} + \phi_{\psi \psi} \right) + (6(\xi + \psi))^{1/3} \phi_{\xi \psi}
      + (6(\xi + \psi))^{-2/3} \left( \phi_\xi + \phi_\psi \right) 
      \\
      \phi_{y y} = \phi_{\xi \xi} - 2 \phi_{\xi \psi} + \phi_{\psi \psi}
    \end{gather*}
    Finally we transform the equation to canonical form.
    \begin{gather*}
      \phi_{x x} + x \phi_{y y} = 0 
      \\
      (6(\xi + \psi))^{1/3} \phi_{\xi \psi} + (6(\xi + \psi))^{1/3} \phi_{\xi \psi}
      + (6(\xi + \psi))^{-2/3} \left( \phi_\xi + \phi_\psi \right) = 0 
      \\
      \boxed{
        \phi_{\xi \psi} = - \frac{ \phi_\xi + \phi_\psi }{ 12 (\xi + \psi) }
        }
    \end{gather*}

    For $x > 0$, the equation is elliptic.  The variables we defined
    before are complex-valued.
    \[
    \xi = \imath \frac{2}{3} x^{3/2} - y, \quad
    \psi = \imath \frac{2}{3} x^{3/2} + y
    \]
    We choose the new real-valued variables.
    \[
    \alpha = \xi - \psi, \quad
    \beta = -\imath (\xi + \psi)
    \]
    We write the derivatives in terms of $\alpha$ and $\beta$.
    \begin{gather*}
      \phi_\xi = \phi_\alpha - \imath \phi_\beta 
      \\
      \phi_\psi = - \phi_\alpha - \imath \phi_\beta 
      \\
      \phi_{\xi \psi} = - \phi_{\alpha \alpha} - \phi_{\beta \beta}
    \end{gather*}
    We transform the equation to canonical form.
    \begin{gather*}
      \phi_{\xi \psi} = - \frac{ \phi_\xi + \phi_\psi }{ 12 (\xi + \psi) } 
      \\
      - \phi_{\alpha \alpha} - \phi_{\beta \beta} = - \frac{ - 2 \imath \phi_{\beta} }{ 12 \imath \beta } 
      \\
      \boxed{
        \phi_{\alpha \alpha} + \phi_{\beta \beta} = - \frac{ \phi_\beta }{ 6 \beta }
        }
    \end{gather*}
  \end{enumerate}
\end{Solution}




\raggedbottom
}
