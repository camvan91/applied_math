\flushbottom




%%============================================================================
%%============================================================================
\chapter{Transform Methods}





%%============================================================================
\section{Fourier Transform for Partial Differential Equations}
Solve Laplace's equation in the upper half plane
\begin{alignat*}{2}
  &\nabla^2 u = 0 &\qquad
  & -\infty < x < \infty,\ y > 0 \\
  &u(x,0) = f(x) &\qquad
  &-\infty < x < \infty
\end{alignat*}

Taking the Fourier transform in the $x$ variable of the equation and the
boundary condition,
\begin{gather*}
  \mathcal{F}\left[ \frac{\partial^2 u}{\partial x^2} + \frac{\partial^2 u}{\partial y^2} \right] = 0, \qquad
  \mathcal{F}\left[u(x,0)\right] = \mathcal{F}\left[f(x)\right] \\
  -\omega^2 U(\omega,y) + \frac{\partial^2}{\partial y^2} U(\omega,y) = 0, \qquad
  U(\omega,0) = F(\omega).
\end{gather*}
The general solution to the equation is
\[U(\omega,y) = a \e^{\omega y} + b \e^{-\omega y}. \]
Remember that in solving the differential equation here we consider
$\omega$ to be a parameter.
Requiring that the solution be bounded for $y \in [0,\infty)$ yields
\[ U(\omega,y) = a \e^{-|\omega| y}. \]
Applying the boundary condition,
\[U(\omega, y) = F(\omega) \e^{-|\omega|y}. \]
The inverse Fourier transform of $\e^{-|\omega|y}$ is
\[ \mathcal{F}^{-1}\left[\e^{-|\omega|y}\right] = \frac{2 y}{x^2 + y^2}. \]
Thus
\begin{gather*}
  U(\omega, y) = F(\omega)\  \mathcal{F}\left[\frac{2 y}{x^2 + y^2}
  \right] \\
  \mathcal{F}\left[u(x,y)\right] =  \mathcal{F}\left[f(x)\right]
  \ \mathcal{F}\left[\frac{2 y}{x^2 + y^2} \right].
\end{gather*}
Recall that the convolution theorem is
\[ \mathcal{F}\left[ \frac{1}{2\pi}
  \int_{-\infty}^\infty f(x-\xi) g(\xi)\,d\xi\right] =
F(\omega)G(\omega). \]
Applying the convolution theorem to the equation for $U$,
\[ u(x,y) = \frac{1}{2\pi} \int_{-\infty}^\infty \frac{f(x-\xi) 2y}
{\xi^2 + y^2} \,d\xi \]
\[\boxed{u(x,y) = \frac{y}{\pi} \int_{-\infty}^\infty
  \frac{f(x-\xi)}{\xi^2 + y^2}\,d\xi.}\]









%%==============================================================================
\section{The Fourier Sine Transform}
%% CONTINUE: use the convolution theorem here.

Consider the problem
\begin{gather*}
  u_t = \kappa u_{x x}, \quad x > 0, \quad t > 0 \\
  u(0,t) = 0, \quad u(x,0) = f(x)
\end{gather*}
Since we are given the position at $x=0$ we apply the Fourier
sine transform.
\begin{gather*}
  \hat{u}_t = \kappa \left(-\omega^2 \hat{u} 
    + \frac{2}{\pi} \omega u(0,t) \right) \\
  \hat{u}_t = - \kappa \omega^2 \hat{u} \\
  \hat{u}(\omega,t) = c(\omega) \e^{-\kappa \omega^2 t}
\end{gather*}
The initial condition is
\[ 
\hat{u}(\omega,0) = \hat{f}(\omega).
\]
We solve the first order differential equation to determine $\hat{u}$.
\begin{gather*}
  \hat{u}(\omega,t) = \hat{f}(\omega) \e^{-\kappa \omega^2 t} \\ 
  \hat{u}(\omega,t) = \hat{f}(\omega) \mathcal{F}_c \left[
    \frac{ 1 }{ \sqrt{ 4 \pi \kappa t } } 
    \e^{ -x^2 / (4 \kappa t) } \right]
\end{gather*}
We take the inverse sine transform with the convolution theorem.
\[
\boxed{
  u(x,t) = \frac{ 1 }{ 4 \pi^{3/2} \sqrt{ \kappa t } } 
  \int_0^\infty f(\xi) \left( \e^{ -|x-\xi|^2 / (4 \kappa t) }
    - \e^{ -(x+\xi)^2 / (4 \kappa t) } \right) \,d \xi
  }
\]



%%=============================================================================
\section{Fourier Transform}

Consider the problem
\[
\frac{\partial u}{\partial t} - \frac{\partial u}{\partial x} + u = 0, \quad -\infty < x < \infty, 
\quad t > 0,
\]
\[
u(x,0) = f(x).
\]

Taking the Fourier Transform of the partial differential equation and 
the initial condition yields
\[
\frac{\partial U}{\partial t} - \imath \omega U + U = 0,
\]
\[
U(\omega,0) = F(\omega) = \frac{1}{2\pi} \int_{-\infty}^\infty f(x) \e^{-\imath \omega x}\,d x.
\]
Now we have a first order differential equation for $U(\omega,t)$ with 
the solution
\[
U(\omega,t) = F(\omega) \e^{(-1+ \imath \omega) t}.
\]
Now we apply the inverse Fourier transform.
\[
u(x,t) = \int_{-\infty}^\infty F(\omega) \e^{(-1+ \imath \omega) t} \e^{\imath \omega x}\,d\omega
\]
\[
u(x,t) = \e^{-t} \int_{-\infty}^\infty F(\omega) \e^{\imath \omega (x+t)} \,d\omega
\]
\[
\boxed{
  u(x,t) = \e^{-t} f(x+t)
  }
\]








\raggedbottom
%%=============================================================================
\exercises{
\pagebreak
\flushbottom
\section{Exercises}





%% u_{x x} + u_{y y} = 0\ \mathrm{in}\ -\infty < x < \infty,\ 0 < y < \infty,
\begin{Exercise}
  Find an integral representation of the solution $u(x,y)$, of
  \[
  u_{x x} + u_{y y} = 0\ \mathrm{in}\ -\infty < x < \infty,\ 0 < y < \infty,
  \]
  subject to the boundary conditions:
  \begin{gather*}
    u(x,0) = f(x),\ -\infty < x < \infty; \\
    u(x,y) \to 0\ \mathrm{as}\ x^2 + y^2 \to \infty.
  \end{gather*}
\end{Exercise}





%% Solve the Cauchy problem for the one-dimensional heat  equation.
\begin{Exercise}
  Solve the Cauchy problem for the one-dimensional heat equation in the
  domain $-\infty < x < \infty$, $t > 0$,
  \[
  u_{t} = \kappa u_{x x}, \quad u(x,0) = f(x), 
  \]
  with the Fourier transform.
\end{Exercise}





%% Solve the Cauchy problem for the 1D heat equation with Laplace transform.
\begin{Exercise}
  Solve the Cauchy problem for the one-dimensional heat equation in the
  domain $-\infty < x < \infty$, $t > 0$,
  \[
  u_{t} = \kappa u_{x x}, \quad u(x,0) = f(x), 
  \]
  with the Laplace transform.
\end{Exercise}




%% Cauchy problem for the 1D heat equation.  Method of images.
\begin{Exercise}
  \begin{enumerate}
  \item
    In Exercise~\ref{exercise_phi_t=a^2phi_xx_laplace} above, 
    let $f(-x) = -f(x)$ for
    all $x$ and verify that $\phi(x,t)$ so obtained is the solution,
    for $x > 0$, of the following problem: find $\phi(x,t)$ satisfying
    \[
    \phi_t = a^2\phi_{xx}
    \]
    in $0 < x < \infty$, $t > 0$, with boundary condition $\phi(0,t) = 0$
    and initial condition $\phi(x,0) = f(x)$. This technique, in which the
    solution for a semi-infinite interval is obtained from that for an
    infinite interval, 
    is an example of what is called the \textit{method of images}. 
  \item
    How would
    you modify the  result of part (a) if the boundary condition
    $\phi(0,t) = 0$ was replaced by $\phi_x(0,t) = 0$?
  \end{enumerate}
\end{Exercise}





%% Solve the Cauchy problem for the one-dimensional wave equation.
\begin{Exercise}
  Solve the Cauchy problem for the one-dimensional wave equation in the
  domain $-\infty < x < \infty$, $t>0$,
  \[
  u_{tt} = c^2 u_{xx}, \quad u(x,0) = f(x), \quad u_t(x,0) = g(x),
  \]
  with the Fourier transform.
\end{Exercise}



\begin{Exercise}
  Solve the Cauchy problem for the one-dimensional wave equation in the
  domain $-\infty < x < \infty$, $t>0$,
  \[
  u_{tt} = c^2 u_{xx}, \quad u(x,0) = f(x), \quad u_t(x,0) = g(x),
  \]
  with the Laplace transform.
\end{Exercise}




%% Heat eqn.  x > 0, t > 0.  \phi(x,0) = 0.  \phi(0,t) = f(t).
\begin{Exercise}
  Consider the problem of determining $\phi(x,t)$
  in the region $ 0 < x < \infty$, $0 < t < \infty$, such that
  \begin{equation}
    \label{1:phi_t=a^2phi_xx}
    \phi_t = a^2 \phi_{xx},
  \end{equation}
  with initial and boundary conditions
  \begin{alignat*}{2}
    &\phi(x,0) = 0            &\quad &\mathrm{for all}\ x > 0, \\
    &\phi(0,t) = f(t) &\quad &\mathrm{for all}\ t > 0,
  \end{alignat*}
  where $f(t)$ is a given function. 
  \begin{enumerate}
  \item 
    Obtain the formula for the Laplace
    transform of $\phi(x,t)$, $\Phi(x,s)$ and use the convolution theorem
    for Laplace transforms to show that
    \[
    \phi(x,t) = \frac{x}{2 a \sqrt{\pi}} \int_0^t f(t - \tau) 
    \frac{1}{\tau^{3/2}} \exp \left( - \frac{x^2}{4 a^2 \tau} 
    \right) \,d\tau.
    \]
  \item
    Discuss the special case obtained by setting $f(t) = 1$ and 
    also that in which $f(t) = 1$ for $0 < t < T$, with $ f(t) = 0$ for 
    $ t > T$. Here $T$ is some positive constant.
  \end{enumerate}
\end{Exercise}





%% Radiating half space problem.
\begin{Exercise}
  Solve the radiating half space problem:
  \begin{gather*}
    u_t = \kappa u_{x x}, \quad x > 0, \quad t > 0, \\
    u_x(0,t) - \alpha u(0,t) = 0, \quad u(x,0) = f(x).
  \end{gather*}
  To do this, define 
  \[
  v(x,t) = u_x(x,t) - \alpha u(x,t)
  \]
  and find the half space problem that $v$ satisfies.  Solve this problem and
  then show that 
  \[
  u(x,t) = - \int_x^\infty \e^{-\alpha (\xi-x)} v(\xi,t) \,d\xi.
  \]
\end{Exercise}



%% Solve heat equation with sine transform.
\begin{Exercise}
  Show that
  \[
  \int_0^\infty \omega \e^{-c \omega^2} \sin(\omega x) \,d \omega 
  = \frac{ x \sqrt{\pi} }{ 4 c^{3/2} } \e^{-x^2 / (4 c)}.
  \]
  Use the sine transform to solve:
  \begin{gather*}
    u_t = u_{x x}, \quad x > 0, \quad t > 0, \\
    u(0,t) = g(t), \quad u(x,0) = 0.
  \end{gather*}
\end{Exercise}


%% Steady state temperature in a semi-infinite rectangular slab.
\begin{Exercise}
  Use the Fourier sine transform to find the steady state temperature
  $u(x,y)$ in a slab: $x \geq 0$,
  $0 \leq y \leq 1$, which has zero temperature on the faces $y = 0$ and
  $y = 1$ and has a given distribution: $u(y,0) = f(y)$ on the
  edge $x = 0$, $0 \leq y \leq 1$.
\end{Exercise}



%% Potential equation in the upper half plane.
\begin{Exercise}
  Find a harmonic function $u(x,y)$ in the upper half plane which takes on the
  value $g(x)$ on the $x$-axis.  Assume that $u$ and $u_x$ vanish as
  $|x| \to \infty$.  Use the Fourier transform with respect to $x$.  
  Express the solution as a single integral by using the convolution 
  formula.
\end{Exercise}



%% Bounded solution of $u_t = \kappa u_{x x} - \a^2 u$.
\begin{Exercise}
  Find the bounded solution of
  \begin{gather*}
    u_t = \kappa u_{x x} - a^2 u, \quad 0 < x < \infty, t > 0, \\
    -u_x(0,t) = f(t), \quad u(x,0) = 0.
  \end{gather*}
\end{Exercise}





%% Finite string, Laplace transform
\begin{Exercise}
  The left end of a taut string of length $L$ is displaced according to
  $u(0,t) = f(t)$.  The right end is fixed, $u(L,t) = 0$.  Initially the
  string is at rest with no displacement.  If $c$ is the wave speed for the 
  string, find its motion for all $t > 0$.
\end{Exercise}









%% $\nabla^2 \phi = 0$, $0 < y < l$, $-\infty < x < \infty$.
\begin{Exercise}
  Let $\nabla^2 \phi = 0$ in the $(x,y)$-plane 
  region defined by $0 < y < l$, $-\infty < x < \infty$, with 
  $\phi(x,0) = \delta(x-\xi)$, $\phi(x,l) = 0$, and $\phi \to 0$ as
  $|x| \to \infty$. Solve for $\phi$ using Fourier transforms. You may
  leave your answer in the form of an integral but in fact it is possible to use
  techniques of contour integration to show that
  \[
  \phi(x,y|\xi) = \frac{1}{2l} \left [ \frac{\sin (\pi y/l)}
    {\cosh[\pi (x-\xi)/l] - \cos(\pi y/l)}\right].
  \]
  Note that as $l \to \infty$ we recover the result derived in 
  class:
  \[
  \phi \to  \frac{1}{\pi} \frac{y}{(x-\xi)^2 + y^2},
  \]
  which clearly approaches $\delta(x-\xi)$ as $y \to 0$.
\end{Exercise}










\raggedbottom
}
%%=============================================================================
\hints{
\pagebreak
\flushbottom
\section{Hints}








%% u_{x x} + u_{y y} = 0\ \mathrm{in}\ -\infty < x < \infty,\ 0 < y < \infty,
\begin{Hint}
  The desired solution form is: 
  $u(x,y) = \int_{-\infty}^\infty K(x-\xi,y) f(\xi)\,d\xi$.  You must find the correct $K$.
  Take the Fourier transform with respect to $x$ and solve for 
  $\hat{u}(\omega, y)$ recalling that $\hat{u}_{x x} = - \omega^2 \hat{u}$.
  By $\hat{u}_{x x}$ we denote the Fourier transform with respect to $x$
  of $u_{x x}(x,y)$.
\end{Hint}




%% Solve the Cauchy problem for the one-dimensional heat  equation.
\begin{Hint}
  Use the Fourier convolution theorem and the table of Fourier transforms
  in the appendix.
\end{Hint}



%% Solve the Cauchy problem for the 1D heat equation with Laplace transform.
\begin{Hint}
  %% CONTINUE
\end{Hint}


%% Cauchy problem for the 1D heat equation.  Method of images.
\begin{Hint}
  %% CONTINUE
\end{Hint}


%% Solve the Cauchy problem for the one-dimensional wave equation.
\begin{Hint}
  Use the Fourier convolution theorem.  The transform pairs,
  \begin{align*}
    \mathcal{F}[\pi ( \delta(x+\tau) + \delta(x-\tau))]
    &= \cos(\omega \tau), \\
    \mathcal{F}[\pi ( H(x+\tau) - H(x-\tau) )]
    &= \frac{\sin(\omega \tau)}{\omega},
  \end{align*}
  will be useful.
\end{Hint}


\begin{Hint}
  %%CONTINUE
\end{Hint}




%% Heat eqn.  x > 0, t > 0.  \phi(x,0) = 0.  \phi(0,t) = f(t).
\begin{Hint}
  %%CONTINUE
\end{Hint}




%% Radiating half space problem.
\begin{Hint}
  $v(x,t)$ satisfies the same partial differential equation.  You can solve
  the problem for $v(x,t)$ with the Fourier sine transform.  Use the 
  convolution theorem to invert the transform.

  To show that 
  \[
  u(x,t) = - \int_x^\infty \e^{-\alpha (\xi-x)} v(\xi,t) \,d\xi,
  \]
  find the solution of 
  \[
  u_x - \alpha u = v
  \]
  that is bounded as $x \to \infty$.
\end{Hint}



%% Solve heat equation with sine transform.
\begin{Hint}
  Note that
  \[
  \int_0^\infty \omega \e^{-c \omega^2} \sin(\omega x) \,d \omega 
  = - \frac{\partial}{\partial x} \int_0^\infty \e^{-c \omega^2} \cos(\omega x) \,d \omega.
  \]
  Write the integral as a Fourier transform.

  Take the Fourier sine transform of the heat equation to obtain a first order,
  ordinary differential equation for $\hat{u}(\omega,t)$.  Solve the 
  differential equation and do the inversion with the convolution 
  theorem.
\end{Hint}






%% Steady state temperature in a semi-infinite rectangular slab.
\begin{Hint}
  %%CONTINUE
\end{Hint}



%% Potential equation in the upper half plane.
\begin{Hint}
  %%CONTINUE
\end{Hint}



%% Bounded solution of $u_t = \kappa u_{x x} - \a^2 u$.
\begin{Hint}
  %%CONTINUE
\end{Hint}








%% Finite string, Laplace transform
\begin{Hint}
  %% CONTINUE
\end{Hint}







%% $\nabla^2 \phi = 0$, $0 < y < l$, $-\infty < x < \infty$.
\begin{Hint}
  %% CONTINUE
\end{Hint}











\raggedbottom
}
%%=============================================================================
\solutions{
\pagebreak
\flushbottom
\section{Solutions}







%% u_{x x} + u_{y y} = 0\ \mathrm{in}\ -\infty < x < \infty,\ 0 < y < \infty,
\begin{Solution}
  \begin{enumerate}
    %%
    %%
  \item
    We take the Fourier transform of the integral equation, noting that
    the left side is the convolution of $u(x)$ and $\frac{1}{x^2 + a^2}$.
    \[
    2 \pi \hat{u}(\omega) \mathcal{F} \left[ \frac{1}{x^2 + a^2} \right]
    = \mathcal{F} \left[ \frac{1}{x^2 + b^2} \right]
    \]

    We find the Fourier transform of $f(x) = \frac{1}{x^2 + c^2}$.
    Note that since $f(x)$ is an even, real-valued function, $\hat{f}(\omega)$
    is an even, real-valued function.
    \begin{align*}
      \mathcal{F} \left[ \frac{1}{x^2 + c^2} \right]
      &= \frac{1}{2\pi} \int_{-\infty}^\infty \frac{1}{x^2 + c^2} \e^{-\imath \omega x} \,d x \\
      \intertext{For $x > 0$ we close the path of integration in the upper half 
        plane and apply Jordan's Lemma to evaluate the integral in terms 
        of the residues.}
      &= \frac{1}{2 \pi} \imath 2 \pi \Res \left( \frac{\e^{- \imath \omega x}}
        {(x - \imath c)(x + \imath c)}, x = \imath c \right) \\
      &= \imath \frac{ \e^{-\imath \omega \imath c } }{ \imath 2 c } \\
      &= \frac{1}{2 c} \e^{- c \omega }
    \end{align*}
    Since $\hat{f}(\omega)$ is an even function, we have
    \[
    \mathcal{F} \left[ \frac{1}{x^2 + c^2} \right]
    = \frac{1}{2 c} \e^{- c |\omega| }.
    \]

    Our equation for $\hat{u}(\omega)$ becomes,
    \begin{gather*}
      2 \pi \hat{u}(\omega) \frac{1}{2 a} \e^{-a |\omega| }
      = \frac{1}{2 b} \e^{-b |\omega|} \\
      \hat{u}(\omega) = \frac{ a }{ 2 \pi b } \e^{- (b-a) |omega| }.
    \end{gather*}
    We take the inverse Fourier transform using the transform pair we derived
    above.
    \begin{gather*}
      u(x) = \frac{ a }{ 2 \pi b } \frac{ 2 (b-a) }{ x^2 + (b-a)^2 } \\
      \boxed{
        u(x) = \frac{ a (b-a) }{ \pi b ( x^2 + (b-a)^2 ) } 
        }
    \end{gather*}
    %%
    %%
  \item
    We take the Fourier transform of the partial differential equation and the
    boundary condtion.
    \begin{align*}
      u_{x x} + u_{y y} = 0, \quad u(x,0) = f(x) \\
      - \omega^2 \hat{u}(\omega,y) + \hat{u}_{y y}(\omega,y) = 0, 
      \quad \hat{u}(\omega,0) = \hat{f}(\omega) 
    \end{align*}
    This is an ordinary differential equation for $\hat{u}$ in which $\omega$ 
    is a parameter.  The general solution is
    \[
    \hat{u} = c_1 \e^{\omega y} + c_2 \e^{- \omega y}.
    \]
    We apply the boundary conditions that $\hat{u}(\omega,0) = \hat{f}(\omega)$
    and $\hat{u} \to 0$ and $y \to \infty$.
    \[
    \hat{u}(\omega,y) = \hat{f}(\omega) \e^{- \omega y}
    \]
    We take the inverse transform using the convolution theorem.
    \[
    \boxed{
      u(x,y) = \frac{1}{2\pi} \int_{-\infty}^\infty \e^{- (x - \xi) y } f(\xi) \,d \xi
      }
    \]
  \end{enumerate}
\end{Solution}








%% Solve the Cauchy problem for the one-dimensional heat  equation.
\begin{Solution}
  \[
  u_{t} = \kappa u_{x x}, \quad u(x,0) = f(x), 
  \]
  We take the Fourier transform of the heat equation and the initial condition.
  \[
  \hat{u}_t = - \kappa \omega^2 \hat{u}, 
  \quad \hat{u}(\omega, 0) = \hat{f}(\omega)
  \]
  This is a first order ordinary differential equation which has the solution,
  \[
  \hat{u}(\omega, t) = \hat{f}(\omega) \e^{- \kappa \omega^2 t}.
  \]
  Using a table of Fourier transforms we can write this in a form that is
  conducive to applying the convolution theorem.
  \begin{gather*}
    \hat{u}(\omega, t) = \hat{f}(\omega) \mathcal{F} \left[
      \sqrt{ \frac{\pi}{\kappa t} } \e^{-x^2 / (4 \kappa t)} \right] \\
    \boxed{
      u(x, t) = \frac{1}{2 \sqrt{ \pi \kappa t} } \int_{-\infty}^\infty \e^{-(x-\xi)^2 / (4 \kappa t)}
      f(\xi)\,\dd \xi
      }
  \end{gather*}
\end{Solution}






%% Solve the Cauchy problem for the 1D heat equation with Laplace transform.
\begin{Solution}
  We take the Laplace transform of the heat equation.
  \begin{gather}
    \nonumber
    u_{t} = \kappa u_{x x}
    \\
    \nonumber
    s \hat{u} - u(x,0) = \kappa \hat{u}_{x x} 
    \\
    \label{hatphi_xx-fracsa^2hatphi=fracfxa^2}
    \hat{u}_{x x} - \frac{s}{\kappa} \hat{u} = - \frac{f(x)}{\kappa}
  \end{gather}
  The Green function problem for
  Equation~\ref{hatphi_xx-fracsa^2hatphi=fracfxa^2} is
  \[
  G'' - \frac{s}{\kappa} G = \delta(x-\xi), \quad G(\pm \infty; \xi)\ \mathrm{is bounded}.
  \]
  The homogeneous solutions that satisfy the left and right boundary
  conditions are, respectively,
  \[
  \exp \left( \frac{ \sqrt{s}{a} } x \right), \quad
  \exp \left( - \frac{ \sqrt{s}{a} } x \right).
  \]
  We compute the Wronskian of these solutions.
  \[
  W = 
  \begin{vmatrix}
    \exp \left( \frac{ \sqrt{s} }{a} x \right) & 
    \exp \left( - \frac{ \sqrt{s} }{a} x \right) 
    \\
    \frac{ \sqrt{s} }{a} \exp \left( \frac{ \sqrt{s}{a} } x \right) & 
    - \frac{ \sqrt{s} }{a} \exp \left( - \frac{ \sqrt{s}{a} } x \right)
  \end{vmatrix}
  = - 2 \sqrt{ \frac{s}{\kappa} }
  \]
  The Green function is
  \begin{gather*}
    G(x;\xi) = \frac{ \exp \left( \sqrt{ \frac{s}{\kappa} } x_< \right)
      \exp \left( - \sqrt{ \frac{s}{\kappa} } x_> \right) }
    { - 2 \sqrt{ \frac{s}{\kappa} } } 
    \\
    G(x;\xi) = - \frac{ \sqrt{\kappa} }{ 2 \sqrt{s} }
    \exp \left( - \sqrt{ \frac{s}{\kappa} } |x-\xi| \right).
  \end{gather*}
  Now we solve Equation~\ref{hatphi_xx-fracsa^2hatphi=fracfxa^2} 
  using the Green function.
  \begin{gather*}
    \hat{u}(x,s) = \int_{-\infty}^\infty - \frac{ f(\xi) }{ \kappa } G(x;\xi) \,\dd \xi 
    \\
    \hat{u}(x,s) = \frac{1}{2 \sqrt{\kappa s}} \int_{-\infty}^{\infty}
    f(\xi) \exp \left(- \sqrt{ \frac{s}{\kappa} } | x - \xi| \right) \,\dd \xi
  \end{gather*}
  Finally we take the inverse Laplace transform to obtain the 
  solution of the heat equation.
  \[
  \boxed{
    u(x,t) = \frac{1}{2  \sqrt{\pi \kappa t} } \int_{-\infty}^{\infty}
    f(\xi) \exp\left( - \frac{(x-\xi)^2}{4 \kappa t} \right) \,\dd \xi
    }
  \]
\end{Solution}






%% Cauchy problem for the 1D heat equation.  Method of images.
\begin{Solution}
  \begin{enumerate}
    %%
    %%
  \item
    Clearly the solution satisfies the differential equation.  We must
    verify that it satisfies the boundary condition, $\phi(0,t) = 0$.
    \begin{gather*}
      \phi(x,t) = \frac{1}{2 a \sqrt{\pi t} } \int_{-\infty}^{\infty}
      f(\xi) \exp\left( - \frac{(x-\xi)^2}{4 a^2 t} \right) \,d\xi \\
      \phi(x,t) = \frac{1}{2 a \sqrt{\pi t} } \int_{-\infty}^{0}
      f(\xi) \exp\left( - \frac{(x-\xi)^2}{4 a^2 t} \right) \,d\xi 
      + \frac{1}{2 a \sqrt{\pi t} } \int_{0}^{\infty}
      f(\xi) \exp\left( - \frac{(x-\xi)^2}{4 a^2 t} \right) \,d\xi \\
      \phi(x,t) = \frac{1}{2 a \sqrt{\pi t} } \int_0^{\infty}
      f(-\xi) \exp\left( - \frac{(x+\xi)^2}{4 a^2 t} \right) \,d\xi 
      + \frac{1}{2 a \sqrt{\pi t} } \int_{0}^{\infty}
      f(\xi) \exp\left( - \frac{(x-\xi)^2}{4 a^2 t} \right) \,d\xi \\
      \phi(x,t) = - \frac{1}{2 a \sqrt{\pi t} } \int_0^{\infty}
      f(\xi) \exp\left( - \frac{(x+\xi)^2}{4 a^2 t} \right) \,d\xi 
      + \frac{1}{2 a \sqrt{\pi t} } \int_{0}^{\infty}
      f(\xi) \exp\left( - \frac{(x-\xi)^2}{4 a^2 t} \right) \,d\xi \\
      \phi(x,t) = \frac{1}{2 a \sqrt{\pi t} } \int_0^{\infty}
      f(\xi) \left( \exp\left( - \frac{(x-\xi)^2}{4 a^2 t} \right)
        \exp\left( - \frac{(x+\xi)^2}{4 a^2 t} \right) \right) \,d\xi \\
      \phi(x,t) = \frac{1}{2 a \sqrt{\pi t} } \int_0^{\infty}
      f(\xi) \exp\left( - \frac{x^2 + \xi^2}{4 a^2 t} \right) \left( 
        \exp\left( \frac{x \xi}{2 a^2 t} \right)
        - \exp\left( - \frac{x \xi}{2 a^2 t} \right) \right) \,d\xi \\
      \boxed{
        \phi(x,t) = \frac{1}{a \sqrt{\pi t} } \int_0^{\infty}
        f(\xi) \exp\left( - \frac{x^2 + \xi^2}{4 a^2 t} \right) 
        \sinh \left( \frac{x \xi}{2 a^2 t} \right) \,d\xi
        }
    \end{gather*}
    Since the integrand is zero for $x = 0$, the solution
    satisfies the boundary condition there.
    %%
    %%
  \item
    For the boundary condition $\phi_x(0,t) = 0$ we would choose $f(x)$
    to be even.  $f(-x) = f(x)$.  The solution is
    \[
    \boxed{
      \phi(x,t) = \frac{1}{a \sqrt{\pi t} } \int_0^{\infty}
      f(\xi) \exp\left( - \frac{x^2 + \xi^2}{4 a^2 t} \right) 
      \cosh \left( \frac{x \xi}{2 a^2 t} \right) \,d\xi
      }
    \]
    The derivative with respect to $x$ is
    \[
    \phi_x(x,t) = \frac{1}{2 a^3 \sqrt{\pi} t^{3/2} } \int_0^{\infty}
    f(\xi) \exp\left( - \frac{x^2 + \xi^2}{4 a^2 t} \right) 
    \left( \xi \sinh \left( \frac{x \xi}{2 a^2 t} \right)
      - x \cosh \left( \frac{x \xi}{2 a^2 t} \right) \right) \,d\xi.
    \]
    Since the integrand is zero for $x = 0$, the solution
    satisfies the boundary condition there.
  \end{enumerate}
\end{Solution}






%% Solve the Cauchy problem for the one-dimensional wave equation.
\begin{Solution}
  \index{wave equation!D'Alembert's solution}
  \index{wave equation!Fourier transform solution}
  \[
  u_{t t} = c^2 u_{x x}, \quad u(x,0) = f(x), \quad u_t(x,0) = g(x),
  \]
  With the change of variables
  \[
  \tau = c t, \quad 
  \frac{\partial}{\partial \tau} = \frac{\partial t}{\partial \tau} \frac{\partial}{\partial t} = \frac{1}{c} \frac{\partial}{\partial t},
  \quad v(x,\tau) = u(x,t),
  \]
  the problem becomes
  \[
  v_{\tau\tau} =  v_{x x}, 
  \quad v(x,0) = f(x), 
  \quad v_\tau(x,0) = \frac{1}{c} g(x).
  \]
  (This change of variables isn't necessary, it just gives us fewer 
  constants to carry around.)
  We take the Fourier transform in $x$ of the equation and the initial 
  conditions, (we consider $\tau$ to be a parameter).
  \[
  \hat{v}_{\tau\tau}(\omega,\tau) = -\omega^2 \hat{v}(\omega,\tau), \quad
  \hat{v}(\omega,\tau) = \hat{f}(\omega), \quad
  \hat{v}_\tau(\omega,\tau) = \frac{1}{c} \hat{g}(\omega)
  \]
  Now we have an ordinary differential equation for $\hat{v}(\omega,\tau)$, 
  (now we consider $\omega$ to be a parameter).  The general solution of this 
  constant coefficient differential equation is,
  \[
  \hat{v}(\omega,\tau) = a(\omega) \cos(\omega \tau) + b(\omega) \sin(\omega \tau),
  \]
  where $a$ and $b$ are constants that depend on the parameter $\omega$.
  We applying the initial conditions to obtain $\hat{v}(\omega,\tau)$.
  \[
  \hat{v}(\omega,\tau) = \hat{f}(\omega) \cos(\omega \tau) 
  + \frac{1}{c \omega} \hat{g}(\omega) \sin(\omega \tau)
  \]
  With the Fourier transform pairs
  \begin{gather*}
    \mathcal{F}[\pi ( \delta(x+\tau) + \delta(x-\tau))] = \cos(\omega \tau), 
    \\
    \mathcal{F}[\pi ( H(x+\tau) - H(x-\tau) )] = \frac{\sin(\omega \tau)}{\omega},
  \end{gather*}
  we can write $\hat{v}(\omega,\tau)$ in a form that is conducive to applying
  the Fourier convolution theorem.
  \[
  \hat{v}(\omega,\tau) = 
  \mathcal{F}[f(x)] \mathcal{F}[\pi ( \delta(x+\tau) + \delta(x-\tau))] +
  \frac{1}{c} \mathcal{F}[g(x)] \mathcal{F}[\pi ( H(x+\tau) - H(x-\tau) )]
  \]
  \begin{multline*}
    v(x,\tau) = 
    \frac{1}{2\pi} \int_{-\infty}^\infty f(\xi) 
    \pi ( \delta(x-\xi+\tau)+\delta(x-\xi-\tau)) \,d\xi \\
    + \frac{1}{c} \frac{1}{2\pi} \int_{-\infty}^\infty g(\xi) \pi (H(x-\xi+\tau)-H(x-\xi-\tau))
    \,d\xi
  \end{multline*}
  \[
  v(x,\tau) = \frac{1}{2} (f(x+\tau) + f(x-\tau))
  + \frac{1}{2c} \int_{x-\tau}^{x+\tau} g(\xi) \,d\xi
  \]
  Finally we make the change of variables $t = \tau/c$, $u(x,t)=v(x,\tau)$
  to obtain D'Alembert's solution of the wave equation,
  \[
  \boxed{
    u(x,t) = \frac{1}{2} (f(x - ct) + f(x + ct))
    + \frac{1}{2c} \int_{x-ct}^{x+ct} g(\xi)\,d\xi.
    }
  \]
\end{Solution}



\begin{Solution}
  \index{wave equation!Laplace transform solution}
  With the change of variables
  \[
  \tau = c t, \quad 
  \frac{\partial}{\partial \tau} = \frac{\partial t}{\partial \tau} \frac{\partial}{\partial t} = \frac{1}{c} \frac{\partial}{\partial t},
  \quad v(x,\tau) = u(x,t),
  \]
  the problem becomes
  \[
  v_{\tau\tau} =  v_{xx}, \quad v(x,0) = f(x), 
  \quad v_\tau(x,0) = \frac{1}{c} g(x).
  \]
  We take the Laplace transform in $\tau$ of the equation,
  (we consider $x$ to be a parameter),
  \[
  s^2 V(x,s) - s v(x,0) - v_\tau(x,0) = V_{xx}(x,s),
  \]
  \[
  V_{xx}(x,s) - s^2 V(x,s) = - s f(x) - \frac{1}{c} g(x),
  \]
  Now we have an ordinary differential equation for $V(x,s)$, 
  (now we consider $s$ to be a parameter).  We impose the boundary conditions
  that the solution is bounded at $x=\pm \infty$.  Consider the Green's 
  function problem
  \[
  g_{xx}(x;\xi) -s^2 g(x;\xi) = \delta(x-\xi), \quad
  g(\pm \infty;\xi)\ \mathrm{bounded}.
  \]
  $\e^{s x}$ is a homogeneous solution that is bounded at $x=-\infty$.
  $\e^{-s x}$ is a homogeneous solution that is bounded at $x=+\infty$.
  The Wronskian of these solutions is 
  \[
  W(x) = 
  \begin{vmatrix}
    \e^{s x} & \e^{-s x} \\
    s \e^{s x} & -s \e^{-s x}
  \end{vmatrix}
  = -2s.
  \]
  Thus the Green's function is
  %% CONTINUE: reference Result 7.4.4
  \[
  g(x;\xi)=
  \begin{cases}
    -\frac{1}{2s} \e^{s x} \e^{-s \xi} &\mathrm{for}\ x<\xi, \\
    -\frac{1}{2s} \e^{s \xi} \e^{-s x} &\mathrm{for}\ x>\xi, 
  \end{cases}
  = -\frac{1}{2s} \e^{-s|x-\xi|}.
  \]
  The solution for $V(x,s)$ is
  \[
  V(x,s) = -\frac{1}{2s} \int_{-\infty}^\infty \e^{-s|x-\xi|} 
  (-s f(\xi) - \frac{1}{c} g(\xi)) \,d\xi,
  \]
  \[
  V(x,s) = \frac{1}{2} \int_{-\infty}^\infty \e^{-s|x-\xi|} f(\xi) \,d\xi +
  \frac{1}{2c s} \int_{-\infty}^\infty \e^{-s|x-\xi|} g(\xi)) \,d\xi,
  \]
  \[
  V(x,s) = \frac{1}{2} \int_{-\infty}^\infty \e^{-s|\xi|} f(x-\xi) \,d\xi +
  \frac{1}{2c} \int_{-\infty}^\infty \frac{\e^{-s|\xi|}}{s} g(x-\xi)) \,d\xi.
  \]
  Now we take the inverse Laplace transform and interchange the order of 
  integration.
  \[
  v(x,\tau) = \frac{1}{2} \mathcal{L}^{-1}\left[ 
    \int_{-\infty}^\infty \e^{-s|\xi|} f(x-\xi) \,d\xi \right] +
  \frac{1}{2c} \mathcal{L}^{-1}\left[ 
    \int_{-\infty}^\infty \frac{\e^{-s|\xi|}}{s} g(x-\xi)) \,d\xi \right]
  \]
  \[
  v(x,\tau) = \frac{1}{2} \int_{-\infty}^\infty \mathcal{L}^{-1}\left[\e^{-s|\xi|}\right] 
  f(x-\xi) \,d\xi +
  \frac{1}{2c} \int_{-\infty}^\infty \mathcal{L}^{-1}\left[\frac{\e^{-s|\xi|}}{s}\right] 
  g(x-\xi)) \,d\xi
  \]
  \[
  v(x,\tau) = \frac{1}{2} \int_{-\infty}^\infty \delta(\tau-|\xi|) f(x-\xi) \,d\xi +
  \frac{1}{2c} \int_{-\infty}^\infty H(\tau-|\xi|) g(x-\xi)) \,d\xi
  \]
  \[ 
  v(x,\tau) = \frac{1}{2}(f(x-\tau)+f(x+\tau)) + 
  \frac{1}{2c} \int_{-\tau}^{\tau} g(x-\xi) \,d\xi
  \]
  \[ 
  v(x,\tau) = \frac{1}{2}(f(x-\tau)+f(x+\tau)) + 
  \frac{1}{2c} \int_{-x-\tau}^{-x+\tau} g(-\xi) \,d\xi
  \]
  \[ 
  v(x,\tau) = \frac{1}{2}(f(x-\tau)+f(x+\tau)) + 
  \frac{1}{2c} \int_{x-\tau}^{x+\tau} g(\xi) \,d\xi
  \]
  Now we write make the change of variables $t = \tau/c$, $u(x,t)=v(x,\tau)$
  to obtain D'Alembert's solution of the wave equation,
  \[
  \boxed{
    u(x,t) = \frac{1}{2} (f(x - ct) + f(x + ct))
    + \frac{1}{2c} \int_{x-ct}^{x+ct} g(\xi)\,d\xi.
    }
  \]
\end{Solution}




%% Heat eqn.  x > 0, t > 0.  \phi(x,0) = 0.  \phi(0,t) = f(t).
\begin{Solution}
  \begin{enumerate}
    %%
    %%
  \item
    We take the Laplace transform of Equation~\ref{1:phi_t=a^2phi_xx}.
    \begin{gather}
      \nonumber
      s \hat{\phi} - \phi(x,0) = a^2 \hat{\phi}_{x x} \\
      \label{hatphi_xx-fracsa^2hatphi=0}
      \hat{\phi}_{x x} - \frac{s}{a^2} \hat{\phi} = 0
    \end{gather}
    We take the Laplace transform of the initial condition, 
    $\phi(0,t) = f(t)$, and use that $\hat{\phi}(x,s)$ vanishes as
    $x \to \infty$ to obtain boundary conditions for $\hat{\phi}(x,s)$.
    \[
    \hat{\phi}(0,s) = \hat{f}(s), \quad
    \hat{\phi}(\infty,s) = 0
    \]
    The solutions of Equation~\ref{hatphi_xx-fracsa^2hatphi=0} are
    \[
    \exp \left( \pm \frac{\sqrt{s}}{a} x \right).
    \]
    The solution that satisfies the boundary conditions is
    \[
    \hat{\phi}(x,s) = \hat{f}(s) \exp \left( -\frac{\sqrt{s}}{a} x \right).
    \]
    We write this as the product of two Laplace transforms.
    \[
    \hat{\phi}(x,s) = \hat{f}(s) 
    \mathcal{L} \left[ \frac{ x }{ 2 a \sqrt{\pi} t^{3/2} }
      \exp \left( - \frac{ x^2 }{ 4 a^2 t } \right) \right]
    \]
    We invert using the convolution theorem.
    \[
    \boxed{
      \phi(x,t) = \frac{x}{2 a \sqrt{\pi}} \int_0^t f(t - \tau) 
      \frac{1}{\tau^{3/2}} \exp \left( - \frac{x^2}{4 a^2 \tau} 
      \right) \,d\tau.
      }
    \]
    %%
    %%
  \item
    Consider the case $f(t) = 1$.
    \begin{gather*}
      \phi(x,t) = \frac{x}{2 a \sqrt{\pi}} \int_0^t 
      \frac{1}{\tau^{3/2}} \exp \left( - \frac{x^2}{4 a^2 \tau} 
      \right) \,d\tau \\
      \xi = \frac{x}{2 a \sqrt{\tau}}, \quad
      d\xi = - \frac{x}{4 a \tau^{3/2}} \\
      \phi(x,t) = - \frac{2}{\sqrt{\pi}} \int_\infty^{x/(2 a \sqrt{t})}
      \e^{-\xi^2} \,d \xi \\
      \boxed{
        \phi(x,t) = \erfc \left( \frac{ x }{ 2 a \sqrt{t} } \right)
        }
    \end{gather*}
    Now consider the case in which $f(t) = 1$ for $0 < t < T$, 
    with $ f(t) = 0$ for $ t > T$.  For $t < T$, $\phi$ is the same 
    as before.
    \[
    \phi(x,t) = \erfc \left( \frac{ x }{ 2 a \sqrt{t} } \right), \quad
    \mathrm{for}\ 0 < t < T
    \]
    Consider $t > T$.
    \begin{gather*}
      \phi(x,t) = \frac{x}{2 a \sqrt{\pi}} \int_{t-T}^t 
      \frac{1}{\tau^{3/2}} \exp \left( - \frac{x^2}{4 a^2 \tau} 
      \right) \,d\tau \\
      \phi(x,t) = - \frac{2}{\sqrt{\pi}} \int_{x/(2 a \sqrt{t-T})}^{x/(2 a \sqrt{t})}
      \e^{-\xi^2} \,d \xi \\
      \boxed{
        \phi(x,t) = \erf \left( \frac{ x }{ 2 a \sqrt{t-T} } \right)
        - \erf \left( \frac{ x }{ 2 a \sqrt{t} } \right)
        }
    \end{gather*}
  \end{enumerate}
\end{Solution}





%% Radiating half space problem.
\begin{Solution}
  \begin{gather*}
    u_t = \kappa u_{x x}, \quad x > 0, \quad t > 0, \\
    u_x(0,t) - \alpha u(0,t) = 0, \quad u(x,0) = f(x).
  \end{gather*}
  First we find the partial differential equation that $v$ satisfies.  We 
  start with the partial differential equation for $u$,
  \[
  u_t = \kappa u_{x x}.
  \]
  Differentiating this equation with respect to $x$ yields,
  \[
  u_{t x} = \kappa u_{x x x}.
  \]
  Subtracting $\alpha$ times the former equation from the latter yields,
  \begin{gather*}
    u_{t x} - \alpha u_t = \kappa u_{x x x} - \alpha \kappa u_{x x}, \\
    \frac{\partial}{\partial t} \left( u_x - \alpha u \right)
    = \kappa \frac{\partial^2}{\partial x^2} \left( u_x - \alpha u \right),\\
    v_t = \kappa v_{x x}.
  \end{gather*}
  Thus $v$ satisfies the same partial differential equation as $u$.  This is
  because the equation for $u$ is linear and homogeneous and $v$ is a linear
  combination of $u$ and its derivatives.  The problem for $v$ is,
  \begin{gather*}
    v_t = \kappa v_{x x}, \quad x > 0, \quad t > 0, \\
    v(0,t) = 0, \quad v(x,0) = f'(x) - \alpha f(x).
  \end{gather*}
  With this new boundary condition, we can solve the problem with the 
  Fourier sine transform.  We take the sine transform of the partial differential
  equation and the initial condition.  
  \begin{gather*}
    \hat{v}_t(\omega,t) = \kappa \left( - \omega^2 \hat{v}(\omega, t) 
      + \frac{1}{\pi} \omega v(0,t) \right), \\
    \hat{v}(\omega,0) = \mathcal{F}_s \left[ f'(x) - \alpha f(x) \right]
  \end{gather*}
  \begin{gather*}
    \hat{v}_t(\omega,t) = - \kappa \omega^2 \hat{v}(\omega, t) \\
    \hat{v}(\omega,0) = \mathcal{F}_s \left[ f'(x) - \alpha f(x) \right]
  \end{gather*}
  Now we have a first order, ordinary differential equation for $\hat{v}$.  The 
  general solution is,
  \[
  \hat{v}(\omega,t) = c \e^{-\kappa \omega^2 t}.
  \]
  The solution subject to the initial condition is,
  \[
  \hat{v}(\omega,t) = \mathcal{F}_s \left[ f'(x) - \alpha f(x) \right]
  \e^{-\kappa \omega^2 t}.
  \]
  Now we take the inverse sine transform to find $v$.  We utilize the Fourier
  cosine transform pair,
  \[
  \mathcal{F}_c^{-1} \left[ \e^{- \kappa \omega^2 t } \right]
  = \sqrt{ \frac{\pi}{\kappa t} } \e^{- x^2 / (4 \kappa t) },
  \]
  to write $\hat{v}$ in a form that is suitable for the convolution theorem.
  \[
  \hat{v}(\omega,t) = \mathcal{F}_s \left[ f'(x) - \alpha f(x) \right]
  \mathcal{F}_c \left[ \sqrt{ \frac{\pi}{\kappa t} } 
    \e^{- x^2 / (4 \kappa t) } \right]
  \]
  Recall that the Fourier sine convolution theorem is,
  \[
  \mathcal{F}_s \left[ \frac{1}{2\pi} \int_0^\infty f(\xi) \left( g(|x-\xi|) 
      - g(x+\xi) \right) \,d \xi \right] 
  = \mathcal{F}_s [f(x)] \mathcal{F}_c [g(x)].
  \]
  Thus $v(x,t)$ is
  \[
  \boxed{
    v(x,t) = \frac{1}{2 \sqrt{ \pi \kappa t } } \int_0^\infty (f'(\xi) - \alpha f(\xi))
    \left( \e^{- |x-\xi|^2 / (4 \kappa t) }
      - \e^{- (x+\xi)^2 / (4 \kappa t) } \right) \,d \xi.
    }
  \]
  With $v$ determined, we have a first order, ordinary differential equation 
  for $u$,
  \[
  u_x - \alpha u = v.
  \]
  We solve this equation by multiplying by the integrating factor and 
  integrating.
  \begin{gather*}
    \frac{\partial}{\partial x} \left( \e^{-\alpha x} u \right) = \e^{- \alpha x} v \\
    \e^{-\alpha x} u = \int^x \e^{-\alpha \xi} v(x,t) \,d\xi + c(t) \\
    u = \int^x \e^{-\alpha (\xi - x)} v(x,t) \,d\xi + \e^{\alpha x} c(t) 
  \end{gather*}
  The solution that vanishes as $x \to \infty$ is
  \[
  \boxed{
    u(x,t) = - \int_x^\infty \e^{- \alpha (\xi - x) } v(\xi, t) \,d \xi.
    }
  \]
\end{Solution}



%% Solve heat equation with sine transform.
\begin{Solution}
  \begin{align*}
    \int_0^\infty \omega \e^{-c \omega^2} \sin(\omega x) \,d \omega 
    &= - \frac{\partial}{\partial x} \int_0^\infty e^{-c \omega^2} \cos(\omega x) \,d \omega \\
    &= - \frac{1}{2} \frac{\partial}{\partial x} \int_{-\infty}^\infty e^{-c \omega^2} 
    \cos(\omega x) \,d \omega \\
    &= - \frac{1}{2} \frac{\partial}{\partial x} \int_{-\infty}^\infty e^{-c \omega^2 + \imath \omega x} 
    \,d \omega \\
    &= - \frac{1}{2} \frac{\partial}{\partial x} \int_{-\infty}^\infty e^{-c (\omega + \imath x/(2c))^2}
    \e^{-x^2/(4c)} \,d \omega \\
    &= - \frac{1}{2} \frac{\partial}{\partial x} \e^{-x^2/(4c)} \int_{-\infty}^\infty e^{-c \omega^2}
    \,d \omega \\
    &= - \frac{1}{2} \sqrt{ \frac{\pi}{c} } \frac{\partial}{\partial x} \e^{-x^2/(4c)}\\
    &= \frac{ x \sqrt{\pi} }{ 4 c^{3/2} } \e^{-x^2/(4c)}
  \end{align*}
  \begin{gather*}
    u_t = u_{x x}, \quad x > 0, \quad t > 0, \\
    u(0,t) = g(t), \quad u(x,0) = 0.
  \end{gather*}
  We take the Fourier sine transform of the partial differential equation
  and the initial condition.
  \[
  \hat{u}_t(\omega, t) = - \omega^2 \hat{u}(\omega, t) + \frac{\omega}{\pi} g(t),
  \qquad \hat{u}(\omega,0) = 0
  \]
  Now we have a first order, ordinary differential equation for 
  $\hat{u}(\omega,t)$.
  \begin{gather*}
    \frac{\partial}{\partial t} \left( \e^{\omega^2 t} \hat{u}_t(\omega, t) \right) 
    = \frac{\omega}{\pi} g(t) \e^{\omega^2 t} \\
    \hat{u}(\omega, t) = \frac{\omega}{\pi} \e^{-\omega^2 t} \int_0^t
    g(\tau) \e^{\omega^2 \tau} \,d\tau + c(\omega) \e^{-\omega^2 t}
  \end{gather*}
  The initial condition is satisfied for $c(\omega) = 0$.
  \[
  \hat{u}(\omega, t) = \frac{\omega}{\pi} \int_0^t
  g(\tau) \e^{- \omega^2 (t - \tau) } \,d\tau 
  \]
  We take the inverse sine transform to find $u$.
  \[
  u(x,t) = \mathcal{F}_s^{-1} \left[ \frac{\omega}{\pi} \int_0^t
    g(\tau) \e^{- \omega^2 (t - \tau) } \,d\tau \right]
  \]
  \[
  u(x,t) = \int_0^t g(\tau) \mathcal{F}_s^{-1} \left[ \frac{\omega}{\pi} 
    \e^{- \omega^2 (t - \tau) } \right] \,d\tau 
  \]
  \[
  u(x,t) = \int_0^t g(\tau) \frac{x}{2 \sqrt{\pi} (t-\tau)^{3/2} }
  \e^{-x^2 / (4(t-\tau))} \,d\tau 
  \]
  \[
  \boxed{
    u(x,t) = \frac{x}{2 \sqrt{\pi}} \int_0^t g(\tau) 
    \frac{ \e^{-x^2 / (4(t-\tau))} }{ (t-\tau)^{3/2} } \,d\tau 
    }
  \]
\end{Solution}








%% Steady state temperature in a semi-infinite rectangular slab.
\begin{Solution}
  The problem is
  \begin{gather*}
    u_{x x} + u_{y y} = 0, \quad 0 < x, 0 < y < 1, \\
    u(x,0) = u(x,1) = 0, \quad u(0,y) = f(y).
  \end{gather*}
  We take the Fourier sine transform of the partial differential equation
  and the boundary conditions.
  \begin{gather*}
    -\omega^2 \hat{u}(\omega, y) + \frac{k}{\pi} u(0,y) + \hat{u}_{y y}(\omega,y)
    = 0 \\
    \hat{u}_{y y}(\omega,y) -\omega^2 \hat{u}(\omega, y) = -  \frac{k}{\pi} f(y),
    \quad \hat{u}(\omega,0) = \hat{u}(\omega,1) = 0 
  \end{gather*}
  This is an inhomogeneous, ordinary differential equation that we can solve 
  with Green functions.  The homogeneous solutions are
  \[
  \{ \cosh(\omega y), \sinh( \omega y) \}.
  \]
  The homogeneous solutions that satisfy the left and right boundary conditions
  are
  \[
  y_1 = \sinh(\omega y), \quad y_2 = \sinh(\omega(y-1)).
  \]
  The Wronskian of these two solutions is,
  \begin{align*}
    W(x)    &= \begin{vmatrix} \sinh(\omega y) & \sinh(\omega(y-1)) \\
      \omega \cosh(\omega y) & \omega \cosh(\omega(y-1)) 
    \end{vmatrix} \\
    &= \omega \left( \sinh(\omega y) \cosh(\omega(y-1))
      - \cosh(\omega y) \sinh(\omega(y-1)) \right) \\
    &= \omega \sinh(\omega ).
  \end{align*}
  The Green function is
  \[
  G(y|\eta) = \frac{ \sinh(\omega y_<) \sinh( \omega(y_> -1)) }
  { \omega \sinh( \omega ) }.
  \]
  The solution of the ordinary differential equation for $\hat{u}(\omega,y)$ is
  \begin{align*}
    \hat{u}(\omega,y)
    &= - \frac{\omega}{\pi} \int_0^1 f(\eta) G(y|\eta) \,d\eta \\
    &= - \frac{1}{\pi} \int_0^y f(\eta) 
    \frac{ \sinh(\omega \eta) \sinh(\omega(y-1)) }
    { \sinh(\omega) } \,d \eta
    - \frac{1}{\pi} \int_y^1 f(\eta) 
    \frac{ \sinh(\omega y) \sinh(\omega(\eta-1)) }
    { \sinh(\omega) } \,d \eta.
  \end{align*}
  With some uninteresting grunge, you can show that,
  \[
  2 \int_0^\infty \frac{ \sinh(\omega \eta) \sinh( \omega(y-1)) }{ \sinh(\omega) }
  \sin( \omega x ) \,d \omega
  = - 2 \frac{ \sin( \pi \eta ) \sin( \pi y ) }
  { ( \cosh(\pi x) - \cos(\pi(y-\eta)))(\cosh(\pi x)-\cos(\pi(y+\eta)))}.
  \]
  Taking the inverse Fourier sine transform of $\hat{u}(\omega, y)$ and 
  interchanging the order of integration yields,
  \begin{multline*}
    u(x,y) = \frac{2}{\pi} \int_0^y f(\eta) \frac{ \sin( \pi \eta ) \sin( \pi y ) }
    { (\cosh(\pi x) - \cos(\pi(y-\eta)))(\cosh(\pi x)-\cos(\pi(y+\eta)))}\,d \eta \\
    + \frac{2}{\pi} \int_y^1 f(\eta) \frac{ \sin( \pi y ) \sin( \pi \eta ) }
    { ( \cosh(\pi x) - \cos(\pi(\eta-y)))(\cosh(\pi x)-\cos(\pi(\eta+y)))}\,d \eta.
  \end{multline*}
  \[
  \boxed{
    u(x,y) = \frac{2}{\pi} \int_0^1 f(\eta)
    \frac{ \sin( \pi \eta ) \sin( \pi y ) }
    { ( \cosh(\pi x) - \cos(\pi(y-\eta)))(\cosh(\pi x)-\cos(\pi(y+\eta)))}\,d \eta
    }
  \]
\end{Solution}



%% Potential equation in the upper half plane.
\begin{Solution}
  The problem for $u(x,y)$ is,
  \begin{gather*}
    u_{x x} + u_{y y} = 0, \quad -\infty < x < \infty, y > 0, \\
    u(x,0) = g(x).
  \end{gather*}
  We take the Fourier transform of the partial differential equation and the
  boundary condition.
  \[
  -\omega^2 \hat{u}(\omega, y) + \hat{u}_{y y}(\omega, y) = 0, \quad
  \hat{u}(\omega, 0) = \hat{g}(\omega).
  \]
  This is an ordinary differential equation for $\hat{u}(\omega,y)$.  So 
  far we only have one boundary condition.  In order that $u$ is bounded
  we impose the second boundary condition $\hat{u}(\omega, y)$ is bounded as
  $y \to \infty$.  The general solution of the differential equation is
  \[
  \hat{u}(\omega,y) = \begin{cases}
    c_1(\omega) \e^{\omega y} + c_2(\omega) \e^{- \omega y}, 
    &\mathrm{for}\ \omega \neq 0, \\
    c_1(\omega) + c_2(\omega) y,
    &\mathrm{for}\ \omega = 0. 
  \end{cases}
  \]
  Note that $\e^{\omega y}$ is the bounded solution for $\omega < 0$, $1$ is
  the bounded solution for $\omega = 0$ and $\e^{- \omega y}$ is the 
  bounded solution for $\omega > 0$.  Thus the bounded solution is
  \[
  \hat{u}(\omega,y) = c(\omega) \e^{-|\omega| y}.
  \]
  The boundary condition at $y = 0$ determines the constant of integration.
  \[
  \hat{u}(\omega,y) = \hat{g}(\omega) \e^{-|\omega| y}
  \]
  Now we take the inverse Fourier transform to obtain the solution for 
  $u(x,y)$.  To do this we use the Fourier transform pair,
  \[
  \mathcal{F} \left[ \frac{2 c}{x^2 + c^2} \right] = \e^{-c |\omega|},
  \]
  and the convolution theorem,
  \[
  \mathcal{F} \left[ \frac{1}{2\pi} \int_{-\infty}^\infty f(\xi) g(x-\xi) \,d\xi \right]
  = \hat{f}(\omega) \hat{g}(\omega).
  \]
  \[
  \boxed{
    u(x,y) = \frac{1}{2\pi} \int_{-\infty}^\infty g(\xi) \frac{2 y}{ (x-\xi)^2 + y^2}\,d\xi.
    }
  \]
\end{Solution}



%% Bounded solution of $u_t = \kappa u_{x x} - \a^2 u$.
\begin{Solution}
  Since the derivative of $u$ is specified at $x = 0$, we take the cosine 
  transform of the partial differential equation and the initial condition.
  \begin{gather*}
    \hat{u}_t(\omega,t) = \kappa \left( - \omega^2 \hat{u}(\omega,t) 
      - \frac{1}{\pi} u_x(0,t) \right) - a^2 \hat{u}(\omega,t), \quad
    \hat{u}(\omega,0) = 0 \\
    \hat{u}_t + \left( \kappa \omega^2 + a^2 \right) \hat{u} 
    = \frac{\kappa}{\pi} f(t), \quad \hat{u}(\omega,0) = 0
  \end{gather*}
  This first order, ordinary differential equation for $\hat{u}(\omega,t)$
  has the solution,
  \[
  \hat{u}(\omega,t) = \frac{\kappa}{\pi} \int_0^t \e^{- (\kappa \omega^2 + a^2)
    (t - \tau) } f(\tau) \,d\tau.
  \]
  We take the inverse Fourier cosine transform to find the solution $u(x,t)$.
  \begin{gather*}
    u(x,t) = \frac{\kappa}{\pi} \mathcal{F}_c^{-1} \left[        
      \int_0^t \e^{- (\kappa \omega^2 + a^2)
        (t - \tau) } f(\tau) \,d\tau \right] \\
    u(x,t) = \frac{\kappa}{\pi} 
    \int_0^t \mathcal{F}_c^{-1} \left[ \e^{- \kappa \omega^2
        (t - \tau) } \right] \e^{-a^2 (t - \tau)} f(\tau) \,d\tau \\
    u(x,t) = \frac{\kappa}{\pi} 
    \int_0^t \sqrt{ \frac{\pi}{\kappa (t-\tau)} } 
    \e^{- x^2 / (4 \kappa (t - \tau)) }
    \e^{-a^2 (t - \tau)} f(\tau) \,d\tau 
  \end{gather*}
  \[
  \boxed{
    u(x,t) = \sqrt{ \frac{\kappa}{\pi} } \int_0^t
    \frac{ \e^{- x^2 / (4 \kappa (t - \tau)) - a^2 (t - \tau)} }
    { \sqrt{t-\tau} }
    f(\tau) \,d\tau 
    }
  \]
\end{Solution}







%% Finite string, Laplace transform
\begin{Solution}
  Mathematically stated we have
  \begin{gather*}
    u_{t t} = c^2 u_{x x}, \quad 0 < x < L, \quad t > 0, \\
    u(x,0) = u_t(x,0) = 0, \\
    u(0,t) = f(t), \quad u(L,t) = 0.
  \end{gather*}
  We take the Laplace transform of the partial differential equation and the
  boundary conditions.
  \begin{gather*}
    s^2 \hat{u}(x,s) - s u(x,0) - u_t(x,0) = c^2 \hat{u}_{x x}(x,s) \\
    \hat{u}_{x x} = \frac{s^2}{c^2} \hat{u}, \quad \hat{u}(0,s) = \hat{f}(s),
    \quad \hat{u}(L,s) = 0
  \end{gather*}
  Now we have an ordinary differential equation.  A set of solutions is
  \[
  \left\{ \cosh \left( \frac{s x}{c} \right), \sinh \left( \frac{s x}{c} \right)
  \right\}.
  \]
  The solution that satisfies the right boundary condition is
  \[
  \hat{u} = a \sinh \left( \frac{s (L-x)}{c} \right).
  \]
  The left boundary condition determines the multiplicative constant.
  \[
  \hat{u}(x,s) = \hat{f}(s) \frac{ \sinh(s(L-x)/c) }{ \sinh(s L/c) }
  \]
  If we can find the inverse Laplace transform of 
  \[
  \hat{u}(x,s) = \frac{ \sinh(s(L-x)/c) }{ \sinh(s L/c) }
  \]
  then we can use the convolution theorem to write $u$ in terms of a single 
  integral.  We proceed by expanding this function in a sum.
  \begin{align*}
    \frac{ \sinh(s(L-x)/c) }{ \sinh(s L/c) }
    &= \frac{ \e^{s(L-x)/c} - \e^{-s(L-x)/c} }
    { \e^{s L/c} - \e^{-s L/c} } \\
    &= \frac{ \e^{-s x/c} - \e^{-s (2L-x)/c} }{ 1 - \e^{-2 s L/c} } \\
    &= \left( \e^{-s x/c} - \e^{-s(2L-x)/c} \right)
    \sum_{n = 0}^\infty \e^{-2 n s L/c} \\
    &= \sum_{n = 0}^\infty \e^{-s (2 n L + x)/c} 
    - \sum_{n = 0}^\infty \e^{-s (2 (n+1) L - x)/c} \\
    &= \sum_{n = 0}^\infty \e^{-s (2 n L + x)/c} 
    - \sum_{n = 1}^\infty \e^{-s (2 n L - x)/c} 
  \end{align*}
  Now we use the Laplace transform pair:
  \[
  \mathcal{L}[\delta(x-a)] = \e^{-s a}.
  \]
  \[
  \mathcal{L}^{-1} \left[ \frac{ \sinh(s(L-x)/c) }{ \sinh(s L/c) } \right]
  = \sum_{n = 0}^\infty \delta( t - (2 n L + x)/c)
  - \sum_{n = 1}^\infty \delta( t - (2 n L - x)/c)
  \]
  We write $\hat{u}$ in the form,
  \[
  \hat{u}(x,s) = \mathcal{L}[f(t)]
  \mathcal{L} \left[ \sum_{n = 0}^\infty \delta( t - (2 n L + x)/c)
    - \sum_{n = 1}^\infty \delta( t - (2 n L - x)/c) \right].
  \]
  By the convolution theorem we have
  \[
  u(x,t) = \int_0^t f(\tau) \left(
    \sum_{n = 0}^\infty \delta( t - \tau - (2 n L + x)/c)
    - \sum_{n = 1}^\infty \delta( t - \tau - (2 n L - x)/c) \right) \,d \tau.
  \]
  We can simplify this a bit.  First we determine which Dirac delta functions 
  have their singularities in the range $\tau \in (0..t)$.  For the first 
  sum, this condition is
  \[
  0 < t - (2 n L + x) / c < t.
  \]
  The right inequality is always satisfied.  The left inequality becomes
  \begin{gather*}
    (2 n L + x) / c < t, \\
    n < \frac{c t - x}{2 L}.
  \end{gather*}
  For the second sum, the condition is
  \[
  0 < t - (2 n L - x) / c < t.
  \]
  Again the right inequality is always satisfied.  The left inequality becomes
  \[
  n < \frac{c t + x}{2 L}.
  \]
  We change the index range to reflect the nonzero contributions and do the 
  integration.
  \[
  u(x,t) = \int_0^t f(\tau) \left(
    \sum_{n=0}^{\lfloor \frac{c t - x}{2 L} \rfloor} 
    \delta( t - \tau - (2 n L + x)/c)
    \sum_{n=1}^{\lfloor \frac{c t + x}{2 L} \rfloor} 
    \delta( t - \tau - (2 n L - x)/c) \right) \,d \tau.
  \]
  \[
  \boxed{
    u(x,t) = \sum_{n=0}^{\lfloor \frac{c t - x}{2 L} \rfloor} 
    f( t - (2 n L + x)/c)
    \sum_{n=1}^{\lfloor \frac{c t + x}{2 L} \rfloor} 
    f( t - (2 n L - x)/c) 
    }
  \]
\end{Solution}










%% $\nabla^2 \phi = 0$, $0 < y < l$, $-\infty < x < \infty$.
\begin{Solution}
  We take the Fourier transform of the partial differential equation and
  the boundary conditions. 
  \[
  - \omega^2 \hat{\phi} + \hat{\phi}_{y y} = 0, \quad
  \hat{\phi}(\omega,0) = \frac{1}{2 \pi} \e^{- \imath \omega \xi}, 
  \quad \hat{\phi}(\omega,l) = 0
  \]
  We solve this boundary value problem.
  \begin{gather*}
    \hat{\phi}(\omega,y) = c_1 \cosh(\omega(l - y)) + c_2 \sinh(\omega(l - y)) \\
    \hat{\phi}(\omega,y) = \frac{1}{2 \pi} \e^{- \imath \omega \xi}
    \frac{ \sinh(\omega(l - y)) }{ \sinh(\omega l) }
  \end{gather*}
  We take the inverse Fourier transform to obtain an expression for the
  solution.
  \[
  \boxed{
    \phi(x,y) = \frac{1}{2 \pi} \int_{-\infty}^\infty \e^{\imath \omega (x - \xi)}
    \frac{ \sinh(\omega(l - y)) }{ \sinh(\omega l) } \,d \omega
    }
  \]
\end{Solution}








\raggedbottom
}
