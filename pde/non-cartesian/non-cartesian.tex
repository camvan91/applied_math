\flushbottom




%%============================================================================
%%============================================================================
\chapter{Non-Cartesian Coordinates}




%%
%% Give examples of circular, spherical, and cylindrical.
%%






%%==============================================================================
\section{Spherical Coordinates}
Writing rectangular coordinates in terms of spherical coordinates,
\begin{align*}
  x &= r \cos \theta \sin \phi \\
  y &= r \sin \theta \sin \phi \\
  z &= r \cos \phi.
\end{align*}
The Jacobian is
\begin{align*}
  &\begin{vmatrix}
    \cos\theta\sin\phi      & -r\sin\theta\sin\phi  & r\cos\theta\cos\phi \\
    \sin\theta\sin\phi      & r\cos\theta\sin\phi   & r\sin\theta\cos\phi \\
    \cos\phi                & 0                     & -r \sin \phi
  \end{vmatrix} \\
  &\qquad = r^2 \sin\phi
  \begin{vmatrix}
    \cos\theta\sin\phi      & -\sin\theta   & \cos\theta\cos\phi \\
    \sin\theta\sin\phi      & \cos\theta    & \sin\theta\cos\phi \\
    \cos\phi                & 0             & - \sin \phi
  \end{vmatrix} \\
  &\qquad = \big|r^2 \sin\phi(-\cos^2\theta\sin^2\phi - \sin^2\theta\cos^2\phi
  -\cos^2\theta\cos^2\phi - \sin^2\theta \sin^2\phi) \big| \\
  &\qquad = r^2 \sin\phi(\sin^2\phi + \cos^2\phi) \\
  &\qquad = r^2 \sin\phi.
\end{align*}
Thus we have that
\[ \iiint\limits_V f(x,y,z) \,\dd x\,\dd y\,\dd z
= \iiint\limits_V f(r,\theta,\phi)r^2 \sin\phi\,\dd r\,\dd \theta\,\dd \phi.\]




%%==============================================================================
\section{Laplace's Equation in a Disk}
Consider Laplace's equation in polar coordinates
\[ \frac{1}{r}\frac{\partial}{\partial r} \left(r\frac{\partial u}{\partial r}\right) + \frac{1}{r^2}
\frac{\partial^2u}{\partial \theta^2} = 0, \qquad 0 \leq r \leq 1 \]
subject to the the boundary conditions
\begin{enumerate}
\item $u(1,\theta) = f(\theta)$
\item $u_r(1,\theta) = g(\theta)$.
\end{enumerate}


We separate variables with $u(r,\theta) = R(r)T(\theta)$.
\begin{gather*}
  \frac{1}{r}(R' T + r R'' T) + \frac{1}{r^2} R T'' = 0 \\
  r^2\frac{R''}{R} + r \frac{R'}{R} = -\frac{T''}{T} = \lambda
\end{gather*}
Thus we have the two ordinary differential equations
\begin{gather*}
  T''+\lambda T = 0, \qquad T(0)=T(2\pi), \quad T'(0)=T'(2\pi) \\
  r^2 R'' + r R' - \lambda R = 0, \qquad R(0) < \infty.
\end{gather*}

The eigenvalues and eigenfunctions for the equation in $T$ are
\begin{gather*}
  \lambda_0 = 0, \qquad T_0 = \frac{1}{2} \\
  \lambda_n = n^2, \qquad T_n^{(1)} = \cos(n\theta), \quad
  T_n^{(2)} = \sin(n\theta)
\end{gather*}
(I chose $T_0=1/2$ so that all the eigenfunctions have the same norm.)

For $\lambda=0$ the general solution for $R$ is
\[ R = c_1 + c_2 \log r. \]
Requiring that the solution be bounded gives us
\[ R_0 = 1. \]
For $\lambda=n^2>0$ the general solution for $R$ is
\[ R = c_1 r^n + c^2 r^{-n}. \]
Requiring that the solution be bounded gives us
\[ R_n = r^n. \]
Thus the general solution for $u$ is
\[ \boxed{ u(r,\theta) = \frac{a_0}{2} + \sum_{n = 1}^\infty r^n \left[ a_n \cos(n\theta)
    + b_n \sin(n\theta) \right]. } \]

For the boundary condition $u(1,\theta)=f(\theta)$ we have the equation
\[ f(\theta) = \frac{a_0}{2} + \sum_{n = 1}^\infty  \left[ a_n \cos(n\theta)
  + b_n \sin(n\theta) \right]. \]
If $f(\theta)$ has a Fourier series then the coefficients are
\begin{align*}
  a_0 &= \frac{1}{\pi} \int_0^{2\pi} f(\theta)\,\dd \theta \\
  a_n &= \frac{1}{\pi} \int_0^{2\pi} f(\theta) \cos(n\theta)\,\dd \theta \\
  b_n &= \frac{1}{\pi} \int_0^{2\pi} f(\theta) \sin(n\theta)\,\dd \theta .
\end{align*}


For the boundary condition $u_r(1,\theta)=g(\theta)$ we have the equation
\[ g(\theta) =  \sum_{n = 1}^\infty n \left[ a_n \cos(n\theta)
  + b_n \sin(n\theta) \right]. \]
$g(\theta)$ has a series of this form only if
\[ \int_0^{2\pi} g(\theta)\,\dd \theta = 0. \]
The coefficients are
\begin{align*}
  a_n &= \frac{1}{n\pi} \int_0^{2\pi} g(\theta) \cos(n\theta)\,\dd \theta \\
  b_n &= \frac{1}{n\pi} \int_0^{2\pi} g(\theta) \sin(n\theta)\,\dd \theta .
\end{align*}












%%============================================================================
\section{Laplace's Equation in an Annulus}
Consider the problem
\[ \nabla^2 u = \frac{1}{r} \frac{\partial}{\partial r} \left( r \frac{\partial u}{\partial r} \right)
+ \frac{1}{r^2} \frac{\partial^2 u}{\partial \theta^2} = 0,
\qquad 0 \leq r < a, \quad -\pi < \theta \leq \pi, \]
with the boundary condition
\[ u(a,\theta) = \theta^2. \]

\vspace{.1in}

So far this problem only has one boundary condition.  By requiring that the
solution be finite, we get the boundary condition
\[ |u(0,\theta)| < \infty. \]
By specifying that the solution be $C^1$, (continuous and continuous first
derivative) we obtain
\[ u(r,-\pi) = u(r, \pi) \qquad \mathrm{and} \qquad
\frac{\partial u}{\partial \theta}(r, -\pi) = \frac{\partial u}{\partial \theta}(r, \pi). \]

We will use the method of separation of variables.  We seek solutions of the
form
\[ u(r,\theta) = R(r) \Theta(\theta). \]
Substituting into the partial differential equation,
\begin{gather*}
  \frac{\partial^2 u}{\partial r^2} + \frac{1}{r} \frac{\partial u}{\partial r}
  + \frac{1}{r^2} \frac{\partial^2 u}{\partial \theta^2} = 0 \\
  R'' \Theta + \frac{1}{r} R' \Theta = - \frac{1}{r^2} R \Theta'' \\
  \frac{r^2 R''}{R} + \frac{r R'}{R} = - \frac{\Theta''}{\Theta} = \lambda
\end{gather*}
Now we have the boundary value problem for $\Theta$,
\begin{gather*}
  \Theta''(\theta) + \lambda \Theta(\theta) = 0,
  \qquad -\pi < \theta \leq \pi,  \\
  \intertext{subject to}
  \Theta(-\pi) = \Theta(\pi) \qquad \mathrm{and} \qquad
  \Theta'(-\pi) = \Theta'(\pi)
\end{gather*}
We consider the following three cases for the eigenvalue, $\lambda$,
\begin{description}
\item{$\bf \boldsymbol{\lambda} \boldsymbol{<} 0$.}
  No linear combination of the solutions,
  $\Theta = \exp(\sqrt{-\lambda} \theta), \exp(-\sqrt{-\lambda} \theta)$,
  can satisfy the boundary conditions.  Thus there are no negative eigenvalues.
\item{$\bf \boldsymbol{\lambda} \boldsymbol{=} 0$.}
  The general solution solution is $\Theta = a + b \theta$.  By applying
  the boundary conditions, we get $\Theta = a$.  Thus we have the eigenvalue and
  eigenfunction,
  \[\lambda_0 = 0, \qquad  A_0 = 1. \]
\item{$\bf \boldsymbol{\lambda} \boldsymbol{>} 0$.}
  The general solution is $\Theta = a \cos (\sqrt{\lambda} \theta) +
  b \sin (\sqrt{\lambda} \theta)$.  Applying the boundary conditions yields the
  eigenvalues
  \[ \lambda_n = n^2, \quad n = 1, 2, 3, \ldots \]
  with the associated eigenfunctions
  \[ A_n = \cos (n \theta) \quad \mathrm{and} \quad B_n = \sin(n \theta).\]
\end{description}

The equation for $R$ is
\[ r^2 R'' + r R' - \lambda_n R = 0.\]
In the case $\lambda_0 = 0$, this becomes
\begin{gather*}
  R'' = - \frac{1}{r} R'  \\
  R' = \frac{a}{r} \\
  R = a \log r + b \\
  \intertext{Requiring that the solution be bounded at $r=0$ yields (to within
    a constant multiple)}
  R_0 = 1.
\end{gather*}

For $\lambda_n = n^2$, $n \geq 1$, we have
\begin{gather*}
  r^2 R'' + r R' - n^2 R = 0 \\
  \intertext{Recognizing that this is an Euler equation and making the
    substitution $R = r^\alpha$,}
  \alpha(\alpha-1) + \alpha - n^2 = 0 \\
  \alpha = \pm n \\
  R = a r^n + b r^{-n}. \\
  \intertext{requiring that the solution be bounded at $r=0$ we obtain (to within
    a constant multiple)}
  R_n = r^n
\end{gather*}

The general solution to the partial differential equation is a linear combination of the eigenfunctions
\[ u(r,\theta) = c_0 + \sum_{n=1}^\infty \left[ c_n r^n \cos n\theta +
  d_n r^n \sin n\theta \right]. \]
We determine the coefficients of the expansion with the boundary condition
\[ u(a,\theta) = \theta^2 = c_0 + \sum_{n=1}^\infty
\left[ c_n a^n \cos n\theta + d_n a^n \sin n\theta \right]. \]
We note that the eigenfunctions $1$, $\cos n\theta$, and $\sin n\theta$ are
orthogonal on $-\pi \leq \theta \leq \pi$.  Integrating the boundary condition from $-\pi$
to $\pi$ yields
\begin{gather*}
  \int_{-\pi}^\pi \theta^2\,\dd \theta = \int_{-\pi}^\pi c_0\,\dd \theta \\
  c_0 = \frac{\pi^2}{3}.
\end{gather*}
Multiplying the boundary condition by $\cos m \theta$ and integrating gives
\begin{gather*}
  \int_{-\pi}^\pi \theta^2 \cos m\theta \,\dd \theta =
  c_m a^m \int_{-\pi}^{\pi} \cos^2 m\theta\,\dd \theta \\
  c_m = \frac{(-1)^m 8 \pi}{m^2 a^m}.
\end{gather*}
We multiply by $\sin m \theta$ and integrate to get
\begin{gather*}
  \int_{-\pi}^\pi \theta^2 \sin m\theta \, d\theta =
  d_m a^m \int_{-\pi}^{\pi} \sin^2 m\theta\,\dd \theta \\
  d_m = 0
\end{gather*}
Thus the solution is
\[ \boxed{u(r,\theta) = \frac{\pi^2}{3} +
  \sum_{n=1}^\infty \frac{(-1)^n 8 \pi}{n^2 a^n} r^n \cos n \theta. } \]











\raggedbottom
