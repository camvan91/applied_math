\documentclass{letter}
\begin{document}









\begin{enumerate}

\item
% mathematica
A computer program for doing symbolic and numerical mathematics.
Cooler than Jesus.

\item
% continuous
If $\lim_{x \rightarrow \xi} y(x) = \eta$ then $y(x)$ is 
$\underline{\phantom{continuous}}$ at $x = \xi$.

\item
% removable
If $\lim_{x \rightarrow \xi} y(x)$ exists but is not equal to $y(\xi)$ then
the function has a $\underline{\phantom{removable}}$ discontinuity at that
point.

\item
% derivative
Instantaneous rate of change.

\item
% differentiable
A function is $\underline{\phantom{differentiable}}$ at a point if the 
derivative exists there.

\item
% chain
$\frac{d}{d x}(u(v(x))) = \frac{d u}{d v} \frac{d v}{d x} = u'(v(x)) v'(x)$
is the $\underline{\phantom{chain}}$ rule.

\item
% stationary
If $f'(\xi) = 0$ then $x = \xi$ is a $\underline{\phantom{stationary}}$
point.

\item
% maxima
If $f'(\xi) = 0$ and $f''(\xi) < 0$ then $x = \xi$ is a relative
$\underline{\phantom{maxima}}$.

\item
% instantaneous
If $f(x)$ is continuous in $[a..b]$ and differentiable in $(a..b)$ then
there exists a point where the $\underline{\phantom{instantaneous}}$
velocity is equal to the average velocity.

\item
% rolle
Rolle's theorem was discovered by $\underline{\phantom{Rolle}}$.

\item
% taylor
\[
f(x) = f(a) + (x-a) f'(a) + \frac{(x-a)^2}{2!} f''(a) + \cdots +
        \frac{(x-a)^n}{n!} f^{(n)}(a)
        + \frac{(x-a)^{n+1}}{(n+1)!} f^{(n+1)}(\xi).
\]
is $\underline{\phantom{Taylor}}$'s theorem.

\item
% indeterminate
$\frac{0}{0}$ is an $\underline{\phantom{indeterminate}}$ form.

\item
% indefinite
$\int f(x)\,d x$ is an $\underline{\phantom{indefinite}}$ integral.

\item
% constant
If the value of a function's derivative is identically zero then the function
is a $\underline{\phantom{constant}}$.

\item
% parts
$\int u \,d v = u v - \int v \,d u$ is integration by
$\underline{\phantom{parts}}$.

\item
% definite
$\int_a^b f(x) \,d x is a $\underline{\phantom{definite}}$ integral.

\item
% fundamental
$\int_a^b f(x) \,d x = F(b) - F(a)$ is the
$\underline{\phantom{fundamental}}$ theorem of integral calculus.

\item
% partial
The expansion of a rational function as a sum of terms with minimal 
denominators is a $\underline{\phantom{partial}}$ fraction expansion.

\item
% improper
If the range of integration is infinite or $f(x)$ is discontinuous at 
some points then $\int_a^b f(x)\,d x$ is an 
$\underline{\phantom{improper}}$ integral.

\item
% differential
A $\underline{\phantom{differential}}$ equation is an equation involving 
a function, its derivatives and independent variables.

\item
% order
$y^2 + (y')^2 = \sin x$ is a first
$\underline{\phantom{order}}$ differential equation.

\item
% homogeneous
A differential equation is $\underline{\phantom{homogeneous}}$ if it has
no terms that are functions of the independent variable alone.

\item
% general
The $\underline{\phantom{general}}$ solution of a first order differential
equation is a one parameter family of functions that satisfies the 
differential equation.

\item
% exact
The differential equation $P(x, y) + Q(x, y) y' = 0$ is 
$\underline{\phantom{exact}}$ if and only if $P_y = Q_x$.

\item
% separable
$P(x) + Q(y) y' = 0$ is a $\underline{\phantom{separable}}$ equation.

\item
% dependent
The Wronskian of a set of functions vanishes identically on an interval if 
and only if the set of functions is linearly 
$\underline{\phantom{dependent}}$ on that interval.

\item
% CONTINUE
%
$\underline{\phantom{}}$

\item
%
$\underline{\phantom{}}$

\item
%
$\underline{\phantom{}}$

\item
%
$\underline{\phantom{}}$

\item
%
$\underline{\phantom{}}$

\item
%
$\underline{\phantom{}}$

\item
%
$\underline{\phantom{}}$

\item
%
$\underline{\phantom{}}$

\item
%
$\underline{\phantom{}}$

\item
%
$\underline{\phantom{}}$

\item
%
$\underline{\phantom{}}$

\end{enumerate}



\end{document}


